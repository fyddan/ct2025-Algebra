\documentclass[12pt]{article}
\usepackage{graphicx} % Required for inserting images
\usepackage[a4paper, margin=2.5cm]{geometry}
\usepackage[T2A]{fontenc}
\usepackage[utf8]{inputenc}
\usepackage[russian]{babel}
\usepackage{amsfonts}
\usepackage{amsmath}
\usepackage{amssymb}
\usepackage{float}
\usepackage[
colorlinks=true,
linkcolor=black,
urlcolor=black,
citecolor=black
]{hyperref}


\begin{document}
	
	\begin{titlepage}
		
		\centering
		\vspace*{5cm}
		{\bfseries\LARGE Линейная алгебра КТ 2025 M3136 \par}
		\vspace*{3cm}
		\Large created by \href{https://github.com/fyddan}{fyddan}
		\\
		\vspace{3cm}
		\href{https://www.youtube.com/playlist?list=PLd7QXkfmSY7ZbGBitHpwqW7Lv5h6QGDaj}{\includegraphics[width=0.5\textwidth]{sources/label.png}}
		
	\end{titlepage}
	
	%\tableofcontents
	\newpage
	
	\section{Соответствия, композиции соответствий, ассоциативность композиции.}
	
	\subsection{Соответствия}
	\textbf{Определение:} соответствием из $A$ в $B$ называют подмножество $R \subset A \times B$, такое что $\forall \ a \in A \ \exists \ b \ \in B : (a, b) \in R$, разница с отношением в том, что тут мы требуем, чтобы каждый элемент из первого множества имел хотя бы один парный элемент из второго множества.
	\begin{figure}[h!]
		\centering
		\includegraphics[width=0.6\linewidth]{sources/relations.png}
	\end{figure}
	
	\subsection{Композиция соответствий}
	\textbf{Определение:} Пусть $R$ - соответствие из $A$ в $B$, $S$ - соответствие из $B$ в $C$, тогда композиция $S \circ R$ это такое соответствие из $A$ в $C$: $$S \circ R = \{(a, c) \in A \times C \ | \ \exists b \in B: (a, b) \in R, \ (b, c) \in S\}$$
	\begin{figure}[h!]
		\centering
		\includegraphics[width=0.6\linewidth]{sources/rel_composition.png}
	\end{figure}
	
	\subsection{Ассоциативность композиции}
	Композиция обладает свойством ассоциативности, докажем это: пусть у нас есть соответствие $R$ из $A$ в $B$, $S$ из $B$ в $C$, $T$ из $C$ в $D$, тогда сделаем предположение что $\left(T \circ S \right) \circ R$ равносильно $T \circ \left( S  \circ R \right)$, покажем что это одно и то же: 
	\\ \\ 
	$\left(T \circ S \right) \circ R = \{(a, d) \in A \times D : \exists b \in B : (a, b) \in R, (b, d) \in T\circ S\}= \{(a,d)\in A \times D:\exists b \in B, \exists c \in C: (a, b) \in R, (b, c) \in S, (c, d) \in T\}$
	\\ \\
	$T \circ \left(S  \circ R \right) = \{(a, d) \in A \times D : \exists c \in C: (a, c) \in S\circ R, (c, d) \in T\} = \{(a, d) \in A \times D : \exists b \in B,\exists c \in C:(a,b) \in R, (b, c) \in S, (c, d) \in T\}$
	\\ \\
	Заметим что правила задания множества полностью совпадают, следовательно это равные множества, что и требовалось доказать.
	
	\begin{figure}[h!]
		\centering
		\includegraphics[width=0.6\linewidth]{sources/comp_associativity.png}
	\end{figure}
	
	\newpage
	\section{Отображение множеств как частный случай соответствия. Инъекция и сюръекция. Обратимость отображения слева и справа. Совпадение левого и правого обратных.}
	\subsection{Отображения}
	\textbf{Определение:} назовем отображением соответствие $f \subset A\times B$ такое что: $\forall a \in A \ \exists ! \ b \in B: (a, b) \in f$
	\begin{figure}[h!]
		\centering
		\includegraphics[width=0.6\linewidth]{sources/mapping.png}
	\end{figure}
	\subsection{Инъекция}
	\textbf{Определение:} инъекция это такое отображение для которого выполняется: $$f(a) = f(b) \iff a = b$$
	\begin{figure}[h!]
		\centering
		\includegraphics[width=0.6\linewidth]{sources/injection.png}
	\end{figure}
	\subsection{Сюръекция}
	\textbf{Определение:} сюръекция это такое отображение для которого выполняется: $$ \forall b \in B \ \exists a \in A: f(a) = b$$
	\begin{figure}[h!]
		\centering
		\includegraphics[width=0.6\linewidth]{sources/surjection.png}
	\end{figure}
	
	\subsection{Биекция}
	\textbf{Определение:} биекция это отображение, которое и инъективно и биективно, то есть можно составить взаимно однозначное соответствие.
	
	\begin{figure}[H]
		\centering
		\includegraphics[width=0.6\linewidth]{sources/bijection.png}
	\end{figure}
	
	\subsection{Обратное отображение слева}
	\textbf{Определение:} Пусть у нас есть отображение $f: A \to B$, тогда отображение $g:B \to A$ называется обратным слева, если $g\circ f = idA$, то есть элемент из $A$ возвращается в себя же. Интересно что $f$ обязано быть инъективным, так как иначе при $a_1 \neq a_2, f(a_1) = f(a_2)$ случится противоречие, так как $g(f(a_1)) = g(f(a_2)) \implies a_1 = a_2$. Обратный слева также называют $idA$, так как он всегда возвращает свое же значение.
	\begin{figure}[h!]
		\centering
		\includegraphics[width=0.6\linewidth]{sources/reverseL.png}
	\end{figure}
	\subsection{Обратное отображение справа}
	\textbf{Определение:} Пусть у нас есть отображение $f: A \to B$, тогда отображение $g:B \to A$ называется обратным справа, если $f\circ g = idB$, то есть элемент из B возвращается в себя же. $f$ обязано быть сюръективным, то есть содержать прообразы для каждого $b \in B$, иначе мы не сможем для любого элемента из $B$ совершить путь через $A$ обратно в себя. Обратный справа также называют $idB$. 
	\begin{figure}[h!]
		\centering
		\includegraphics[width=0.6\linewidth]{sources/reverseR.png}
	\end{figure}
	\subsection{Совпадение обратного правого и обратного левого}
	Из определений обратного правого и левого следует, что $f: A \to B$ одновременно инъективно и сюръективно, следовательно это биекция, тогда $g:B \to A$ это просто обратное отображение, которое можно также обозначить как $f^{-1}$.
	
	\newpage
	\section{Определение бинарной операции, моноида, группы. Единственность нейтрального и обратного элемента. Примеры моноидов и групп.}
	\subsection{Бинарная операция}
	\textbf{Определение:} Бинарной операцией $\cdot$ на множестве $M$ называется отображение из $M \times M \to M$. Например умножение на $\mathbb{N}$ это бинарная операция, а вот на $\mathbb{N} \cup \{-1\}$ уже нет, так как при умножении числа на -1 мы выйдем за пределы исходного множества.
	\subsection{Полугруппа}
	\textbf{Определение:} Полугруппой $(M, \cdot)$ называется такая алгебраическая структура, в которой соблюдается ассоциативность, то есть: $m_1 \cdot (m_2 \cdot m_3) = (m_1 \cdot m_2) \cdot m_3$, пример - $(\mathbb{N}, \cdot)$. У нас соблюдается ассоциативность.
	\subsection{Моноид}
	\textbf{Определение:} моноидом называется полугруппа, в которую добавили нейтральный элемент $e$, для которого выполняется: $e \cdot m = m = m \cdot e$, если также добавляется коммутативность, то есть $m_1 \cdot m_2 = m_2 \cdot m_1$, тогда мы называем это коммутативным (абелевым) моноидом. Пример: $(\mathbb{N} \cup \{0\}, +)$
	
	\subsection{Единственность нейтрального элемента}
	\textbf{Утверждение:} нейтральный элемент единственный, если он существует. \\
	\textbf{Доказательство:} пусть у нас все таки есть $e_1$ и $e_2$, оба нейтральные элементы, тогда $e_1 = e_2$ : $$e_1 = e_1 \cdot e_2= e_2 $$
	что и требовалось доказать.
	
	\subsection{Группа}
	\textbf{Определение:} группой называют моноид, в котором $\forall m \in M \ \exists m^{-1}: m \cdot m^{-1} = e$. То есть для каждого элемента найдется обратный, который при проведении операции с таковым вернет нейтральный элемент. Пример: $(\mathbb{Q} \setminus \{0\}, \cdot)$ или $(\mathbb{Z}, +)$.
	
	\subsection{Единственность обратного элемента}
	\textbf{Утверждение:} обратный элемент единственный, если он существует. \\
	\textbf{Доказательство:} пусть у нас есть обратные к $g$ элементы $h_1$ и $h_2$, тогда $h_1 = h_2:$  $$h_1 = h_1 \cdot (g \cdot h_2) = (h_1 \cdot g) \cdot h_2 = h_2 $$
	что и требовалось доказать.
	
	\newpage
	\section{Определение кольца. Закон нуля. Умножение на -1. Коммутативность и ассоциативность. Примеры колец.}
	
	\subsection{Кольцо}
	\textbf{Определение:} кольцом называется множество $M$ с определенными двумя бинарными операциями, например: $\cdot, +$. Так, что $(M, +)$ - абелева группа, а $(M, \cdot)$ - полугруппа. А также операции связаны законом дистрибутивности: $\forall a, b, c \in M \ a \cdot (b + c) = a \cdot b + a\cdot c, (a + b) \cdot c = a\cdot c + b\cdot c$.
	
	\subsection{Ассоциативное кольцо}
	\textbf{Определение:} ассоциативным кольцом называется то, у которого вторая операция также ассоциативна.
	
	\subsection{Кольцо с единицей} 
	\textbf{Определение:} кольцом с единицей называется то, у которого для второй операции определен нейтральный элемент.
	
	\subsection{Коммутативное кольцо} 
	\textbf{Определение:} коммутативным кольцом называется то, у которого вторая операция коммутативна.
	
	\subsection{Примеры:}
	\begin{itemize}
		\item $(\mathbb{Z}, +, \cdot)$ - коммутативное ассоциативное с единицей
		\item $(3\mathbb{Z}, +, \cdot)$ - коммутативное ассоциативное
	\end{itemize}
	
	
	\newpage
	\section{Делимость в коммутативном кольце. Делители нуля. Область целостности. Ассоциированные элементы. Критерий ассоциированности в области целостности. Сокращение в области целостности.}
	
	\subsection{Делимость}
	\textbf{Определение:} $a | b, a,b \in R$ означает $\exists c \in R \ b = a \cdot c$. 
	
	\subsection{Делители нуля}
	\textbf{Определение:} $r \neq 0 \in R$ - делитель нуля, если $\exists s \neq 0 \in R: r\cdot s = 0$ 
	
	\subsection{Область целостности}
	\textbf{Определение:} кольцо $R$ называется областью целостности, если в нем нет делителей нуля.
	
	\subsection{Ассоциированные элементы}
	\textbf{Определение:} $a, b \neq 0 \in R$ называют ассоциированными $(a \sim b)$, если $a | b$ и $b | a$.
	\\ 
	\textbf{Утверждение:} Пусть $R$ - область целостности, $a, b \in R$ и $a \sim b$, тогда $\exists u \in R^{\times}: a = u \cdot b$.
	\\
	\textbf{Доказательство:} \begin{align*}
		a \sim b &\implies 
		\begin{cases} 
			a|b \\ b|a  
		\end{cases} 
		\implies 
		\begin{cases}
			\exists c\in R: b = a \cdot c \\ 
			\exists d \in R: a = b \cdot d 
		\end{cases} \implies \\ 
		&\implies b = b \cdot c \cdot d 
		\implies b(1 - c \cdot d) = 0 \implies \\ 
		&\implies c \cdot d = 1 
		\implies c, d \in R^{\times} 
		\implies u = d
	\end{align*}
	
	\subsection{Сокращение в области целостности}
	\textbf{Утверждение:} $R$ - область целостности и $a \neq 0$, тогда $(a\cdot b = a\cdot c) \implies (b = c)$
	\\ \textbf{Доказательство:} $a\cdot(b-c) = 0 \implies b -c = 0 \iff b = c$
	
	\newpage
	\section{Алгоритм Евклида в евклидовой области. Существование НОД в
		евклидовой области. Теорема о линейном представлении НОД в евклидовой области.}
	
	\subsection{Евклидова область}
	\textbf{Определение:} область целостности $R$, в которой существует функция нормы $N:R \setminus \{0\} \to \mathbb{N}$ такая что: $N(a \cdot b) \geq N(a)$ и $\forall a, b \in R \ \exists c, r \in R, b\neq 0: a = b\cdot c + r : N(r) < N(b)$, или $r = 0$.
	
	\subsection{НОД}
	\textbf{Определение:} $R$ - коммутативное кольцо с единицей, $a, b \in R$, $gcd(a, b) = d \in R$: 1) $d|a, d|b$ 2) $\forall c \in R: c | a, c | b  \ c | d$. \\
	\textbf{Свойство:} если $gcd(a, b)$ существует, то единственный с точностью до ассоциированного. Если $gcd(a, b) = d = d'$, то $d | d'$ и $d' | d$, а если $R$ - область целостности, то $d = d' \cdot u, u \in R^{\times}$ \\
	\textbf{Доказательство:} $d | a, d | b, d'|a, d' |b \implies d' | d$, аналогично для $d$
	
	\subsection{Алгоритм Евклида}
	\textbf{Теорема:} $R$ - евклидово кольцо $a, b \in R$ и они одновременно не ноль, тогда: $$1) \exists gcd(a, b) = d \in R \ 2) \exists x, y \in R: d = ax + by $$
	\textbf{Доказательство:} 
	1 случай: $a = 0$ или $b = 0$, тогда $gcd(0, b) = b$ или $gcd(a, 0) = a$, так как $a = gcd(a, 0) = a \cdot 1 + 0 \cdot 1$ и аналогично для $b$ 
	2 случай: $a, b \neq 0$ 
	$a = q_0b + r_0 \land N(r_0) < N(b)$ 
	$b = q_1r_0 + r_1 \land N(r_1) < N(r_0)$ 
	$r_0 = q_2r_1 + r_2 \land N(r_2) < N(r_1)$ 
	$\dots$ 
	$r_{n-1} = q_{n-1}r_n + r_{n+1} \land N(r_{n+1}) < N(r_n)$ 
	на каком то шаге станет:
	$r_n = q_nr_{n + 1} \implies r_{n+2} = 0$ \\
	\textbf{Утверждение:} $r_{n+1} = gcd(a, b)$ \\
	\textbf{Доказательство:}
	\begin{enumerate}
		\item Первое условие НОД \\
		$r_{n+1} | a \land r_{n+1} | b$: 
		$r_{n+1} | r_n \implies r_{n+1} |r_{n-1} \implies \dots \implies r_{n + 1} | b \implies r_{n+1} | a$ 
		\item Существует линейная комбинация с $r_{n+1}$ \\
		$r_0 = a - q_0b$ \\
		$r_1 = b - q_1r_0 = b - q_1(a - q_0 b) = b - q_1a + q_0q_1b = -q_1a + b(1 + q_0q_1b)$, пусть коэффициент $a$ это $S_1$, а коэффициент с $b$ это $T_1$. \\
		$r_2 = S_2a + T_2b$ \\
		$\dots$ \\
		$r_{n + 1} = S_{n + 1} a + T_{n + 1}b$
		\item Также нам нужно доказать второе условие НОД \\
		$\forall c \in R: c | a, c | b \implies c | S_{n + 1}a \land c | T_{n + 1}b \implies c | r_{n + 1} \implies gcd(a, b) = r_{n + 1}$ 
	\end{enumerate}
	
	
	
	
	\newpage
	\section{Евклидова область является областью главных идеалов.}
	\subsection{Область главных идеалов}
	\textbf{Определение:} область целостности $R$ - область главных идеалов, если для любого идеала $I \subseteq R$ существует такой элемент $x \in I: (x)  = I$ 
	\subsection{Идеал}
	\textbf{Определение:} $$I - \text{идеал} \ M \iff \begin{cases}
		I \subseteq M, I\neq \varnothing \\
		\forall a,b \in I \ a + b \in I \\
		\forall m \in M \ m\cdot i \in I
	\end{cases}$$
	\subsection{Главный идеал}
	\textbf{Определение:} $$ (x) - \text{главный идеал} \ M \iff (x) = \{xm : m \in M\}$$ \\
	\textbf{Утверждение:} любое евклидовое кольцо $R$ - область главных идеалов. \\
	\textbf{Доказательство:} Пусть $I$ - идеал в $R$, $I \neq \{0\}$, $S = \{N(x): x \in I\} \implies \exists N(d)$ - наименьшая норма. Можно утверждать, что $(d) = I$:
	\begin{enumerate}
		\item левое включение: $d \in I \implies rd \in I \implies (d) \subset I$ 
		\item правое включение: Пусть $x \in I$, разделим $x$ на $d$ с остатком: $x = dq \cdot r$, нам нужно, чтобы $r = 0$. У нас допустима ситуация, что $N(r) < N(d)$, но такая ситуация на самом деле невозможна, так как $r = x - dq \implies r \in I$, но $N(d)$ - наименьшая норма для элементов из $I$, поэтому $r$ может быть только нулем, тогда $I \subset (d)$ 
	\end{enumerate}
	
	\newpage
	\section{В области главных идеалов каждая возрастающая цепочка идеалов
		стабилизируется.}
	\subsection{Теорема}
	Пусть $R$ - область главных идеалов, также у нас есть цепочка вложенных идеалов: $I_1 \subset I_2 \subset I_3 \dots$, тогда $\exists N \in \mathbb{N}: I_N = I_{N  + 1} = I_{N + 2} \dots$, то есть цепочка стабилизируется. Такое явление еще называют Нетеровым кольцом. \\
	\textbf{Доказательство:} Так как $R$ - ОГИ, каждый $I_k = (i_k)$, то есть каждый идеал главный. Допустим у нас есть $I_{\infty} = \bigcup\limits_{k = 1} I_k$, теперь покажем, что $I_{\infty}$ это тоже идеал:
	\begin{enumerate}
		\item Замкнутость по сложению \\
		Пусть $a, b \in I_{\infty}$, тогда $\exists n_1, n_2 \in \mathbb{N}: a \in I_{n_1}, b \in I_{n_2}$, тогда не умаляя общности: $a, b \in I_{n_2} \implies a + b \in I_{n_2} \in I_{\infty}$ 
		\item Замкнутость по умножению на элемент из кольца \\
		Пусть $a \in I_{\infty}$ и $r \in R$, тогда $\exists n \in \mathbb{N}: a \in I_n$, так как $I_n$ - идеал: $ar \in I_n \in I_{\infty}$
	\end{enumerate}
	Так как $I_{\infty}$ - идеал и он лежит в $R$, $\exists x \in I_{\infty}: (x) = I_{\infty}$, тогда $\exists N: x \in I_N$, но мы можем сказать, что $I_N = I_{\infty}$, тут не очевидно только левое включение: $\forall y \in I_{\infty} \implies y = sx, s \in R$, $x \in I_N \implies sx  = y \in I_N \implies I_N = I_{\infty}$, тогда начиная с $N$: $I_N = I_{N + 1} = I_{N + 2} \dots$ 
	
	\newpage
	\section{В области главных идеалов необратимый элемент раскладывается
		в произведение неприводимых.}
	\subsection{Неприводимый элемент}
	\textbf{Определение:} $r \in R \setminus R^{\times}$ - неприводимый, если из разложения $r = st$ следует, что хотя бы один из элементов обратимый, то есть $s \in R^{\times} \lor t \in R^{\times}$
	\subsection{Теорема}
	\textbf{Утверждение:} $R$ - ОГИ, $r \in R \setminus R^{\times}$, тогда существует разложение $r$ в произведение неприводимых единственное с точностью до перестановки множителей и ассоциированности.
	\\ \textbf{Доказательство:} Пусть у нас есть множество $S = \{(x): x -$ не раскладывается в произведение неприводимых, $ x \neq 0, x \in R\setminus R^{\times}\}$, мы предположим, что оно не пусто, по доказательству конечности цепочки вложенных идеалов мы можем взять максимальный элемент $(z)$, тогда мы можем сказать, что $z$ не может быть неприводимым, так как тогда оно будет иметь разложение на неприводимые $z = z$, что противоречит нашему множеству $S$. Тогда из этого следует, что $z$ - приводимый, тогда он имеет разложение $z = st, s,t \in R \setminus R^{\times}$, но тогда $z$ делится на $s$ или $t$, а это то же самое, что $(z) \subset (s)$ или $(z) \subset (t)$, но $(z)$ - минимальный по включению идеал в $S \implies (s), (t) \notin S$, тогда по определению множества $S$ $s$ и $t$ имеют разложение на неприводимые: $s = q_1 \cdot q_2 \dots q_n$, $t = p_1 \cdot p_2 \dots p_n$, но тогда $z = (q_1 \cdot q_2 \dots q_n) \cdot (p_1 \cdot p_2 \dots p_n)$, следовательно $z$ имеет разложение на неприводимые, получаем противоречие, следовательно $S = \varnothing$, тогда любой элемент в Области главных идеалов имеет разложение на неприводимые.
	
	\newpage
	\section{Основная теорема арифметики в области главных идеалов.}
	\subsection{Простой элемент}
	\textbf{Определение:} элемент $r \in R \setminus{R^{\times}}$ называется простым, если из того, что $r | ab \implies r|a \lor r|b$ 
	\subsection{Лемма}
	\textbf{Утверждение:} $R$ - ОГИ, $r$ - неприводим, тогда $r$ - простой. \\
	\textbf{Доказательство:} Пусть $r | ab$, рассмотрим все возможные линейные комбинации $(r, a)$, тогда $\exists d \in R: (d) = (r, a) \implies d | r, d | a$ , $r \in (r,a) \implies r \in (d) \implies \ \exists s \in R: r = sd \implies$ 1) $d \in R^{\times}$ 2) $r \sim d$.
	\begin{enumerate}
		\item Если $d \in R^{\times} \implies (d) = R \implies d = 1$, получается что наша линейная комбинация это $1 = xr + ya$, мы можем домножить все на $b$, тогда $b = xrb + yab$, тогда $r | xrb, r | ab \implies r | b$
		\item Если $r \sim d$, то по определению ассоциированности $r | d$, а $d | a \implies r | a$
	\end{enumerate}
	То есть для любого случая $r$ - простой
	
	\subsection{Единственность разложения на неприводимые}
	Пусть $r = p_1 \cdot p_2 \dots p_n = q_1 \cdot q_2 \dots q_n$ все элементы неприводимы, а значит простые. Рассмотрим $p_1$: $p_1 | (p_1 \cdot p_2 \dots p_n), p_1 | (q_1 \cdot q_2 \dots q_m)$. $m = n$, иначе бы какие то элементы были единицами, чтобы не нарушить равенство, тогда бы нарушилось условие, что все элементы неприводимы. Затем находим среди элементов $q_i$ такой, что $p_1 \sim q_i$, меняем местами $q_i$ и $q_1$, после всех подобных перестановок получим $p_1 \sim q_1, p_2 \sim q_2, \dots, p_n \sim q_n$.
	
	\newpage
	\section{Фактор-кольцо, корректность операций. Кольцо классов вычетов.
		Обратимые элементы в \texorpdfstring{$\mathbb{Z}/n\mathbb{Z}.$}{Z/nZ}}
	
	\subsection{Отношение эквивалентности}
	\textbf{Определение:} $\sim$ - отношение эквивалентности на множестве $M$, если $\sim \ \subset M \times M$, в котором для пары $(a, b)$ выполняется заданное условие $a \sim b$, а также отношение обладает: 1) рефлексивность, 2) симметричность, 3) транзитивность.
	\subsection{Фактор-кольцо}
	\textbf{Определение:} $R$ - кольцо, $I$ - идеал на этом кольце, тогда $R / I$ это фактор по отношению эквивалентности такому, что $a \sim b \iff a - b \in I$, то есть элементы различаются на элемент из идеала. Тогда $R/I$ это множество из классов эквивалентности $[x] = \{y \in R: x \sim y\}$ 
	\subsection{Корректность операций}
	Докажем, что $[a] + [b] = [a + b] = [a' + b'], a \sim a', b \sim b'$: Нам нужно доказать что $a + b \sim a' + b'$. $(a + b) - (a' + b') = (a - a') + (b - b') \in I$ \\
	Докажем что $[ab] = [a'b']$: $ab - a'b' = ab - ab' + ab' - a'b' = a(b - b') + b'(a - a') \in I$
	\subsection{Кольцо классов вычетов}
	\textbf{Определение:} Кольцо классов вычетов это кольцо $\mathbb{Z}$ факторизованное по главному идеалу $(n)$. Оно содержит классы эквивалентности, в которых $a \sim b \implies a = b \mod  n$. То есть классы эквивалентности содержат элементы, которые дают одинаковый остаток при делении на $n$. 
	\subsection{Обратимые элементы в \texorpdfstring{$\mathbb{Z}/n\mathbb{Z}.$}{Z/nZ}}
	Обратимыми в Z/nZ являются такие $x: gcd(x, n) = 1$, это очень легко доказывается, мы можем представить линейную комбинацию для нод: $xk + yn = 1 \mod n \implies xk = 1 \mod n \implies k$ - обратимый элемент для $x$. То есть в $\mathbb{Z} / p\mathbb{Z}$ все ненулевые элементы обратимы, соответственно это уже не просто кольцо, а поле
	
	
	\newpage
	\section{Гомоморфизм колец. Проекция на фактор - гомоморфизм. Ядро
		гомоморфизма - идеал. Образ гомоморфизма - подкольцо.}
	\subsection{Гомоморфизм колец}
	\textbf{Определение:} отображение $f: R \to S$ если:
	\begin{enumerate}
		\item $f(r_1 + r_2) = f(r_1) + f(r_2)$
		\item $f(r_1 \cdot r_2) = f(r_1) \cdot f(r_2)$
		\item $f(1_R) = 1_S$, требование единицы не является обязательным, если кольца без единицы.
	\end{enumerate}
	
	\subsection{Проекция на фактор кольца}
	\textbf{Утверждение:} Если $I$ - идеал кольца $R$, то $\pi : R \to R/I$ - гомоморфизм. \\
	\textbf{Доказательство:} $\pi(a)$ просто отправляет a в его класс эквивалентности $[a]$.
	\begin{enumerate}
		\item $\pi(a + b) = [a + b] = [a] + [b] = \pi(a) + \pi(b)$
		\item $\pi(ab) = [ab] = [a] \cdot [b] = \pi(a) \cdot \pi(b)$
		\item $\pi(1) = [1] = 1_{R/I}$
	\end{enumerate}
	
	\subsection{Ядро гомоморфизма}
	\textbf{Определение:} для гомоморфизма $f: R \to S$ ядро $\ker f = \{r \in R: f(r) = 0\}$, то есть в ядре лежат все элементы $R$, которые превратятся в ноль. \\
	\textbf{Утверждение:} $\ker f$ - идеал в кольце $R$.
	\\ \textbf{Доказательство:} 
	\begin{enumerate}
		\item $ker f \neq \varnothing$, т.к всегда в ядре находится ноль. 
		\item пусть $a, b \in \ker f$, тогда $f(a) = f(b) = 0 \implies f(a + b) = f(a) + f(b) = 0 + 0 = 0 \in \ker f$
		\item пусть $a \in \ker f$, $b \in R$, тогда $f(ra) = f(r) \cdot f(a) = f(r) \cdot 0 = 0 \in \ker f$
	\end{enumerate}
	Все три условия для того что ядро - идеал доказаны.
	\subsection{Образ гомоморфизма}
	\textbf{Определение:} образ гомоморфизма $f: R \to S$ - $Im \ f \{s \in S: \exists r \in R: f(r) = s\}$
	\\ \textbf{Утверждение:} такой образ - подкольцо $S$.
	\\ \textbf{Доказательство:}
	\begin{enumerate}
		\item $f(0_R) = f(0_R + 0_R) = f(0_R) + f(0_R) \implies f(0_R) = 0_S$
		\item Пусть $x, y \in Im \ f \implies \exists a,b \in R: x = f(a), y = f(b)$, тогда $x - y =f(a) - f(b) = f(a - b) \in Im \ f$ - замкнуто по вычитанию, значит есть для каждого обратный элемент по сложению
		\item $xy = f(a) \cdot f(b) = f(ab) \in Im \ f$ - замкнуто по умножению
	\end{enumerate}
	Следовательно $Im \ f$ - подкольцо $S$.
	
	
	\newpage
	\section{Критерий инъективности гомоморфизма колец через определение
		ядра. Гомоморфизм является изоморфизмом тттк он биекция.}
	\subsection{Критерий инъективности}
	\textbf{Утверждение:} $f : R \to S$ - гомоморфизм, тогда $f$ - инъекция $\iff \ker f = \{0\}$ \\
	\textbf{Доказательство:} 
	\begin{enumerate}
		\item достаточность: \\
		Пусть $\exists a \in \ker f: f(a) = 0$, но $f(0) = 0$, тогда по инъективности $f$ $a = 0$
		\item необходимость: \\
		Пусть $a, b \in R$ и $f(a) = f(b) \implies f(a) - f(b) = 0 \implies f(a - b) = 0 \implies$
		$\implies a - b \in \ker f \implies a - b = 0 \implies a = b \implies f -$ инъекция
	\end{enumerate}
	\subsection{Изоморфизм}
	\textbf{Определение:} Кольца $R$ и $S$ называют изоморфными если для гомоморфизма $f: R \to S$, существует гомоморфизм $g : S \to R$, такой что у $f$ есть обратный слева $g \circ f = id_R$ и обратный справа $f \circ g = id_S$.
	\subsection{Гомоморфизм являтся изоморфизмом}
	\textbf{Утверждение:} если гомоморфизм $f: R \to S$ - изоморфизм $\iff f$ - биекция. \\
	\textbf{Доказательство:} 
	\begin{enumerate}
		\item достаточность: \\
		если $f$ - изоморфизм, то по определению существует $g: S \to R$ такой, что $g \circ f = id_R \implies f$ - инъекция и $f \circ g = id_S \implies f$ - сюръекция, тогда $f$ - биекция.
		\item необходимость: \\
		если $f$ - биекция, то существует обратное отображение $f^{-1}$, нам надо доказать, что это гомоморфизм (наличие обратного левого и правого исходит из биективности $f$). Возьмем любые $y_1, y_2$ такие что $x_1 = f^{-1}(y_1), x_2 = f^{-1}(y_2)$:
		\begin{enumerate}
			\item сложение: \\
			нам нужно доказать, что $f^{-1}(y_1 + y_2) = f^{-1}(y_1) + f^{-1}(y_2)$. 
			$y_1 + y_2 = f(x_1 + x_2)$. $f^{-1}(y_1 + y_2) = f^{-1}(f(x_1 + x_2)) = x_1 + x_2 = f^{-1}(y_1) + f^{-1}(y_2)$
			\item умножение: \\
			нам нужно доказать, что $f^{-1}(y_1 \cdot y_2) = f^{-1}(y_1) \cdot f^{-1}(y_2)$. 
			$y_1 \cdot y_2 = f(x_1 \cdot x_2)$. $f^{-1}(y_1 \cdot y_2) = f^{-1}(f(x_1 \cdot x_2)) = x_1 \cdot x_2 = f^{-1}(y_1) \cdot f^{-1}(y_2)$
		\end{enumerate}
	\end{enumerate}
	Выходит, что $f^{-1}$ - гомоморфизм, тогда $f$ - изоморфизм.
	
	\newpage
	\section{Лемма о пропуске гомоморфизма через фактор-кольцо. Теорема об
		изоморфизме.}
	\subsection{Лемма}
	\textbf{Утверждение:} $f: R \to S$ - гомоморфизм, $I$ - идеал $R$, тогда $\exists \overline{f}: R/I \to S$ - гомоморфизм, такой, что $\overline{f} \circ \pi = f \iff I \subset \ker f$. \\
	\textbf{Доказательство:}
	\begin{enumerate}
		\item достаточность: \\
		$\forall a \in I \ f(a) = \overline{f}(\pi(a)) = \overline{f}(0) = 0 \implies a \in \ker f$
		\item необходимость: \\
		Пусть $[r] \in R/I$, тогда $\overline{f}([r]) = f(r), r \in [r]$. 
		Но нам необходимо доказать корректность:
		Пусть $r, r' \in [r]$, $f(r') = \overline{f}([r]) = f(r)$? Это действительно так: $f(r) - f(r') = f(r - r') \in I$ и равно нулю, по предположению, что $I \in \ker f$.
		Почему $\overline{f}$ - гомоморфизм:
		\begin{itemize}
			\item $\overline{f}([1]) = f(1) = 1$
			\item $\overline{f}([a] + [b]) = \overline{f}([a + b]) = f(a + b) = f(a) + f(b) = \overline{f}([a]) + \overline{f}([b])$
			\item $\overline{f}([a] \cdot [b]) = \overline{f}([a \cdot b]) = f(a \cdot b) = f(a) \cdot f(b) = \overline{f}([a]) \cdot \overline{f}([b])$
		\end{itemize}
	\end{enumerate}
	Также покажем коммутативность: Пусть $r \in R$, $\overline{f}(\pi(r)) = \overline{f}([r]) = f(r)$
	\subsection{Первая теорема об изоморфизме}
	\textbf{Утверждение:} $f: R \to S$ - гомоморфизм, тогда $R/\ker f \xrightarrow{\cong} Im \ f$ \\
	\textbf{Доказательство:} существует гомоморфизм $R \to Im \ f$, тогда, чтобы построить $R/\ker f \to Im \ f$ необходимо и достаточно по лемме, чтобы $\ker f \subset \ker f$, что всегда правда. Обозначим этот гомоморфизм $\overline{f}$, теперь нам надо доказать, что он биективен:
	\begin{enumerate}
		\item инъекция: \\
		$\ker f = \{[r] \in R/\ker f: \overline{f}([r]) = 0\} = \{[r] : f(r) = 0\} = \{0\}$, тогда это инъекция
		\item Сюръекция: \\
		Пусть $s \in Im \ f$, тогда существует $r \in R: f(r) = s$, тогда $\overline{f}([r]) = f(r) = s$, получается сюръекция.
	\end{enumerate}
	$\overline{f}$ - биекция, а значит и изоморфизм. 
	
	\newpage
	\section{Китайская теорема об остатках в \texorpdfstring{$\mathbb{Z}$}{Z}.}
	\textbf{Утверждение:} $gcd(m, n) = 1$, тогда $\sigma: \mathbb{Z}/mn\mathbb{Z} \xrightarrow{\cong} (\mathbb{Z}/n\mathbb{Z}) \times (\mathbb{Z}/m\mathbb{Z})$ \\
	\textbf{Доказательство:} $\sigma(k) = (k \mod n, k \mod m)$ 
	Покажем, что $\sigma$ - гомоморфизм:
	\begin{enumerate}
		\item Сложение: $\sigma(k + k') = ([k + k']_n, [k + k']_m) = ([k]_n + [k']_n, [k]_m + [k']_m) = ([k]_n, [k]_m) + ([k']_n, [k']_m) = \sigma(k) + \sigma(k')$
		\item Умножение: Умножение: $\sigma(k \cdot k') = ([k \ cdot k']_n, [k \cdot k']_m) = ([k]_n \cdot [k']_n, [k]_m \cdot [k']_m) = ([k]_n, [k]_m) \cdot ([k']_n, [k']_m) = \sigma(k) \cdot \sigma(k')$
		\item Сохраняет единицу: $\sigma([1]_{mn}) = ([1]_{mn} \mod n, [1]_{mn} \mod m) = ([1]_n, [1]_m)$
	\end{enumerate}
	Определим $\ker \sigma = \{k \in \mathbb{Z}/nm\mathbb{Z}: k = 0 \mod n \land k = 0 \mod m\}$, пусть у нас $k \in \ker \sigma$, тогда $k$ делится на $m$ и $n$, тогда $k = 0 \mod nm \implies \sigma$ - инъекция. Так как у нас одинаковое количество элементов в $\mathbb{Z}/nm\mathbb{Z}$ и $\mathbb{Z}/n\mathbb{Z} \times \mathbb{Z}/m\mathbb{Z}$ и $\sigma$ - инъекция, каждый элемент слева получает уникальный справа, при этом каждый слева имеет связь, получается левое полностью покрывает правое, тогда это сюръекция и биекция соответственно. Если $\sigma$ - биекция, то мы уже ранее доказывали, что $\sigma$ - изоморфизм.
	
	\section{Количество элементов в произведении колец. Обратимые элементы
		в произведении колец. Мультипликативность функции Эйлера.}
	
	\section{Вычисление функции Эйлера. Теорема Эйлера.}
	
	\section{Построение кольца многочленов над полем, деление с остатком. Деление на двучлен. Теорема Безу. Следствие о количестве различных
		корней многочлена.}
	
	\section{Теорема о формальном и функциональном равенстве многочленов.
		Антипример для конечного поля.}
	
	\section{Сумма, произведение и пересечение идеалов. Проверка. Связь произведения и пересечения в общем случае и в случае взаимно простых (комаксимальных) идеалов.}
	
	\section{Китайская теорема об остатках для колец.}
	
	\section{Формула Тейлора для многочлена. Критерий кратности корня.}
	
	\section{Количество корней с кратностями не больше степени многочлена.}
	
	\section{Интерполяционная формула Лагранжа, единственность.}
	
\end{document}