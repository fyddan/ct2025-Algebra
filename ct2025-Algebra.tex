\documentclass[12pt]{article}
\usepackage{graphicx} % Required for inserting images
\usepackage[a4paper, margin=2.5cm]{geometry}
\usepackage[T2A]{fontenc}
\usepackage[utf8]{inputenc}
\usepackage[russian]{babel}
\usepackage{amsfonts}
\usepackage{amsmath}
\usepackage{amssymb}
\usepackage{float}
\usepackage[
colorlinks=true,
linkcolor=black,
urlcolor=black,
citecolor=black
]{hyperref}


\begin{document}
	
	\begin{titlepage}
		
		\centering
		\vspace*{5cm}
		{\bfseries\LARGE Линейная алгебра КТ 2025 M3136 \par}
		\vspace*{3cm}
		\Large created by \href{https://github.com/fyddan}{fyddan}
		\\
		\vspace{3cm}
		\href{https://www.youtube.com/playlist?list=PLd7QXkfmSY7ZbGBitHpwqW7Lv5h6QGDaj}{\includegraphics[width=0.5\textwidth]{sources/label.png}} \\
		\vspace{3cm}
		\Large также спасибо Ерину Алексею за определения и Давыденко Тарасу за билеты по тензорам и внешним степеням
		
	\end{titlepage}
	
	\tableofcontents
	\newpage
	
	\section{Соответствия, композиции соответствий, ассоциативность композиции.}
	
	\subsection{Соответствия}
	\textbf{Определение:} соответствием из $A$ в $B$ называют подмножество $R \subset A \times B$, такое что $\forall \ a \in A \ \exists \ b \ \in B : (a, b) \in R$, разница с отношением в том, что тут мы требуем, чтобы каждый элемент из первого множества имел хотя бы один парный элемент из второго множества.
	\begin{figure}[h!]
		\centering
		\includegraphics[width=0.6\linewidth]{sources/relations.png}
	\end{figure}
	
	\subsection{Композиция соответствий}
	\textbf{Определение:} Пусть $R$ - соответствие из $A$ в $B$, $S$ - соответствие из $B$ в $C$, тогда композиция $S \circ R$ это такое соответствие из $A$ в $C$: $$S \circ R = \{(a, c) \in A \times C \ | \ \exists b \in B: (a, b) \in R, \ (b, c) \in S\}$$
	\begin{figure}[h!]
		\centering
		\includegraphics[width=0.6\linewidth]{sources/rel_composition.png}
	\end{figure}
	
	\subsection{Ассоциативность композиции}
	Композиция обладает свойством ассоциативности, докажем это: пусть у нас есть соответствие $R$ из $A$ в $B$, $S$ из $B$ в $C$, $T$ из $C$ в $D$, тогда сделаем предположение что $\left(T \circ S \right) \circ R$ равносильно $T \circ \left( S  \circ R \right)$, покажем что это одно и то же: 
	\\ \\ 
	$\left(T \circ S \right) \circ R = \{(a, d) \in A \times D : \exists b \in B : (a, b) \in R, (b, d) \in T\circ S\}= \{(a,d)\in A \times D:\exists b \in B, \exists c \in C: (a, b) \in R, (b, c) \in S, (c, d) \in T\}$
	\\ \\
	$T \circ \left(S  \circ R \right) = \{(a, d) \in A \times D : \exists c \in C: (a, c) \in S\circ R, (c, d) \in T\} = \{(a, d) \in A \times D : \exists b \in B,\exists c \in C:(a,b) \in R, (b, c) \in S, (c, d) \in T\}$
	\\ \\
	Заметим что правила задания множества полностью совпадают, следовательно это равные множества, что и требовалось доказать.
	
	\begin{figure}[h!]
		\centering
		\includegraphics[width=0.6\linewidth]{sources/comp_associativity.png}
	\end{figure}
	
	\section{Отображение множеств как частный случай соответствия. Инъекция и сюръекция. Обратимость отображения слева и справа. Совпадение левого и правого обратных.}
	\subsection{Отображения}
	\textbf{Определение:} назовем отображением соответствие $f \subset A\times B$ такое что: $\forall a \in A \ \exists ! \ b \in B: (a, b) \in f$
	\begin{figure}[h!]
		\centering
		\includegraphics[width=0.6\linewidth]{sources/mapping.png}
	\end{figure}
	\subsection{Инъекция}
	\textbf{Определение:} инъекция это такое отображение для которого выполняется: $$f(a) = f(b) \iff a = b$$
	\begin{figure}[h!]
		\centering
		\includegraphics[width=0.6\linewidth]{sources/injection.png}
	\end{figure}
	\subsection{Сюръекция}
	\textbf{Определение:} сюръекция это такое отображение для которого выполняется: $$ \forall b \in B \ \exists a \in A: f(a) = b$$
	\begin{figure}[h!]
		\centering
		\includegraphics[width=0.6\linewidth]{sources/surjection.png}
	\end{figure}
	
	\subsection{Биекция}
	\textbf{Определение:} биекция это отображение, которое и инъективно и сюръективно, то есть можно составить взаимно однозначное соответствие.
	
	\begin{figure}[H]
		\centering
		\includegraphics[width=0.6\linewidth]{sources/bijection.png}
	\end{figure}
	
	\subsection{Обратное отображение слева}
	\textbf{Определение:} Пусть у нас есть отображение $f: A \to B$, тогда отображение $g:B \to A$ называется обратным слева, если $g\circ f = idA$, то есть элемент из $A$ возвращается в себя же. Интересно что $f$ обязано быть инъективным, так как иначе при $a_1 \neq a_2, f(a_1) = f(a_2)$ случится противоречие, так как $g(f(a_1)) = g(f(a_2)) \implies a_1 = a_2$. Обратный слева также называют $idA$, так как он всегда возвращает свое же значение.
	\begin{figure}[h!]
		\centering
		\includegraphics[width=0.6\linewidth]{sources/reverseL.png}
	\end{figure}
	\subsection{Обратное отображение справа}
	\textbf{Определение:} Пусть у нас есть отображение $f: A \to B$, тогда отображение $g:B \to A$ называется обратным справа, если $f\circ g = idB$, то есть элемент из B возвращается в себя же. $f$ обязано быть сюръективным, то есть содержать прообразы для каждого $b \in B$, иначе мы не сможем для любого элемента из $B$ совершить путь через $A$ обратно в себя. Обратный справа также называют $idB$. 
	\begin{figure}[h!]
		\centering
		\includegraphics[width=0.6\linewidth]{sources/reverseR.png}
	\end{figure}
	\subsection{Совпадение обратного правого и обратного левого}
	Из определений обратного правого и левого следует, что $f: A \to B$ одновременно инъективно и сюръективно, следовательно это биекция, тогда $g:B \to A$ это просто обратное отображение, которое можно также обозначить как $f^{-1}$.
	
	\section{Определение бинарной операции, моноида, группы. Единственность нейтрального и обратного элемента. Примеры моноидов и групп.}
	\subsection{Бинарная операция}
	\textbf{Определение:} Бинарной операцией $\cdot$ на множестве $M$ называется отображение из $M \times M \to M$. Например умножение на $\mathbb{N}$ это бинарная операция, а вот на $\mathbb{N} \cup \{-1\}$ уже нет, так как при умножении числа на -1 мы выйдем за пределы исходного множества.
	\subsection{Полугруппа}
	\textbf{Определение:} Полугруппой $(M, \cdot)$ называется такая алгебраическая структура, в которой соблюдается ассоциативность, то есть: $m_1 \cdot (m_2 \cdot m_3) = (m_1 \cdot m_2) \cdot m_3$, пример - $(\mathbb{N}, \cdot)$. У нас соблюдается ассоциативность.
	\subsection{Моноид}
	\textbf{Определение:} моноидом называется полугруппа, в которую добавили нейтральный элемент $e$, для которого выполняется: $e \cdot m = m = m \cdot e$, если также добавляется коммутативность, то есть $m_1 \cdot m_2 = m_2 \cdot m_1$, тогда мы называем это коммутативным (абелевым) моноидом. Пример: $(\mathbb{N} \cup \{0\}, +)$
	
	\subsection{Единственность нейтрального элемента}
	\textbf{Утверждение:} нейтральный элемент единственный, если он существует. \\
	\textbf{Доказательство:} пусть у нас все таки есть $e_1$ и $e_2$, оба нейтральные элементы, тогда $e_1 = e_2$ : $$e_1 = e_1 \cdot e_2= e_2 $$
	что и требовалось доказать.
	
	\subsection{Группа}
	\textbf{Определение:} группой называют моноид, в котором $\forall m \in M \ \exists m^{-1}: m \cdot m^{-1} = e$. То есть для каждого элемента найдется обратный, который при проведении операции с таковым вернет нейтральный элемент. Пример: $(\mathbb{Q} \setminus \{0\}, \cdot)$ или $(\mathbb{Z}, +)$.
	
	\subsection{Единственность обратного элемента}
	\textbf{Утверждение:} обратный элемент единственный, если он существует. \\
	\textbf{Доказательство:} пусть у нас есть обратные к $g$ элементы $h_1$ и $h_2$, тогда $h_1 = h_2:$  $$h_1 = h_1 \cdot (g \cdot h_2) = (h_1 \cdot g) \cdot h_2 = h_2 $$
	что и требовалось доказать.
	
	\section{Определение кольца. Закон нуля. Умножение на -1. Коммутативность и ассоциативность. Примеры колец.}
	
	\subsection{Кольцо}
	\textbf{Определение:} кольцом называется множество $M$ с определенными двумя бинарными операциями, например: $\cdot, +$. Так, что $(M, +)$ - абелева группа, а $(M, \cdot)$ - полугруппа. А также операции связаны законом дистрибутивности: $\forall a, b, c \in M \ a \cdot (b + c) = a \cdot b + a\cdot c, (a + b) \cdot c = a\cdot c + b\cdot c$.
	
	\subsection{Ассоциативное кольцо}
	\textbf{Определение:} ассоциативным кольцом называется то, у которого вторая операция также ассоциативна.
	
	\subsection{Кольцо с единицей} 
	\textbf{Определение:} кольцом с единицей называется то, у которого для второй операции определен нейтральный элемент.
	
	\subsection{Коммутативное кольцо} 
	\textbf{Определение:} коммутативным кольцом называется то, у которого вторая операция коммутативна.
	
	\subsection{Примеры:}
	\begin{itemize}
		\item $(\mathbb{Z}, +, \cdot)$ - коммутативное ассоциативное с единицей
		\item $(3\mathbb{Z}, +, \cdot)$ - коммутативное ассоциативное
	\end{itemize}
	
	\section{Делимость в коммутативном кольце. Делители нуля. Область целостности. Ассоциированные элементы. Критерий ассоциированности в области целостности. Сокращение в области целостности.}
	
	\subsection{Делимость}
	\textbf{Определение:} $a | b, a,b \in R$ означает $\exists c \in R \ b = a \cdot c$. 
	
	\subsection{Делители нуля}
	\textbf{Определение:} $r \neq 0 \in R$ - делитель нуля, если $\exists s \neq 0 \in R: r\cdot s = 0$ 
	
	\subsection{Область целостности}
	\textbf{Определение:} кольцо $R$ называется областью целостности, если в нем нет делителей нуля.
	
	\subsection{Ассоциированные элементы}
	\textbf{Определение:} $a, b \neq 0 \in R$ называют ассоциированными $(a \sim b)$, если $a | b$ и $b | a$.
	\\ 
	\textbf{Утверждение:} Пусть $R$ - область целостности, $a, b \in R$ и $a \sim b$, тогда $\exists u \in R^{\times}: a = u \cdot b$.
	\\
	\textbf{Доказательство:} \begin{align*}
		a \sim b &\implies 
		\begin{cases} 
			a|b \\ b|a  
		\end{cases} 
		\implies 
		\begin{cases}
			\exists c\in R: b = a \cdot c \\ 
			\exists d \in R: a = b \cdot d 
		\end{cases} \implies \\ 
		&\implies b = b \cdot c \cdot d 
		\implies b(1 - c \cdot d) = 0 \implies \\ 
		&\implies c \cdot d = 1 
		\implies c, d \in R^{\times} 
		\implies u = d
	\end{align*}
	
	\subsection{Сокращение в области целостности}
	\textbf{Утверждение:} $R$ - область целостности и $a \neq 0$, тогда $(a\cdot b = a\cdot c) \implies (b = c)$
	\\ \textbf{Доказательство:} $a\cdot(b-c) = 0 \implies b -c = 0 \iff b = c$
	
	\section{Алгоритм Евклида в евклидовой области. Существование НОД в
		евклидовой области. Теорема о линейном представлении НОД в евклидовой области.}
	
	\subsection{Евклидова область}
	\textbf{Определение:} область целостности $R$, в которой существует функция нормы $N:R \setminus \{0\} \to \mathbb{N}$ такая что: $N(a \cdot b) \geq N(a)$ и $\forall a, b \in R \ \exists c, r \in R, b\neq 0: a = b\cdot c + r : N(r) < N(b)$, или $r = 0$.
	
	\subsection{НОД}
	\textbf{Определение:} $R$ - коммутативное кольцо с единицей, $a, b \in R$, $gcd(a, b) = d \in R$: 1) $d|a, d|b$ 2) $\forall c \in R: c | a, c | b  \ c | d$. \\
	\textbf{Свойство:} если $gcd(a, b)$ существует, то единственный с точностью до ассоциированного. Если $gcd(a, b) = d = d'$, то $d | d'$ и $d' | d$, а если $R$ - область целостности, то $d = d' \cdot u, u \in R^{\times}$ \\
	\textbf{Доказательство:} $d | a, d | b, d'|a, d' |b \implies d' | d$, аналогично для $d$
	
	\subsection{Алгоритм Евклида}
	\textbf{Теорема:} $R$ - евклидово кольцо $a, b \in R$ и они одновременно не ноль, тогда: $$1) \exists gcd(a, b) = d \in R \ 2) \exists x, y \in R: d = ax + by $$
	\textbf{Доказательство:} 
	1 случай: $a = 0$ или $b = 0$, тогда $gcd(0, b) = b$ или $gcd(a, 0) = a$, так как $a = gcd(a, 0) = a \cdot 1 + 0 \cdot 1$ и аналогично для $b$ \\
	2 случай: $a, b \neq 0$ \\
	$a = q_0b + r_0 \land N(r_0) < N(b)$ \\
	$b = q_1r_0 + r_1 \land N(r_1) < N(r_0)$ \\
	$r_0 = q_2r_1 + r_2 \land N(r_2) < N(r_1)$ \\
	$\dots$  \\
	$r_{n-1} = q_{n-1}r_n + r_{n+1} \land N(r_{n+1}) < N(r_n)$ \\
	на каком то шаге станет: \\
	$r_n = q_nr_{n + 1} \implies r_{n+2} = 0$ \\
	\textbf{Утверждение:} $r_{n+1} = gcd(a, b)$ \\
	\textbf{Доказательство:}
	\begin{enumerate}
		\item Первое условие НОД \\
		$r_{n+1} | a \land r_{n+1} | b$: 
		$r_{n+1} | r_n \implies r_{n+1} |r_{n-1} \implies \dots \implies r_{n + 1} | b \implies r_{n+1} | a$ 
		\item Существует линейная комбинация с $r_{n+1}$ \\
		$r_0 = a - q_0b$ \\
		$r_1 = b - q_1r_0 = b - q_1(a - q_0 b) = b - q_1a + q_0q_1b = -q_1a + b(1 + q_0q_1b)$, пусть коэффициент $a$ это $S_1$, а коэффициент с $b$ это $T_1$. \\
		$r_2 = S_2a + T_2b$ \\
		$\dots$ \\
		$r_{n + 1} = S_{n + 1} a + T_{n + 1}b$
		\item Также нам нужно доказать второе условие НОД \\
		$\forall c \in R: c | a, c | b \implies c | S_{n + 1}a \land c | T_{n + 1}b \implies c | r_{n + 1} \implies gcd(a, b) = r_{n + 1}$ 
	\end{enumerate}
	
	\section{Евклидова область является областью главных идеалов.}
	\subsection{Область главных идеалов}
	\textbf{Определение:} область целостности $R$ - область главных идеалов, если для любого идеала $I \subseteq R$ существует такой элемент $x \in I: (x)  = I$ 
	\subsection{Идеал}
	\textbf{Определение:} $$I - \text{идеал} \ M \iff \begin{cases}
		I \subseteq M, I\neq \varnothing \\
		\forall a,b \in I \ a + b \in I \\
		\forall m \in M \ m\cdot i \in I
	\end{cases}$$
	\subsection{Главный идеал}
	\textbf{Определение:} $$ (x) - \text{главный идеал} \ M \iff (x) = \{xm : m \in M\}$$ \\
	\textbf{Утверждение:} любое евклидовое кольцо $R$ - область главных идеалов. \\
	\textbf{Доказательство:} Пусть $I$ - идеал в $R$, $I \neq \{0\}$, $S = \{N(x): x \in I\} \implies \exists N(d)$ - наименьшая норма. Можно утверждать, что $(d) = I$:
	\begin{enumerate}
		\item левое включение: $d \in I \implies rd \in I \implies (d) \subset I$ 
		\item правое включение: Пусть $x \in I$, разделим $x$ на $d$ с остатком: $x = dq \cdot r$, нам нужно, чтобы $r = 0$. У нас допустима ситуация, что $N(r) < N(d)$, но такая ситуация на самом деле невозможна, так как $r = x - dq \implies r \in I$, но $N(d)$ - наименьшая норма для элементов из $I$, поэтому $r$ может быть только нулем, тогда $I \subset (d)$ 
	\end{enumerate}
	
	\section{В области главных идеалов каждая возрастающая цепочка идеалов
		стабилизируется.}
	\subsection{Теорема}
	Пусть $R$ - область главных идеалов, также у нас есть цепочка вложенных идеалов: $I_1 \subset I_2 \subset I_3 \dots$, тогда $\exists N \in \mathbb{N}: I_N = I_{N  + 1} = I_{N + 2} \dots$, то есть цепочка стабилизируется. Такое явление еще называют Нетеровым кольцом. \\
	\textbf{Доказательство:} Так как $R$ - ОГИ, каждый $I_k = (i_k)$, то есть каждый идеал главный. Допустим у нас есть $I_{\infty} = \bigcup\limits_{k = 1} I_k$, теперь покажем, что $I_{\infty}$ это тоже идеал:
	\begin{enumerate}
		\item Замкнутость по сложению \\
		Пусть $a, b \in I_{\infty}$, тогда $\exists n_1, n_2 \in \mathbb{N}: a \in I_{n_1}, b \in I_{n_2}$, тогда не умаляя общности: $a, b \in I_{n_2} \implies a + b \in I_{n_2} \in I_{\infty}$ 
		\item Замкнутость по умножению на элемент из кольца \\
		Пусть $a \in I_{\infty}$ и $r \in R$, тогда $\exists n \in \mathbb{N}: a \in I_n$, так как $I_n$ - идеал: $ar \in I_n \in I_{\infty}$
	\end{enumerate}
	Так как $I_{\infty}$ - идеал и он лежит в $R$, $\exists x \in I_{\infty}: (x) = I_{\infty}$, тогда $\exists N: x \in I_N$, но мы можем сказать, что $I_N = I_{\infty}$, тут не очевидно только левое включение: $\forall y \in I_{\infty} \implies y = sx, s \in R$, $x \in I_N \implies sx  = y \in I_N \implies I_N = I_{\infty}$, тогда начиная с $N$: $I_N = I_{N + 1} = I_{N + 2} \dots$ 
	
	\section{В области главных идеалов необратимый элемент раскладывается
		в произведение неприводимых.}
	\subsection{Неприводимый элемент}
	\textbf{Определение:} $r \in R \setminus R^{\times}$ - неприводимый, если из разложения $r = st$ следует, что хотя бы один из элементов обратимый, то есть $s \in R^{\times} \lor t \in R^{\times}$
	\subsection{Теорема}
	\textbf{Утверждение:} $R$ - ОГИ, $r \in R \setminus R^{\times}$, тогда существует разложение $r$ в произведение неприводимых единственное с точностью до перестановки множителей и ассоциированности.
	\\ \textbf{Доказательство:} Пусть у нас есть множество $S = \{(x): x -$ не раскладывается в произведение неприводимых, $ x \neq 0, x \in R\setminus R^{\times}\}$, мы предположим, что оно не пусто, по доказательству конечности цепочки вложенных идеалов мы можем взять максимальный элемент $(z)$, тогда мы можем сказать, что $z$ не может быть неприводимым, так как тогда оно будет иметь разложение на неприводимые $z = z$, что противоречит нашему множеству $S$. Тогда из этого следует, что $z$ - приводимый, тогда он имеет разложение $z = st, s,t \in R \setminus R^{\times}$, но тогда $z$ делится на $s$ или $t$, а это то же самое, что $(z) \subset (s)$ или $(z) \subset (t)$, но $(z)$ - минимальный по включению идеал в $S \implies (s), (t) \notin S$, тогда по определению множества $S$ $s$ и $t$ имеют разложение на неприводимые: $s = q_1 \cdot q_2 \dots q_n$, $t = p_1 \cdot p_2 \dots p_n$, но тогда $z = (q_1 \cdot q_2 \dots q_n) \cdot (p_1 \cdot p_2 \dots p_n)$, следовательно $z$ имеет разложение на неприводимые, получаем противоречие, следовательно $S = \varnothing$, тогда любой элемент в Области главных идеалов имеет разложение на неприводимые.
	
	\section{Основная теорема арифметики в области главных идеалов.}
	\subsection{Простой элемент}
	\textbf{Определение:} элемент $r \in R \setminus{R^{\times}}$ называется простым, если из того, что $r | ab \implies r|a \lor r|b$ 
	\subsection{Лемма}
	\textbf{Утверждение:} $R$ - ОГИ, $r$ - неприводим, тогда $r$ - простой. \\
	\textbf{Доказательство:} Пусть $r | ab$, рассмотрим все возможные линейные комбинации $(r, a)$, тогда $\exists d \in R: (d) = (r, a) \implies d | r, d | a$ , $r \in (r,a) \implies r \in (d) \implies \ \exists s \in R: r = sd \implies$ 1) $d \in R^{\times}$ 2) $r \sim d$.
	\begin{enumerate}
		\item Если $d \in R^{\times} \implies (d) = R \implies d = 1$, получается что наша линейная комбинация это $1 = xr + ya$, мы можем домножить все на $b$, тогда $b = xrb + yab$, тогда $r | xrb, r | ab \implies r | b$
		\item Если $r \sim d$, то по определению ассоциированности $r | d$, а $d | a \implies r | a$
	\end{enumerate}
	То есть для любого случая $r$ - простой
	
	\subsection{Единственность разложения на неприводимые}
	Пусть $r = p_1 \cdot p_2 \dots p_n = q_1 \cdot q_2 \dots q_n$ все элементы неприводимы, а значит простые. Рассмотрим $p_1$: $p_1 | (p_1 \cdot p_2 \dots p_n), p_1 | (q_1 \cdot q_2 \dots q_m)$. $m = n$, иначе бы какие то элементы были единицами, чтобы не нарушить равенство, тогда бы нарушилось условие, что все элементы неприводимы. Затем находим среди элементов $q_i$ такой, что $p_1 \sim q_i$, меняем местами $q_i$ и $q_1$, после всех подобных перестановок получим $p_1 \sim q_1, p_2 \sim q_2, \dots, p_n \sim q_n$.
	
	\section{Фактор-кольцо, корректность операций. Кольцо классов вычетов.
		Обратимые элементы в \texorpdfstring{$\mathbb{Z}/n\mathbb{Z}.$}{Z/nZ}}
	
	\subsection{Отношение эквивалентности}
	\textbf{Определение:} $\sim$ - отношение эквивалентности на множестве $M$, если $\sim \ \subset M \times M$, в котором для пары $(a, b)$ выполняется заданное условие $a \sim b$, а также отношение обладает: 1) рефлексивность, 2) симметричность, 3) транзитивность.
	\subsection{Фактор-кольцо}
	\textbf{Определение:} $R$ - кольцо, $I$ - идеал на этом кольце, тогда $R / I$ это фактор по отношению эквивалентности такому, что $a \sim b \iff a - b \in I$, то есть элементы различаются на элемент из идеала. Тогда $R/I$ это множество из классов эквивалентности $[x] = \{y \in R: x \sim y\}$ 
	\subsection{Корректность операций}
	Докажем, что $[a] + [b] = [a + b] = [a' + b'], a \sim a', b \sim b'$: Нам нужно доказать что $a + b \sim a' + b'$. $(a + b) - (a' + b') = (a - a') + (b - b') \in I$ \\
	Докажем что $[ab] = [a'b']$: $ab - a'b' = ab - ab' + ab' - a'b' = a(b - b') + b'(a - a') \in I$
	\subsection{Кольцо классов вычетов}
	\textbf{Определение:} Кольцо классов вычетов это кольцо $\mathbb{Z}$ факторизованное по главному идеалу $(n)$. Оно содержит классы эквивалентности, в которых $a \sim b \implies a = b \mod  n$. То есть классы эквивалентности содержат элементы, которые дают одинаковый остаток при делении на $n$. 
	\subsection{Обратимые элементы в \texorpdfstring{$\mathbb{Z}/n\mathbb{Z}.$}{Z/nZ}}
	Обратимыми в Z/nZ являются такие $x: gcd(x, n) = 1$, это очень легко доказывается, мы можем представить линейную комбинацию для нод: $xk + yn = 1 \mod n \implies xk = 1 \mod n \implies k$ - обратимый элемент для $x$. То есть в $\mathbb{Z} / p\mathbb{Z}$ все ненулевые элементы обратимы, соответственно это уже не просто кольцо, а поле
	
	
	\section{Гомоморфизм колец. Проекция на фактор - гомоморфизм. Ядро
		гомоморфизма - идеал. Образ гомоморфизма - подкольцо.}
	\subsection{Гомоморфизм колец}
	\textbf{Определение:} отображение $f: R \to S$ - гомоморфизм если:
	\begin{enumerate}
		\item $f(r_1 + r_2) = f(r_1) + f(r_2)$
		\item $f(r_1 \cdot r_2) = f(r_1) \cdot f(r_2)$
		\item $f(1_R) = 1_S$, требование единицы не является обязательным, если кольца без единицы.
	\end{enumerate}
	
	\subsection{Проекция на фактор кольца}
	\textbf{Утверждение:} Если $I$ - идеал кольца $R$, то $\pi : R \to R/I$ - гомоморфизм. \\
	\textbf{Доказательство:} $\pi(a)$ просто отправляет a в его класс эквивалентности $[a]$.
	\begin{enumerate}
		\item $\pi(a + b) = [a + b] = [a] + [b] = \pi(a) + \pi(b)$
		\item $\pi(ab) = [ab] = [a] \cdot [b] = \pi(a) \cdot \pi(b)$
		\item $\pi(1) = [1] = 1_{R/I}$
	\end{enumerate}
	
	\subsection{Ядро гомоморфизма}
	\textbf{Определение:} для гомоморфизма $f: R \to S$ ядро $\ker f = \{r \in R: f(r) = 0\}$, то есть в ядре лежат все элементы $R$, которые превратятся в ноль. \\
	\textbf{Утверждение:} $\ker f$ - идеал в кольце $R$.
	\\ \textbf{Доказательство:} 
	\begin{enumerate}
		\item $ker f \neq \varnothing$, т.к всегда в ядре находится ноль. 
		\item пусть $a, b \in \ker f$, тогда $f(a) = f(b) = 0 \implies f(a + b) = f(a) + f(b) = 0 + 0 = 0 \in \ker f$
		\item пусть $a \in \ker f$, $b \in R$, тогда $f(ra) = f(r) \cdot f(a) = f(r) \cdot 0 = 0 \in \ker f$
	\end{enumerate}
	Все три условия для того что ядро - идеал доказаны.
	\subsection{Образ гомоморфизма}
	\textbf{Определение:} образ гомоморфизма $f: R \to S$ - $Im \ f \{s \in S: \exists r \in R: f(r) = s\}$
	\\ \textbf{Утверждение:} такой образ - подкольцо $S$.
	\\ \textbf{Доказательство:}
	\begin{enumerate}
		\item $f(0_R) = f(0_R + 0_R) = f(0_R) + f(0_R) \implies f(0_R) = 0_S$
		\item Пусть $x, y \in Im \ f \implies \exists a,b \in R: x = f(a), y = f(b)$, тогда $x - y =f(a) - f(b) = f(a - b) \in Im \ f$ - замкнуто по вычитанию, значит есть для каждого обратный элемент по сложению
		\item $xy = f(a) \cdot f(b) = f(ab) \in Im \ f$ - замкнуто по умножению
	\end{enumerate}
	Следовательно $Im \ f$ - подкольцо $S$.
	
	
	\section{Критерий инъективности гомоморфизма колец через определение
		ядра. Гомоморфизм является изоморфизмом тттк он биекция.}
	\subsection{Критерий инъективности}
	\textbf{Утверждение:} $f : R \to S$ - гомоморфизм, тогда $f$ - инъекция $\iff \ker f = \{0\}$ \\
	\textbf{Доказательство:} 
	\begin{enumerate}
		\item достаточность: \\
		Пусть $\exists a \in \ker f: f(a) = 0$, но $f(0) = 0$, тогда по инъективности $f$ $a = 0$
		\item необходимость: \\
		Пусть $a, b \in R$ и $f(a) = f(b) \implies f(a) - f(b) = 0 \implies f(a - b) = 0 \implies$
		$\implies a - b \in \ker f \implies a - b = 0 \implies a = b \implies f -$ инъекция
	\end{enumerate}
	\subsection{Изоморфизм}
	\textbf{Определение:} Кольца $R$ и $S$ называют изоморфными если для гомоморфизма $f: R \to S$, существует гомоморфизм $g : S \to R$, такой что у $f$ есть обратный слева $g \circ f = id_R$ и обратный справа $f \circ g = id_S$.
	\subsection{Гомоморфизм являтся изоморфизмом}
	\textbf{Утверждение:} если гомоморфизм $f: R \to S$ - изоморфизм $\iff f$ - биекция. \\
	\textbf{Доказательство:} 
	\begin{enumerate}
		\item достаточность: \\
		если $f$ - изоморфизм, то по определению существует $g: S \to R$ такой, что $g \circ f = id_R \implies f$ - инъекция и $f \circ g = id_S \implies f$ - сюръекция, тогда $f$ - биекция.
		\item необходимость: \\
		если $f$ - биекция, то существует обратное отображение $f^{-1}$, нам надо доказать, что это гомоморфизм (наличие обратного левого и правого исходит из биективности $f$). Возьмем любые $y_1, y_2$ такие что $x_1 = f^{-1}(y_1), x_2 = f^{-1}(y_2)$:
		\begin{enumerate}
			\item сложение: \\
			нам нужно доказать, что $f^{-1}(y_1 + y_2) = f^{-1}(y_1) + f^{-1}(y_2)$. 
			$y_1 + y_2 = f(x_1 + x_2)$. $f^{-1}(y_1 + y_2) = f^{-1}(f(x_1 + x_2)) = x_1 + x_2 = f^{-1}(y_1) + f^{-1}(y_2)$
			\item умножение: \\
			нам нужно доказать, что $f^{-1}(y_1 \cdot y_2) = f^{-1}(y_1) \cdot f^{-1}(y_2)$. 
			$y_1 \cdot y_2 = f(x_1 \cdot x_2)$. $f^{-1}(y_1 \cdot y_2) = f^{-1}(f(x_1 \cdot x_2)) = x_1 \cdot x_2 = f^{-1}(y_1) \cdot f^{-1}(y_2)$
		\end{enumerate}
	\end{enumerate}
	Выходит, что $f^{-1}$ - гомоморфизм, тогда $f$ - изоморфизм.
	
	\section{Лемма о пропуске гомоморфизма через фактор-кольцо. Теорема об
		изоморфизме.}
	\subsection{Лемма}
	\textbf{Утверждение:} $f: R \to S$ - гомоморфизм, $I$ - идеал $R$, тогда $\exists \overline{f}: R/I \to S$ - гомоморфизм, такой, что $\overline{f} \circ \pi = f \iff I \subset \ker f$. \\
	\textbf{Доказательство:}
	\begin{enumerate}
		\item достаточность: \\
		$\forall a \in I \ f(a) = \overline{f}(\pi(a)) = \overline{f}(0) = 0 \implies a \in \ker f$
		\item необходимость: \\
		Пусть $[r] \in R/I$, тогда $\overline{f}([r]) = f(r), r \in [r]$. 
		Но нам необходимо доказать корректность:
		Пусть $r, r' \in [r]$, $f(r') = \overline{f}([r]) = f(r)$? Это действительно так: $f(r) - f(r') = f(r - r') \in I$ и равно нулю, по предположению, что $I \in \ker f$.
		Почему $\overline{f}$ - гомоморфизм:
		\begin{itemize}
			\item $\overline{f}([1]) = f(1) = 1$
			\item $\overline{f}([a] + [b]) = \overline{f}([a + b]) = f(a + b) = f(a) + f(b) = \overline{f}([a]) + \overline{f}([b])$
			\item $\overline{f}([a] \cdot [b]) = \overline{f}([a \cdot b]) = f(a \cdot b) = f(a) \cdot f(b) = \overline{f}([a]) \cdot \overline{f}([b])$
		\end{itemize}
	\end{enumerate}
	Также покажем коммутативность: Пусть $r \in R$, $\overline{f}(\pi(r)) = \overline{f}([r]) = f(r)$
	\subsection{Первая теорема об изоморфизме}
	\textbf{Утверждение:} $f: R \to S$ - гомоморфизм, тогда $R/\ker f \xrightarrow{\cong} Im \ f$ \\
	\textbf{Доказательство:} существует гомоморфизм $R \to Im \ f$, тогда, чтобы построить $R/\ker f \to Im \ f$ необходимо и достаточно по лемме, чтобы $\ker f \subset \ker f$, что всегда правда. Обозначим этот гомоморфизм $\overline{f}$, теперь нам надо доказать, что он биективен:
	\begin{enumerate}
		\item инъекция: \\
		$\ker f = \{[r] \in R/\ker f: \overline{f}([r]) = 0\} = \{[r] : f(r) = 0\} = \{0\}$, тогда это инъекция
		\item Сюръекция: \\
		Пусть $s \in Im \ f$, тогда существует $r \in R: f(r) = s$, тогда $\overline{f}([r]) = f(r) = s$, получается сюръекция.
	\end{enumerate}
	$\overline{f}$ - биекция, а значит и изоморфизм. 
	
	\section{Китайская теорема об остатках в \texorpdfstring{$\mathbb{Z}$}{Z}.}
	\textbf{Утверждение:} $gcd(m, n) = 1$, тогда $\sigma: \mathbb{Z}/mn\mathbb{Z} \xrightarrow{\cong} (\mathbb{Z}/n\mathbb{Z}) \times (\mathbb{Z}/m\mathbb{Z})$ \\
	\textbf{Доказательство:} $\sigma(k) = (k \mod n, k \mod m)$ 
	Покажем, что $\sigma$ - гомоморфизм:
	\begin{enumerate}
		\item Сложение: $\sigma(k + k') = ([k + k']_n, [k + k']_m) = ([k]_n + [k']_n, [k]_m + [k']_m) = ([k]_n, [k]_m) + ([k']_n, [k']_m) = \sigma(k) + \sigma(k')$
		\item Умножение: Умножение: $\sigma(k \cdot k') = ([k \cdot k']_n, [k \cdot k']_m) = ([k]_n \cdot [k']_n, [k]_m \cdot [k']_m) = ([k]_n, [k]_m) \cdot ([k']_n, [k']_m) = \sigma(k) \cdot \sigma(k')$
		\item Сохраняет единицу: $\sigma([1]_{mn}) = ([1]_{mn} \mod n, [1]_{mn} \mod m) = ([1]_n, [1]_m)$
	\end{enumerate}
	Определим $\ker \sigma = \{k \in \mathbb{Z}/nm\mathbb{Z}: k = 0 \mod n \land k = 0 \mod m\}$, пусть у нас $k \in \ker \sigma$, тогда $k$ делится на $m$ и $n$, тогда $k = 0 \mod nm \implies \sigma$ - инъекция. Так как у нас одинаковое количество элементов в $\mathbb{Z}/nm\mathbb{Z}$ и $\mathbb{Z}/n\mathbb{Z} \times \mathbb{Z}/m\mathbb{Z}$ и $\sigma$ - инъекция, каждый элемент слева получает уникальный справа, при этом каждый слева имеет связь, получается левое полностью покрывает правое, тогда это сюръекция и биекция соответственно. Если $\sigma$ - биекция, то мы уже ранее доказывали, что $\sigma$ - изоморфизм.
	
	\section{Количество элементов в произведении колец. Обратимые элементы
		в произведении колец. Мультипликативность функции Эйлера.}
	\subsection{Произведение колец}
	\textbf{Определение:} $R, S$ - кольца, тогда $R \times S = \{(r, s): r\in R, s \in S\}$ с операциями:
	\begin{enumerate}
		\item сложение: $(r_1, s_1) + (r_2, s_2) = (r_1 + r_2, s_1 + s_2)$
		\item умножение: $(r_1, s_1) \cdot (r_2, s_2) = (r_1r_2, s_1s_2)$
	\end{enumerate}
	
	Докажем, что это кольцо:
	\begin{enumerate}
		\item Ассоциативность сложения: \\
		$(r_1, s_1) + ((r_2, s_2) + (r_3, s_3)) = (r_1, s_1) + (r_2 + r_3, s_2 + s_3) = (r_1 + r_2 + r_3, s_1 + s_2 + s_3) = (r_1 + r_2, s_1 + s_2) + (r_3 + s_3) = ((r_1, s_1) + (r_2, s_2)) + (r_3, s_3)$
		\item Существование нейтрального элемента: \\
		$(r_1, s_1) + (0_R, 0_S) = (r_1, s_1) \implies (0_R, 0_S)$ - нейтральный элемент по сложению. 
		\item Существование обратных по сложению: \\
		$\forall r \in R, s \in S, -r, -s$ - обратные по сложению, тогда $(r, s) + (-r, -s) = (0_R, 0_S)$.
		\item Ассоциативность умножения: \\
		$((r_1, s_1) \cdot (r_2, s_2)) \cdot (r_3, s_3) = (r_1 \cdot r_2 \cdot r_3, s_1 \cdot s_2 \cdot s_3) = (r_1, s_1) \cdot ((r_2, s_2) \cdot (r_3, s_3))$
		\item Дистрибутивность: \\
		$(r_1, s_1) \cdot ((r_2, s_2) + (r_3, s_3)) = (r_1, s_1) \cdot (r_2 + r_3, s_2 + s_3) = (r_1r_2 + r_1r_3, s_1s_2 + s_1s_3) = (r_1r_2, s_1s_2) + (r_1r_3, s_1s_3) = (r_1, s_1) \cdot (r_2, s_2) + (r_1, s_1) \cdot (r_3, s_3)$
	\end{enumerate}
	Доказали все аксиомы кольца, значит это кольцо.
	\subsection{Количество элементов в произведении колец}
	\textbf{Утверждение:} $R, S$ - конечные кольца, тогда $|R \times S| = |R| \cdot |S|$ \\
	\textbf{Доказательство:} допустим зафиксируем $r \in R$, тогда пар $(r, s_i)$ будет $|S|$, и всего таких возможных пар с разными $r$ будет $|R| \cdot |S|$.
	\subsection{Обратимые элементы в произведении колец}
	\textbf{Утверждение:} $R, S$ - кольца, тогда $(R \times S)^{\times} = R^{\times} \times S^{\times}$ \\
	\textbf{Доказательство:} 
	\begin{enumerate}
		\item правое включение: \\
		$(r, s) \in R^{\times} \times S^{\times} \implies (r, s)^{-1} = (r^{-1}, s^{-1})$
		\item левое включение: \\
		$(r, s) \in (R \times S)^{\times} \implies (t, u) \in R \times S: (r, s) \cdot (t, u) = (1, 1) \implies (rt, su) = (1, 1) \implies r, s$  - обратимы.
	\end{enumerate}
	\subsection{Функция эйлера}
	\textbf{Определение:} $\varphi: \mathbb{N} \to \mathbb{N}$, $\varphi(n) = |\{k : 1\leq k \leq n, gcd(k, n) = 1\}$, такая функция считает сколько взаимно простых элементов с $n$, которые меньше $n$.
	\subsection{Мультипликативность функции Эйлера}
	\textbf{Утверждение:} Пусть $gcd(m, n) = 1$, тогда $\varphi(mn) = \varphi(m) \cdot \varphi(n)$ \\
	\textbf{Доказательство:} так как в кольце вычетов $\mathbb{Z}/n\mathbb{Z}$ элемент $k$ обратим только когда $gcd(k, n) = 1$, мы можем сказать, что $|(\mathbb{Z}/n\mathbb{Z})^{\times}| \cong \varphi(n)$. 
	Теперь мы можем сказать, что $\varphi(mn) = |(\mathbb{Z}/mn\mathbb{Z})^{\times}| \overset{\text{КТО}}{=} |(\mathbb{Z}/m\mathbb{Z} \times \mathbb{Z}/n\mathbb{Z})^{\times}| = |(\mathbb{Z}/m\mathbb{Z})^{\times} \times (\mathbb{Z}/n\mathbb{Z})^{\times}| = |(\mathbb{Z}/m\mathbb{Z})^{\times}| \cdot |(\mathbb{Z}/n\mathbb{Z})^{\times}| = \varphi(m) \cdot \varphi(n)$   
	
	\section{Вычисление функции Эйлера. Теорема Эйлера.}
	\subsection{Вычисление функции Эйлера} $\varphi(n) = \varphi(\prod p_i^k) = \prod \varphi(p_i^k)$, теперь нам надо понять как быстро считать $\varphi(p^k)$:
	$\varphi(p) = p -1$ 
	$\varphi(p^2)=p^2 - \frac{p^2}{p}$, здесь такая логика, всего у нас элементов $p^2$, мы можем мысленно разделить их на $p$ блоков по $p$ элементов так, что у нас в каждом блоке только последний будет делиться на $p$, например тут это $p, 2p \dots p^k$, их будет столько же, сколько у нас блоков, а их $\frac{p^2}{p}$. 
	В общем случае это $\varphi(p^k) = p^k - \frac{p^k}{p} = p^k - p^{k - 1} = p^{k - 1} \cdot (p - 1)$. 
	\subsection{Теорема Эйлера}
	\textbf{Утверждение:} если $gcd(a, n) = 1$, то $a^{\varphi(n)} = 1 \mod n$ \\
	\textbf{Доказательство:} пусть у нас есть множество взаимно простых с n $\{r_1, r_2, \dots r_{\varphi(n)}\}$, все эти числа обратимы в $\mathbb{Z}/n\mathbb{Z}$, домножим каждый такой элемент на $a$, скажем, что у нас количество элементов осталось прежним, так как если $ar_i = ar_j$, то мы можем сократить на $a$, потому что оно обратимо по модулю $n$, тогда $r_i = r_j \implies i = j$.
	$r_1 \cdot r_2 \dots \cdot r_{\varphi(n)} = ar_1 \cdot ar_2 \dots \cdot ar_{\varphi(n)} \mod n \implies r_1 \cdot r_2 \dots \cdot r_{\varphi(n)} =$
	$= a^{\varphi(n)} \cdot (r_1 \cdot r_2 \dots \cdot r_{\varphi(n)}) \mod n$ , Так как у нас произведение элементов $r$ состоит из обратимых элементов мы можем сократить на него, тогда $a^{\varphi(n)} = 1 \mod n$.
	
	\section{Построение кольца многочленов над полем, деление с остатком. Деление на двучлен. Теорема Безу. Следствие о количестве различных
		корней многочлена.}
	\subsection{Кольцо многочленов}
	\textbf{Определение:} $R[x] = \{(r_1, r_2, \dots, r_n, 0, 0 \dots): r_i \in R \land \exists n\}$ - кольцо многочленов. $(r_1, r_2, \dots, r_n, 0, 0, \dots)$ - финитная последовательность, пусть $e_i \in R[x] = (0, 0, 0, \dots, 1, 0, \dots)$, где единица стоит на $i$-том месте. Тогда $r \cdot e_i = (0, 0, 0, \dots, r, 0, \dots)$, $e_i \cdot e_j = e_{i+j}$. Также мы можем сложить две такие последовательности: $a, b \in R[x]$, $m > n$ , $(a_0, a_1, \dots, a_n, 0, \dots) + (b_0, b_1, \dots, b_m, \dots) = (a_0 + b_0, a_1 + b_1, \dots, a_n + b_n, \dots, b_m, 0, \dots)$. Тогда мы также можем разложить нашу последовательность: $(r_0, r_1, \dots, r_n, 0, \dots) = (r_0, 0, \dots) + (0, r_1, 0, \dots) + (0, 0, \dots, r_n, 0, \dots)$, что то же самое, что и $r_0 \cdot e_0 + r_1 \cdot e_1 + \dots + \dots + r_n \cdot e_n$. Обозначим $e_0$ как $1$, $e_1$ как $x$, $e_k = e_1 \cdot e_1 \cdot e_1 \cdot \dots \cdot e_1 = x^k$.
	Тогда мы можем сказать, что $R[x] = \{r_0 + r_1x + \dots + r_nx^n: r_i \in R \land \exists n\}$.
	\subsection{Поле}
	\textbf{Определение:} кольцо $R$ называется полем, если $\forall r \neq 0 \in R$ существует $r^{-1}: r \cdot r^{-1} = 1$
	\subsection{Кольцо многочленов над полем}
	\textbf{Утверждение:} $(k[x])^{\times} = k^{\times} = k \setminus \{0\}$ \\
	\textbf{Доказательство:} $\deg p(x)$ - номер последнего ненулевого элемента последовательности
	$p(x) \in k[x]$ пусть также $\exists q(x) \in k[x]: p(x) \cdot q(x) = 1 = 1 + 0x + 0x^2 \dots$  
	$(p_0 + p_1x + \dots + p_nx^n) \cdot (q_0 + q_1x + \dots + q_mx^m) = \dots + p_nq_mx^{n+m} = 1$ 
	У нас два варианта: либо $p_nq_m = 0$, но они оба не ноль по условию, а делителей нуля не существует в поле $k$. Тогда $n + m = 0$, в таком случае $n = m = 0$, тогда $p(x) = p_0 \in k$ и $q(x) = q_0 \in k$.
	
	\textbf{Утверждение:} $k[x]$ - область целостности \\
	\textbf{Доказательство:} $p(x), q(x) \in k[x], p(x) \neq 0, q(x) \neq 0$, $\deg(pq) = deg(p) + deg(q)$.
	$\deg(p(x)q(x)) = \deg(p(x)) + \deg(q(x)) = 0 \implies \deg(p(x)) = \deg(q(x)) = 0$
	$p_0 \cdot q_0 = 0$, но в $k$ нет делителей нуля - противоречие.
	
	\textbf{Утверждение:} $k[x]$ - евклидова область. \\
	\textbf{Доказательство:} $k[x]$ - область целостности из прошлого доказательства. Теперь нам надо доказать существование функции $N$. 
	$N(p) = \deg p \ \forall p \in k[x], p \neq 0$, проверим свойства нормы:
	\begin{enumerate}
		\item $N(p(x)q(x)) \geq N(p(x))$, равенство только когда $q(x)$ - обратим
		\item Покажем, что $\forall f(x), g(x) \in k[x], g(x) \neq 0 \ \exists ! q(x), r(x) \in k[x]$, так что: 
		$f(x) = g(x) \cdot q(x) + r(x) :$ либо $r(x) = 0$, либо $N(r(x)) < N(g(x))$ или $\deg r < \deg g$
	\end{enumerate}
	Рассмотрим два случая деления многочленов с остатком: \\
	1 случай: \\
	$\deg f < \deg g$, то $q(x) = 0, r(x) = f(x)$, тогда $f(x) = 0\cdot g(x) + f(x)$, пример: $x = 0\cdot x^2 + x$ \\
	2 случай: \\
	Пусть $\deg f(x) \geq g(x)$, $f(x) = a_nx^n + \dots + a_0, g(x) = b_mx^m + \dots + b_0, n \geq m$. \\
	тогда: 
	$f_1(x) = f(x) - \frac{a_n}{b_m}x^{n - m}g(x), \deg f_1(x) < n$ \\
	$f_2(x) = f_1(x) - ?g(x),$ если подставить $f_1(x)$, то $f_2(x) = f(x) - \frac{a_n}{b_m}x^{n - m}g(x) - ?x^{?}g(x)$, можно выносить $g(x)$ за скобки \\
	$\dots$ \\ 
	$f_k(x) = f(x) - (?)g(x)$ до тех пор, пока $\deg f_k < \deg g$, тогда $f_k(x) = 0\cdot g(x) + f_k(x)$ \\
	Получается: $f(x) = (?)g(x) + f_k(x)$, тогда у нас два варианта: $f_k = 0$ либо $\deg f_k < \deg g$ \\
	Также докажем единственность такого деления: \\
	Пусть существуют $q_1, q_2, r_1, r_2 \in k[x]: f(x) = q_1(x)g(x) + r_1(x) = q_2(x)g(x) + r_2(x)$ \\
	$g(q_1 - q_2) = r_2 - r_1$, Пусть $r_2 - r_1 \neq 0, r_1 \neq 0, \deg r_1 > \deg r_2$, тогда $\deg ((q_1 - q_2)g) = \deg(q_1 - q_2) + \deg g > \deg (r_1) \geq (r_2 - r_1)$, но у нас $\deg((q_1 - q_2)g) = \deg(r_1 - r_2)$, получаем противоречие. Тогда $r_1 - r_2 = 0 \implies r_1 = r_2$. Получается, что $(q_1 - q_2) = 0 \implies q_1 = q_2$. 
	\subsection{Деление на двучлен}
	Пусть мы хотим разделить $f(x)$ на $g(x) = (x - a)$ 
	$f(x) = q(x)(x-a) + r(x)$, стоит заметить, что в данном случае $r(x)$ это просто $r \in k$, так как либо он ноль, либо его степень меньше 1, а значит это тоже константа.
	\subsection{Теорема Безу}
	\textbf{Теорема:} при делении $f(x)$ на $g(x) = x - a$ $r(x) = f(a)$ \\
	\textbf{Доказательство: } $f(x) = q(x)(x-a) + r(x)$, $f(a) = q(x)\cdot 0 + r(x) \implies f(a) = r(x)$
	\subsection{Первое следствие:} $\exists \alpha: f(\alpha) = 0$, тогда $f(x) = q(x)(x - \alpha)$, то есть $f(x)$ делится на $(x - \alpha)$
	\subsection{Второе следствие:} 
	\textbf{Утверждение:} Пусть $\deg f = n$, тогда количество различных корней $f(x) \leq n$. \\
	\textbf{Доказательство:} Пусть $a_1, a_2, \dots a_n$ - корни $f(x)$, тогда $f(x) = g(x)(x - a_1)$, подставим $a_2$: $f(a_2) = g(a_2)(a_2 - a_1) = 0 \implies g(a_2) = 0$, тогда по первому следствию $g(x)$ можно дальше разложить так, чтобы $f(x) = g(x)(x - a_1)(x - a_2)$. таким образом мы можем разложить все корни так, что $\deg$ такого разложения будет равен количеству корней + степени оставшегося $g(x)$, что меньше чем исходный $n$, так как у нас при разложении всегда уменьшается степень $g(x)$.
	
	\section{Теорема о формальном и функциональном равенстве многочленов.
		Антипример для конечного поля.}
	\subsection{Характеристика}
	\textbf{Определение:} Пусть $R$ - кольцо, Характеристика $R$ ($char \ R$) - это наименьшее натуральное $n$, такое что сумма $n$ единиц равняется нулю. Если такого $n$ не существует, то $char \ R = 0$. Из этого следует, что $R$ - бесконечно.
	\subsection{Формальное и функциональное равенство многочленов}
	\textbf{Определение:} $f(x), g(x) \in k[x]$. Они равны формально, если просто равны в кольце многочленов. И равны функционально, если $\forall c \in k$  $f(c) = g(c)$. \\
	\textbf{Теорема:} из функционального равенства следует формальное, если $char \ k = 0$ \\
	\textbf{Доказательство:} Возьмем $h(x) = f(x) - f(g) \in k[x]$, если у нас не формально равны, то $h \neq 0$, тогда допустим $\deg h = n$. Но тогда $h$ имеет бесконечное количество различных корней, хотя по следствию из теоремы безу он имеет только не больше $n$ различных корней - противоречие, значит $f(x) = g(x)$ \\
	\textbf{Антипример:} если мы берем конечное поле, например $\mathbb{Z}/p\mathbb{Z}$, то в случае например многочленов $f(x) = x^p$ и $g(x) = x$ у нас они совпадают функционально, так как по малой теореме ферма $a^{p-1} = 1 \mod p \implies a^p = a \mod p$, но при этом формально они отличаются.
	
	\section{Сумма, произведение и пересечение идеалов. Проверка. Связь произведения и пересечения в общем случае и в случае взаимно простых (комаксимальных) идеалов.}
	\subsection{Сумма идеалов} 
	\textbf{Определение:} $R$ - кольцо, $I, J$ - идеалы в $R$, тогда $I + J = \{i + j: i \in I, j \in J\}$ \\
	\textbf{Утверждение:} $I + J$ - идеал. \\
	\textbf{Доказательство:} 
	\begin{enumerate}
		\item $(i_1 + j_1) + (i_2 + j_2) = (i_1 + i_2) + (j_1 + j_2)  \in I + J$
		\item $r \in R$, $r(i_1 + j_1) = ri_1 + rj_1 \in I + J$
	\end{enumerate}
	Пример: в $\mathbb{Z}$ сложим $12\mathbb{Z} + 18\mathbb{Z}$ = $\{18m + 12n\}_{n,m \in \mathbb{Z}}$ = $6\mathbb{Z}$ ($gcd(12, 18)$) так работает для всех целых чисел
	\subsection{Пересечение Идеалов} 
	\textbf{Определение:} $I \cap J$ - пересечение идеалов \\
	\textbf{Утверждение:} $I \cap J$ - идеал \\
	\textbf{Доказательство:} пусть $a, b \in I \cap J$
	\begin{enumerate}
		\item $a + b \in I, a + b \in J \implies a + b \in I \cap J$
		\item $r \in R,$ $ra \in I, ra \in J \implies ra \in I \cap J$
	\end{enumerate}
	Пример: $R$ = $\mathbb{Z}$ , $I  = 12\mathbb{Z}, J = 18\mathbb{Z}$, тогда $I \cap J = 36 \mathbb{Z}$ или же Нок(12, 18)
	\subsection{Умножение идеалов}
	\textbf{Определение:} $I \cdot J = \{\sum\limits_{k = 1} i_kj_k: i_k \in I, j_k \in J\}$ \\
	\textbf{Утверждение:} $I \cdot J$ - идеал \\
	\textbf{Доказательство:} 
	\begin{enumerate}
		\item $\sum\limits_k i_k j_k  + \sum\limits_s i'_s j'_s \in I \cdot J$
		\item $r \in R$ $r \sum\limits_k i_kj_k = \sum\limits_k (ri_k) \cdot j_k \in I\cdot J$
	\end{enumerate}
	Пример: $I \cdot J = \{\sum\limits_k 12a_k \cdot 18b_k\} = \{216 \sum\limits_k a_k b_k\} = 216 \mathbb{Z}$
	\subsection{Взаимная простота идеалов}
	\textbf{Определение:} $I, J$ - взаимнопросты, если $I + J = R \iff 1 \in I + J$
	\subsection{Связь умножения и пересечения}
	\textbf{Лемма:} 1)$I\cdot J \subset I \cap J$ и если $I, J$ - взаимно просты, то 2)$I\cap J = I \cdot J$ \\
	\textbf{Доказательство:} 
	\begin{enumerate}
		\item Пусть $\sum\limits_k i_kj_k \in I\cdot J$, тогда $i_k j_k \in I \implies \sum\limits_k i_k j_k \in I$, аналогично для $J$, тогда сумма лежит в пересечении $I\cap J$
		\item Пусть $a \in I\cap J$, так как $I, J$ - взаимно просты, $\exists i \in I, j \in J: 1 = j + i \implies a = ai + aj$, тогда пусть $a \in J$ для $i$ и $a \in I$ для $j$, тогда a - сумма элементов $i_k j_k$, а значит она лежит в $I\cdot J$
	\end{enumerate}
	
	\section{Китайская теорема об остатках для колец.}
	\subsection{Теорема:} $I, J$ - взаимно простые идеалы в кольце $R$, тогда $R/I\cdot J \cong R/I \times R/J$ \\
	\textbf{Доказательство:} 
	$\varphi: R \to R/I \times R/J$ - гомоморфизм \\
	$r \mapsto ([r]_I, [r]_J)$ \\
	$Im \ \varphi$: Докажем, что $\varphi$ - сюръекция.\\
	т.к $I, J$ - взаимно простые, $1 \in I + J$ или $\exists i \in I, j \in J: 1 = i + j$ \\
	$\varphi(i) = ([i]_I, [i]_J) = (0, [1 - j]_J) = (0, 1)$ \\
	$\varphi(j) = ([1 - i]_I, [j]_J) = (1, 0)$ \\
	Пусть $([r]_I, [r]_J)$ = $\varphi(rj + si) \implies Im \ \varphi = R/I \times R/J$ \\
	$\ker \varphi$: $r \in \ker \varphi$, $\varphi(r) = ([r]_I, [r]_j) = (0, 0)$ $\iff r \in I \land r \in J \iff r \in I\cap J$, но так как $I, J$ - взаимно просты $I\cap J = I \cdot J$ \\
	Получается $\ker \varphi = I \cdot J$. \\
	Тогда по теореме об изоморфизме: \\
	$\varphi: R/\ker \varphi \xrightarrow{\cong} Im \ \varphi$, что то же самое, что $\varphi: R/I\cdot J \xrightarrow{\cong} R/I \times R/J$, что и требовалось доказать.
	\subsection{Более общая формулировка}
	\textbf{Теорема:} Пусть $R$ - кольцо с идеалами $I_1, I_2, \dots, I_n$, все попарно взаимно просты, тогда $R/I_1 \cdot \dots \cdot I_n \cong R/I_1 \times \dots \times R/I_n$ \\
	\textbf{Доказательство:} Берем просто пару идеалов $I = I_1$ и $J = I_2 \cdot \dots \cdot I_n$, доказываем как для двух идеалов. Но нужно показать, что $I, J$ - взаимно простые. Сделаем линейные комбинации для $I_1$ и каждого другого идеала, затем перемножим их, тогда получится $1 = (i_1 + y_2) (i_2 + y_3)\dots (i_{n-1} + y_n) = y_2y_3 \dots y_n +$ какой то элемент из идеала $I$, тут $y_i$ - элемент $I_i$ идеала. Тогда мы составили линейную комбинацию для $I$ и $J$.
	
	\section{Формула Тейлора для многочлена. Критерий кратности корня.}
	\subsection{Формула Тейлора для многочлена}
	$f(x) \in k[x], f(x) = a_0 + a_1x + \dots + a_nx^n$, 
	$x = y+c, y = x - c$, 
	$f(x)=a_0 + a_1(y+c) + a_2(y+c)^2 + \dots + a_n(y+c)^n$.
	Пусть такой $f(x) = b_0 + b_1y + \dots + b_ny^n$, тогда $f(x) = b_0 + b_1(x - c) + b_2(x - c)^2 + \dots + b_n(x-c)^n$ \\
	\textbf{Определение:} $f'(x)$ - производная многочлена $f(x) \in k[x]$, если для $f(x) = a_0 + a_1x + \dots a_nx^n$, $f'(x) = a_1 + 2a_2x + 3a_3x^2 + \dots + na_nx^{n-1}$ \\
	\textbf{Утверждение:} $b_m = \frac{f^{(m)}(c)}{m!}$, если $char \ k = 0$ \\
	\textbf{Доказательство:} $f(x) = b_0 + b_1(x-c) + \dots + b_m(x-c)^m + \dots + b_n(x - c)^n$ , если мы возьмем производную $m$ раз, то все члены до $(x - c)^m$ занулятся, а $f^{(m)}(x) = m! \cdot b_m + ?(x - c)$, тогда при подстановке $f^{(m)}(c) = m! \cdot b_m \implies b_m = \frac{f^{(m)}(c)}{m!}$
	\subsection{Критерий кратности корня}
	\textbf{Определение:} $p(x) \neq 0 \in k[z]$, элемент $\alpha \in k$ называется корнем кратности $m \neq 1$, если $p(x) = (x - \alpha)^m \cdot g(x)$, где $g(\alpha) \neq 0$ \\
	\textbf{Определение:} корень $\alpha$ называют кратным, если $\alpha$ кратности $\geq 2$ \\
	\textbf{Утверждение:} $\alpha$ - кратный корень $p(x) \iff p(\alpha) = 0$ и $p'(\alpha) = 0$ \\
	\textbf{Доказательство:} 
	\begin{enumerate}
		\item Достаточность: $\alpha$ - кратный корень $p(x) \implies p(x) = (x - \alpha)^2\cdot h(x)$ \\
		$p(\alpha) = 0$ \\
		$p'(x) = 2(x-\alpha) \cdot h(x) + (x - \alpha)^2 \cdot h'(x) \implies p'(\alpha) = 0$
		\item Необходимость: $p(\alpha) = 0, p'(\alpha) = 0$ \\
		По теореме Безу: $p(x) = (x - \alpha) s(x)$, $p'(x) = s(x) + (x - \alpha)\cdot s'(x)$, но так как $p'(\alpha) = 0 \implies s(\alpha) = 0$, тогда снова по теореме безу: $s(x) = (x - \alpha) \cdot h(x)$, тогда $p(x) = (x - \alpha)^2 \cdot h(x) \implies \alpha$ - корень кратности $\geq 2$.
	\end{enumerate}
	\textbf{Утверждение:} $p(x) \neq 0 \in k[x]$, $\alpha$ - корень кратности $m \iff f(\alpha) = 0, f'(\alpha) = 0, \dots f^{(m-1)}(\alpha) = 0$ и $f^{(m)}(\alpha) \neq 0$
	
	\section{Количество корней с кратностями не больше степени многочлена.}
	\textbf{Теорема:} $p(x) \neq 0 \in k[x], \alpha_1,\dots, \alpha_n$ - различные корни кратности $l_1, \dots, l_n$ соответственно.
	Тогда $p(x) = (x - \alpha_1)^{l_1} \cdot (x - \alpha_2)^{l_2} \cdot \dots \cdot (x - \alpha_n)^{l_n} \cdot g(x), g(\alpha) \neq 0$ \\
	\textbf{Доказательство:} Так как $k[x]$ - евклидова область, это ОГИ, значит работает разложение на неприводимые, пусть $p(x) = u_1(x) \cdot \dots \cdot u_n(x)$ - разложение на неприводимые. Пусть $\alpha_1, \dots, \alpha_n$ - наши корни кратности $l_1, \dots, l_n$ соответственно, тогда каждый двучлен вида $(x - \alpha_i)$ забирает ровно $l_i$ элементов $u_i$. Тогда наш многочлен выглядит так: $p(x) = (x - \alpha_1)^{l_1} \cdot \dots (x - \alpha_n)^{l_n} \cdot h(x)$, где $h(\alpha_i) \neq 0 \forall i$. \\
	\textbf{Следствие:} $l_1 + l_2 + \dots + l_n \leq \deg p(x)$
	
	\section{Интерполяционная формула Лагранжа, единственность.}
	\subsection{Интерполяционная формула Лагранжа}
	Если мы хотим, чтобы наша кривая проходила через определенные точки $y_0, y_1, \dots, y_n \in k$, то мы можем построить многочлен $f(x) \in k[x]: \deg f \leq n, f(x_i) = y_i$. 
	$f(x) = \sum\limits_{i = 0}^n y_i l_i(x)$, где $l_i(x_j) \in k[x]$ - такой многочлен, который равен одному если $i = j$, то есть в этой точке мы должны подняться на $y_i$ вверх, иначе этот многочлен дает 0. $l_i(x) = \frac{(x - x_0)(x - x_1) \dots (x - x_{i-1})(x - x_{i+1}) \dots (x - x_n)}{(x_i - x_0) (x_i - x_1) \dots (x_i - x_{i - 1})(x_i - x_{i + 1}) \dots (x_i - x_n)}$, тогда $f(x_j) = \sum\limits_{i = 0}^n y_i \cdot l_i(x_j) = y_j$
	\subsection{Единственность}
	Пусть $h(x) = f(x) - g(x)$, где $f(x), g(x)$  - решают задачу интерполяции, тогда $h(x)$ будет иметь $n + 1$ корень при условии $\deg h \leq n$, что противоречит теореме безу, значит $h(x) = 0 \implies f(x) = g(x)$.
	
	\section{Критерий максимальности идеала. Критерий простоты идеала.}
	\subsection{Критерий простоты идеала}
	\textbf{Теорема:} $I \lhd R$ - простой $\iff R / I$ - область целостности \\
	\textbf{Доказательство:} 
	\begin{enumerate}
		\item Достаточность: \\
		$\left[a\right] \cdot \left[b\right] = 0$ в $R/I \implies \left[a \cdot b\right] = 0 \implies ab \in I \implies \left[ \begin{array}{l}
			a \in I \\
			b \in I
		\end{array}\right.$ \\ Получается, если $ab = 0$, то либо $a = 0$, либо $b = 0 \implies R/I$ - область целостности 
		\item Необходимость: \\
		$ab \in I \implies \left[ab\right] = \left[a\right] \left[b\right] = 0 \implies \left[\begin{array}{l}
			\left[a\right] = 0 \\
			\left[b\right] = 0
		\end{array}\right. \implies \left[\begin{array}{l}
		a \in I \\
		b \in I \\
		\end{array}\right.$
	\end{enumerate}
	\subsection{Критерий максимальности идеала}
	\textbf{Теорема:} $I \lhd R$ - максимален $\iff R / I$ - поле \\
	\textbf{Доказательство:} 
	\begin{enumerate}
		\item Достаточность: \\
		Пусть $\left[a\right] \in R/I$ и $\left[a\right] \neq 0$, мы хотим найти $\left[a\right]^{-1}$. $\left[a\right] \neq 0 \implies a \notin I$, введем идеал $J = I + \left(a\right), I \nsubseteqq J \implies J = R \implies 1 \in J \implies 1 \in I + \left(a\right) \implies 1 = i + ra \implies \left[1\right] = \left[ra\right] = \left[r\right] \left[a\right] \implies \left[r\right] = \left[a\right]^{-1} \implies R/I$ - поле
		\item Необходимость: \\
		$I \nsubseteqq J$, мы хотим показать, что $R = J$. Пусть $x \in J \setminus I \implies \left[x\right] \neq 0$. Введем $y: \left[x\right]\left[y\right] = 1 \implies \left[xy\right] = 1 \implies \left[xy - 1\right] = 0 \implies xy - 1 = i \in I \implies 1 = xy - i$, где $xy \in J, i \in J \implies 1 \in J \implies J = R$
	\end{enumerate}
	
	\section{ Квадратичные вычеты и невычеты. Символ Лежандра. Формула
		для символа Лежандра. Мультипликативность. Подсчет \texorpdfstring{$\left(\frac{-1}{p}\right)$}{(-1/p)}.}
	\subsection{Квадратичные вычеты и невычеты}
	Пусть мы хотим определить имеет ли уравнение $x^2 = a$ в поле $\mathbb{Z}/p\mathbb{Z}, a \neq 0, p \notin 2\mathbb{Z}$. Если такой $x$ существует, то $x^{p-1} = 1$ по малой теореме Ферма. \\ $1 = x^{p-1} = \left(x^2\right)^{\frac{p-1}{2}}=a^{\frac{p-1}{2}}$ \\
	$\left(\mathbb{Z}/p\mathbb{Z}\right)^{\times} = \left(\mathbb{Z}/p\mathbb{Z}\right)\setminus{\left\{0\right\}}$ является циклической группой по умножению: \\
	$\exists \gamma \in \left(\mathbb{Z}/p\mathbb{Z}\right):\forall b \in \left(\mathbb{Z}/p\mathbb{Z}\right)^{\times} \exists k \in \mathbb{N} \cup \{0\}: b = \gamma^k$ \\
	$\gamma^{p-1} = 1$ и $\gamma^m \neq 1 \forall m \in \left[1, p-2\right]$ \\
	Пусть $a^{\frac{p-1}{2}} = 1$ и $\exists k: a=\gamma^k$, $a^{\frac{p-1}{2}} = 1 = \gamma^{k\frac{p-1}{2}} \implies \frac{p-1}{2}k \ \vdots \ p-1 \iff \frac{k}{2} \in \mathbb{Z} \iff k \in 2\mathbb{Z}$. Пусть $k = 2n$, тогда если $x = \gamma^n$, то $x^2 = \gamma^{2n} = \gamma^k = a$ \\
	$a^{\frac{p-1}{2}}$ - решение $x^2 = 1$, т.к мы находимся в поле, то у этого уравнения не больше двух решений по теореме безу, оба решения мы знаем: $x=1, x=-1$ \\
	\textbf{Утверждение:} уравнение $x^2 = a$, где $a \neq 0$ имеет решения в $\mathbb{Z}/p\mathbb{Z} \iff a^{\frac{p-1}{2}} = 1$, иначе $a^{\frac{p-1}{2}} = -1$ \\
	\subsection{Символ Лежандра. Формула для символа Лежандра. Мультипликативность}
	\textbf{Определение:} символом Лежандра $\left(\frac{a}{p}\right)$ называется функция $\left(\mathbb{Z}/p\mathbb{Z}\right)^{\times} \to \{1, -1\}:$ 
	$$\left(\frac{a}{p}\right) = \begin{cases}
		1, x^2 = a \ \text{имеет решения} \\
		-1, x^2 = a \ \text{не имеет решений}
	\end{cases}$$
	$$\left(\frac{a}{p}\right) = a^{\frac{p-1}{2}}$$ 
	\textbf{Свойства:} 
	\begin{enumerate}
		\item $\left(\frac{a^2}{p}\right) = 1$
		\item $\left(\frac{ab}{p}\right) = \left(\frac{a}{p}\right) \left(\frac{b}{p}\right)$ - мультипликативность
	\end{enumerate}
	\subsection{Подсчет \texorpdfstring{$\left(\frac{-1}{p}\right)$}{(-1/p)}}
	$x^2 = -1 \mod p$,
	\begin{itemize}
		\item если имеет решение, то $\left(-1\right)^{\frac{p-1}{2}} = 1 \iff \frac{p-1}{2} \in 2\mathbb{Z} \iff p-1 \in 4\mathbb{Z} \iff p = 1 \mod 4$
		\item если не имеет решения, то $\left(-1\right)^{\frac{p-1}{2}} = -1 \iff \frac{p-1}{2} \not in 2\mathbb{Z} \iff \frac{p-1}{2} = 2m + 1, m \in Z \iff p-1 = 4m + 2 \iff p = 4m + 3 \mod p \iff p = 3 \mod p$
	\end{itemize} 
	$$\left(\frac{-1}{p}\right) - \begin{cases}
		1, p = 1 \mod 4 \\
		-1, p = 3 \mod 4
	\end{cases}$$
	\section{Формула Гаусса для символа Лежандра (с доказательством). Вычисление \texorpdfstring{$\left(\frac{2}{p}\right)$}{2/p}
	}
	\subsection{Лемма:} $a \in \left(\mathbb{Z}/p\mathbb{Z}\right), p \notin 2\mathbb{Z}$.
	Рассмотрим $\{a, 2a, 3a, \dots, \frac{p-1}{2} a\}$ по модулю $p = \{r_1, r_2,\dots, r_{p-1}\}, 1 < r_i < p-1$. Назовем такое множество $S$, пусть $L = |\{x\in S: x > \frac{p}{2}\}|$, тогда $\left(\frac{a}{p}\right) = \left(-1\right)^{L}$ \\
	\textbf{Доказательство:} 
	$S = \{u_1, \dots, u_L\} \cup \{v_{L+1}, \dots, v_{\frac{p-1}{2}}\}$, где $\frac{p}{2} < u_i < p - 1$ и $v_j < \frac{p}{2}$ \\
	Перемножим все элементы в $S$: \\
	$\prod\limits_{s\in S} s \underset{p}{\equiv} a \cdot 2a \cdot 3a \cdot \dots \cdot \frac{p-1}{2}a \underset{p}{\equiv} \left(\frac{p-1}{2}\right)!a^{\frac{p-1}{2}}$ \\
	Рассмотрим $\{p-u_1, p-u_2, \dots, p-u_L\}, 1 \leq p - u_i < \frac{p}{2}$ \\
	\textit{Лемма:} все элементы $\{p-u, \dots, p-u_L\} \cup \{v_{L+1}, \dots, v_{\frac{p-1}{2}}\}$ попарно различны \\
	\textit{Доказательство:} \begin{itemize} 
		\item 1 случай: $p - u_i = p - u_j \implies u_i = u_j \implies k_ia = k_ja \implies k_i = k_j \implies$ противоречие 
		\item 2 случай: $v_i  = v_j$ аналогично первому случаю
		\item 3 случай: $p - u_i = v_j \implies u_i + v_j \underset{p}{\equiv} 0 \implies k_ia + k_ja \underset{p}{\equiv} 0 \implies k_i + k_j \underset{p}{\equiv} 0$ но $k_i \in\left[1;\frac{p-1}{2}\right] \implies k_i + k_j \in \left[2, p-1\right]$
	\end{itemize}
	В $\{p - u_i, \dots, p-u_L\} \cup \{v_{L+1}, \dots, v_{\frac{p-1}{2}}\}$ все элементы $\in \left[1;\frac{p-1}{2}\right]$ \\
	Их $\frac{p-1}{2}$ штук и они все различны по нашей лемме, тогда они просто равны $\{1,2,\dots, \frac{p-1}{2}\}$ \\
	$\prod\limits_{r \in \{1, \dots, \frac{p-1}{2}\}}r = \left(\frac{p-1}{2}\right)! = \prod\limits_{i = 1}^{L} \left(p - u_i\right)\prod\limits_{j=L+1}^{\frac{p-1}{2}}v_j = \left(-1\right)^L \prod\limits_{i = 1}^{L}u_i \prod\limits_{j=L+1}^{\frac{p-1}{2}}v_j = \left(-1\right) ^ {L} \prod\limits_{s\in S} s = \left(\frac{p-1}{2}\right)a^{\frac{p-1}{2}}\left(-1\right)^L$ \\
	$\left(\frac{p-1}{2}\right)! \left(-1\right)^L = \left(\frac{p-1}{2}\right)! a^{\frac{p-1}{2}}$, так как $\left(\frac{p-1}{2}\right)! \neq 0 \mod p$ на них можно сократить: $$a^{\frac{p}{2}} = \left(-1\right)^L \implies \left(\frac{a}{p}\right) = \left(-1\right)^L$$
	\subsection{Вычисление \texorpdfstring{$\left(\frac{2}{p}\right)$}{(2/p)}}
	$\left\{2, 4, 6, \dots, p-1\right\} = S$ \\
	$L = \frac{p-1}{2} - \lfloor\frac{p}{4}\rfloor$ \\
	$\left(\frac{2}{p}\right) = \left(-1\right)^L$ \\
	Пусть $p = 1 \mod 8 \iff p = 1 + 8t$ \\
	$L = 4t - 2t \in 2\mathbb{Z}$ \\
	Пусть $p = 3 \mod 8 \iff p = 3 + 8t$ \\
	$L = 1 + 4t - 2t \notin 2\mathbb{Z}$ \\
	Пусть $p = 5 \mod 8 \iff p = 5 + 8t$ \\
	$L = 2 + 4t - 2t - 1 \notin 2\mathbb{Z}$ \\
	Пусть $p = 7 \mod 8 \iff p = 7 + 8t$ \\
	$L = 3 + 4t - 1 - 2t \in 2\mathbb{Z}$ \\
	Можно подобрать функцию, которая ведет себя таким же образом: $\frac{p^2 - 1}{8}$
	$$\left(\frac{2}{p}\right) = \left(-1\right)^{\frac{p^2-1}{8}}$$
	\section{Квадратичный закон взаимности*.}
	\textbf{Формулировка:} $$\left(\frac{q}{p}\right) = \left(-1\right)^{\frac{p-1}{2} \cdot \frac{q-1}{2}} \cdot \left(\frac{p}{q}\right), p \neq q$$
	\section{Теорема Гаусса. Теорема о мультипликативной группе поля.}
	\subsection{Теорема Гаусса}
	\textbf{Теорема:} $$ n = \sum\limits_{d | n, d \geq 1}\varphi\left(d\right), \forall n \geq 1$$
	\textbf{Доказательство:} \\ $S = \left\{1, \dots, n\right\} = \bigcup\limits_{d | n, d \geq 1}S_d$ \\ $S_d = \left\{k \in S: \gcd\left(k, n\right) = d\right\}$, $S_d \cap S_d' \neq \varnothing \implies d = d'$ \\
	$|S_d| = |\left\{k \in S : \gcd \left(\frac{k}{d}, \frac{n}{d}\right) = 1\right\} = \varphi\left(\frac{n}{d}\right)$ \\
	$n = |S| = \sum\limits_{d | n, d \geq 1}|S_d| = \sum\limits_{d | n, d \geq 1}\varphi\left(\frac{n}{d}\right) = \sum\limits_{d | n, d \geq 1}\varphi\left(d\right)$
	\subsection{Теорема о мультипликативной группе поля}
	\textbf{Теорема:} пусть $F$ - конечное поле, тогда $F^\times = F \setminus \{0\}$ - циклическая группа, то есть $\exists a \in F^\times : \forall g \in F^\times \exists k \in \mathbb{N}: g = a^k$ \\
	\textbf{Доказательство:} $G = F^\times, \psi\left(d\right) =$ количество элементов $G$ порядка $d$ $$g \in G \text{ имеет порядок } d, \text{ если } g^d = 1, \text{ но } g^k \neq 1 \forall k \in \left[1;d-1\right]$$ Пусть мы нашли $h \in G$ порядка $d$, рассмотрим $\left\langle h\right\rangle = \{1, h, h^2, h^3, \dots, h^{d-1}\}$ - сколько тут элементов порядка $d$? $$h^k \in \left\langle h\right\rangle \text{ имеет порядок } d \iff \left(h^k\right)^m \text{ закрасит всю группу}$$
	Когда $\left(h^k\right)^m = h$? $h^{km}=h^1 \iff km = 1 \mod d \iff \gcd \left(k, d\right) = 1$ Получается в $\left\langle h\right\rangle$ ровно $\varphi\left(d\right)$ элементов, которые имеют порядок $d$. Если мы находим хотя бы один элемент порядка $d$, то автоматически находим таких $\varphi\left(d\right)$ штук. \\
	Любой элемент $\left\langle h\right\rangle$ удовлетворяет уравнению $x^d=1$. Так как мы существуем в поле, у этого уравнения $\leq d$ различных решений, значит $\left\langle h\right\rangle$ - все решения $x^d = 1$. \\
	Пусть $t$ - элемент порядка $d$, тогда $t$ решает $x^d = 1 \implies t \in \left\langle h\right\rangle$ $$\psi\left(d\right) = \left[\begin{array}{l}
		0, \text{ если не нашли} \\
		\varphi\left(d\right), \text{ иначе}
	\end{array}\right.\implies \psi\left(d\right) \leq \varphi\left(d\right)$$ $n = \sum\limits_{d | n} \psi\left(d\right) \leq \sum\limits_{d | n}\varphi\left(d\right) = n \implies \psi\left(d\right) = \varphi\left(d\right) \forall d | n$ \\
	Для $d = n \psi\left(n\right) = \varphi(n) \geq 1 \implies$ нашли хотя бы 1 элемент порядка $n$ \\
	\textbf{Лемма:} $G$ - конечная группа, $|G| = n$, $\alpha \in G$, тогда $ord \ \alpha | n$ \\
	\textbf{Доказательство:} $C = \left\langle\alpha\right\rangle$, $|C| = ord \ \alpha$, $G = \bigcup\limits_{g \in G} gC$, это действительно равенство, т.к $\forall g \in G g \in gC$. $$gC \cap g'C = \left[\begin{array}{l}
		\varnothing \\
		gC = g'C
	\end{array}\right.$$ Пусть $z \in gC \cap g'C$ \\ $z = gc_1 = g'c_2 \implies g = g'c_2c^{-1}_1$ \\
	Пусть $h \in gC \implies h = gc = g'c_2c^{-1}_1c \in g'C \implies gC \subset g'C$, аналогично можно доказать в обратную сторону, тогда $gC = g'C$ \\
	$n = |G| = |C \sqcup g_1C \sqcup g_2C \sqcup \dots \sqcup g_kC| = |C| + |g_1C| + \dots + |g_kC| = ord \ \alpha + ord \ \alpha + \dots + ord \ \alpha = \left(k + 1\right) ord \ \alpha \implies ord \ \alpha | n$
	\section{Критерий максимальности для идеалов в k[x]. Построение поля
		комплексных чисел. Алгебраическая запись.}
	\subsection{Критерий максимальности для идеалов в k[x]}
	$k\left[x\right]$ - ОГИ \\
	\textbf{Утверждение: } идеал $\left(f\right) \text{ - максимален } \iff f \text{ - неприводим}$ \\
	\textbf{Доказательство:} \begin{enumerate}
		\item Достаточность: \\
		Пусть $f = g \cdot h$ и $\deg g \geq 1, \deg h \geq 1$, $\left(f\right) \nsubseteqq \left(g\right)$, так как $g \in \left(g\right)$ и $ g \notin \left(f\right)$, но $\left(g\right) \notin R \implies \left(f\right)$ - не максимален, противоречие, следовательно $f$ - неприводим. 
		\item Необходимость: \\
		Пусть $\left(f\right) \nsubseteqq J, J$ - главный идеал $\implies J = \left(g\right)$
		$$\left(f\right) \nsubseteqq \left(g\right) \implies g | f \implies \left[\begin{array}{l}
			g \equiv const \ \neq 0 \implies \left(g\right) k\left[x\right] \\
			g \sim f \implies \left(f\right) = \left(g\right) \text{, но по условию не равны}
		\end{array}\right.$$
	\end{enumerate}
	\subsection{Построение поля комплексных чисел. Алгебраическая запись.}
	Возьмем кольцо многочленов и факторизуем его по неприводимому $x^2 + 1$, таким образом получается новое поле комплексных чисел, это действительно поле, так как идеал максимален.
	$$ R\left[x\right]/\left(x^2+1\right)  = \mathbb{C}$$
	$$ \mathbb{C} = \left\{a + bx: a,b \in \mathbb{R}\right\}$$ 
	И если заменить $x$ на $i$ то получим алгебраическую запись: $a + bi$
	\section{Классификация автоморфизмов \texorpdfstring{$\mathbb{C}$}{C} над \texorpdfstring{$\mathbb{R}$}{R}, модуль комплексного числа. Его мультипликативность. Оценка суммы.}
	\subsection{Классификация автоморфизмов \texorpdfstring{$\mathbb{C}$}{C} над \texorpdfstring{$\mathbb{R}$}{R}} 
	\textbf{Определение:} автоморфизмом $\mathbb{C}$ над $\mathbb{R}$ называется изоморфизм $$\sigma: \mathbb{C} \to \mathbb{C}: \forall r \in \mathbb{R} \sigma\left(r\right) = r$$
	\textbf{Утверждение:} существует два автоморфизма $\mathbb{C}$ над $\mathbb{R}$: $id, -$, $Gal(\mathbb{C}/\mathbb{R}) \overset{\sim}{=} \mathbb{Z}/2\mathbb{Z}$ \\
	\textbf{Доказательство:} Пусть $\sigma$ - автоморфизм $\mathbb{C}$ над $\mathbb{R}$ \\
	$\sigma\left(a + ib\right) = \sigma\left(a\right) + \sigma\left(i\right)\sigma\left(b\right) = a + \sigma\left(i\right)b$ \\
	$\sigma\left(i\right)^2 = \sigma\left(i^2\right) = \sigma(-1) = -1$
	$$ \text{так как } \sigma\left(i\right) \text{ - решение } x^2 = -1 \sigma\left(i\right) = \left[\begin{array}{l}
		i \implies id \\
		-i
	\end{array}\right.$$
	$$ \text{сопряжение $\overline{a + ib}$ : } \sigma\left(a + ib\right) = a - ib$$
	Свойство: $z \in \mathbb{C}, \overline{z} = z \iff z \in \mathbb{R}$
	\subsection{Модуль комплексного числа. Его мультипликативность. Оценка суммы}
	Хотим построить отображение $\mathbb{C} \to \mathbb{R}$
	$\frac{z + \overline{z}}{2} - $ Re$z$, вещественная часть \\
	$\frac{z - \overline{z}}{2i} - $ Im$z$, мнимая часть \\
	$\overline{z \cdot \overline{z}} = z \cdot \overline{z} \implies z \cdot \overline{z} \in \mathbb{R}$
	$z \cdot \overline{z} = (a +ib)(a - ib) = a^2 + b^2 \geq 0$
	$|z| = \sqrt{z \cdot \overline{z}}$ \\
	\textbf{Свойства модуля:}
	\begin{enumerate}
		\item $|z_1 \cdot z_2| = \sqrt{z_1z_2\overline{z_1}\overline{z_2}} = \sqrt{z_1\overline{z_1}} \cdot \sqrt{z_2\overline{z_2}} = |z_1| \cdot |z_2|$
		\item $|z_1 + z_2| \leq |z_1| + |z_2|$ - неравенство треугольника для векторов $\left(a, b\right)$ и $\left(c, d\right)$, если $z_1 = a + ib, z_2 = c + id$
	\end{enumerate}
	\section{Основная теорема алгебры*.}
	\textbf{Теорема: } $$p\left(z\right) \in \mathbb{C}\left[z\right], \deg p \geq 1 \implies \exists z_0 \in \mathbb{C} : p\left(z_0\right) = 0$$
	\textbf{Следствие: } $$ p \in \mathbb{C}\left[z\right], \deg p = n \implies p \text{ имеет ровно $n$ корней}$$
	\section{Эквивалентность трех определений базиса конечномерного векторного пространства.}
	\textbf{Определение:} $B \subset V$. Говорят, что $B$ - базис, если $B$ порождающее и линейно независимое. \\
	\textbf{Теорема:} Следующие условия равносильны \begin{enumerate}
		\item $B$ - базис 
		\item $B$ - максимальное по включению линейно независимое множество
		\item $B$ - минимальное по включению порождающее множество
	\end{enumerate}
	\textbf{Доказательство:} \begin{itemize}
		\item $1 \implies 2$ \\
		Пусть $B$ - базис, $v \in V$, тогда $v = \sum\limits_{i=1}^{n}\alpha_ie_i, e_i \in B$. \\
		Тогда $B \cup \left\{v\right\}$ - линейно зависимое $\implies B$ - максимальное по включению линейно независимое.
		\item $2 \implies 1$ \\
		Нам надо показать, что $B$ - порождающее. \\
		Пусть $v \in V\setminus B$, тогда $B \cup \left\{v\right\}$ -линейно зависимое $\implies \sum\limits_{i = 1}^{n}\alpha_ie_i + \beta v = 0, e_i \in B$ \\
		$\beta \neq 0$ так как $B$ - линейно независимое, тогда $v = -\sum\limits_{i = 1}^{n}\frac{\alpha_i}{\beta}e_i$
		\item $1 \implies 3$ \\
		Пусть $B$ - базис. Предположим, что $B\setminus\left\{e\right\}$ все еще порождающее. Тогда $e = \sum\limits_{i = 1}^{n}\alpha_ie_i, e_i \in B\setminus\left\{e\right\}$, но $B$ должно быть линейно независимым.
		\item $3 \implies 1$ \\
		Покажем, что $B$ - линейно независимое. Пусть $\sum\limits_{i = 1}^n\alpha_ie_i = 0$, где $\exists \alpha_k \neq 0$, тогда $$e_k = -\sum\limits_{i = 1, i\neq k}^n \frac{\alpha_i}{\alpha_k}e_i$$
		Тогда $B \setminus \left\{e_k\right\}$ все еще порождающее.
	\end{itemize}
	\section{Дополнение линейно независимого множества до базиса.}
	Пусть $V$ - векторное пространство, $S$ - линейное независимое множество, тогда $\exists B$ - базис, такой что $S \subset B$. \\
	\textbf{Доказательство:} $A = \left\{\text{подмножества в $V$, которые содержат $S$ и линейно независимые}\right\}, A \neq \varnothing$, т.к $S \in A$, $A$ - частично упорядоченное множество по $\leq: x\leq x` \leq \dots$ \\
	Рассмотрим произвольную цепочку $C$ множества $A$: $x\leq x' \leq x'' \leq \dots$, назовем $A_c = \bigcup\limits_{x \in C}X$, почему $A_c \in A$? \\
	\begin{enumerate}
		\item $\subseteq S$
		\item пусть $\left\{e_1, \dots, e_k\right\} \subset A_c$, тогда $\exists X \in C: \left\{e_1, \dots, e_n\right\} \subset X$, но $X$ - линейно независимое, тогда $A_c$ тоже линейно независимое, тогда $A_c \in A$ 
	\end{enumerate}
	Заметим, что $\forall X \in C X \subseteq A_c$ \\
	\textbf{Лемма Цорна:} если в частично упорядоченном множестве для любой цепочки есть верхняя граница, то существует максимальный элемент \\
	По лемме Цорна существует $B \subset A_c$ - максимальный элемент, тогда $S\subseteq B$ и $B$ максимальное по включению линейно независимое множество, которое содержит $S$, тогда оно вообще максимальное по включению линейно независимое, а значит является базисом
	\section{Лемма Стейница (лемма о подмене).}
	\textbf{Лемма:} $V$ - векторное пространство. Пусть $S = \left\{w_1, \dots, w_m\right\}$ - порождающее множество $V$, $L = \left\{v_1, \dots, v_n\right\}$ - линейно независимое множество $V$. \\
	\textbf{Тогда:} \begin{enumerate}
		\item $n \leq m$
		\item после перенумерования $S$ множество $\left\{v_1, v_2, \dots, v_n, w_{n+1}, \dots, w_m\right\}$ все еще порождающее.
	\end{enumerate}
	\textbf{Доказательство:} по индукции по $n$ \\
	База: При $n = 0$ верно (ничего не меняем) \\
	Переход: пусть для $n-1$ векторов уже знаем. \\
	Перенумеруем $S$ и заменим $n-1$ элементов, получим $S'=\left\{v_1, \dots, v_{n-1}, w_n, w_{n+1}, \dots, w_m\right\}$ \\
	Так как $S'$ - порождающее множество, $v_n$ можно выразить из его элементов. $$v_n = \sum\limits_{i = 1}^{n-1}\alpha_i v_i + \sum\limits_{i = n}^{m}\beta_i w_i$$
	Мы хотим выразить $w_n$, чтобы заменить его на $v_n$ 
	\begin{itemize}
		\item 1 случай: $\beta_i = 0 \implies \sum\limits_{i = 1}^{n-1}\alpha_i v_i - v_n = 0$, но это невозможно из-за того, что $L$ - линейно независимое множество
		\item 2 случай: $\exists \beta_i \neq 0$, перенумеруем $S'$ так, чтобы $\beta_n \neq 0$, тогда $$w_n = \frac{v_n}{\beta_n} - \sum\limits_{i = 1}^{n-1}\frac{\alpha_i}{\beta_n}v_i - \sum\limits_{i = n + 1}^{m} \frac{\beta_i}{\beta_n}w_i$$
	\end{itemize}
	Рассмотрим множество $S'' = \left\{v_1, \dots, v_{n-1}, w_{n+1}, \dots, w_m\right\}$\\
	\textit{Замечание:} если $S' = \left\{v_1, \dots, v_{n-1}\right\}$, то $v_n = \sum\limits_{i = 1}^{n-1}\alpha_i v_i$, но $L$ - линейно независимое. \\
	Утверждаем, что $S''$ - порождающее. Так как $S'$ - порождающее, $v \in V$ $$v = \sum\limits_{i = 1}^{n-1}\alpha_i v_i + \gamma_n w_n + \sum\limits_{i = n + 1}^{m}\gamma_iw_i$$
	Теперь мы можем подставить вместо $w_n$ наше выражение из второго случая, таким образом любой вектор выражается из элементов множества $S'' \implies S''$ - порождающее
	
	\section{Корректность размерности: любые два базиса конечномерного векторного пространства имеют одинаковую мощность.}
	\textbf{Определение:} пусть для векторного пространства $V$ существует конечный базис $B$, тогда назовем $\dim V  = |B|$ - размерность $V$ \\
	\textbf{Утверждение:} Определение размерности корректно \\
	\textbf{Доказательство:} $B, B'$ - базисы $V$ и $B$ - конечный \\
	$B$ - порождающий и $B'$ линейно независимый $\implies \forall \text{ конечного подмножества } B'$ линейно независимо, пусть $S \subset B'$ - конечноe подмножество, тогда $|S| \leq |B| \forall S \subset B'$ \\
	Тогда $|B'| \leq |B|$, теперь скажем, что $B$ - линейно независимый и $B'$ - порождающий, тогда $|B| \leq |B'| \implies |B| = |B'|$
	\section{Формула Грассмана. Антипример. Критерий прямой суммы.}
	\subsection{Формула Грассмана}
	\textbf{Теорема:} если $V = U + W$, где все конечномерны, то $$\dim \left(U + W\right) = \dim U + \dim W - \dim U \cap \dim W$$
	\textbf{Доказательство:} пусть $\left\{e_1, \dots, e_k\right\}$ - базис $U \cap W$, т.к $U \cap W \subset U$, базис $U \cap W$ можно расширить до базиса $U$ - $\left\{e_1, \dots, e_k, u_1, \dots, u_m\right\}$. Аналогично для $W$ - $\left\{e_1, \dots, e_k, w_1, \dots, w_n\right\}$, тогда $\left\{e_1, \dots,e_k, u_1, \dots, u_m, w_1, \dots, w_n\right\}$ - базис $U + W$ 
	\begin{itemize}
		\item порождаемость: пусть $u + w \in U + W, u = \sum\alpha_ie_i + \sum \beta_iu_i, w = \sum \mu_ie_i + \sum \gamma_iw_i \implies u + w = \sum \left(\mu_i + \alpha_i\right)e_i + \sum \beta_iu_i + \sum\gamma_iw_i$
		\item линейная независимость: пусть $$ \sum\alpha_ie_i + \sum\beta_iu_i + \sum\gamma_iw_i = 0 \left(*\right)$$
		$$ \sum\beta_iu_i = -\sum\alpha_ie_i - \sum\gamma_iw_i$$
		$\sum\beta_iu_i \in U, -\sum\alpha_ie_i - \sum\gamma_iw_i \implies \sum\beta_iu_i \in U\cap W\implies \sum\beta_iu_i = \sum s_i e_i \implies \sum\beta_iu_i - \sum s_ie_i = 0\implies \beta_i = 0 \forall i = 1\dots n$ \\
		$\left(*\right) \sum\alpha_ie_i + \sum\gamma_iw_i = 0 \implies \alpha_i = 0 \forall i = 1\dots k, \gamma_i = 0 \forall i = 1\dots n$
	\end{itemize}
	$$\dim\left(U + W\right) = k + n + m = \left(k + n\right) + \left(k + m\right) - k = \dim W + \dim U - \dim U \cap W$$
	\subsection{Антипример}
	Допустим у нас есть пространства в $\mathbb{R}^3$: \\
	$U_1 = \left\langle x, y\right\rangle, U_2 = \left\langle x + z\right\rangle, U_3 = \left\langle z\right\rangle$, $$\dim\left(U_1 + U_2 + U_3\right) = 3 \neq 2 + 1 + 1 - 0 - 0 - 0 + 0 = 4$$
	\subsection{Критерий прямой суммы}
	\textbf{Определение:} сумма векторных подпространств называется прямой, если $U \cap W = 0$ $$U \oplus W$$ \\
	\textbf{Утверждение:} сумма $V + W$ - прямая $ \iff \forall v \in U + W \exists!u \in U \exists! w \in W: v = u + w$ \\
	\textbf{Доказательство:} \begin{itemize}
		\item Достаточность: \\
		$v = u_1 + w_1 = u_2 + w_2 \implies u_1 - u_2 \in U = w_1 - w_2 \in W, U\cap W =0 \implies u_1 = u_2, w_1 = w_2$
		\item Необходимость: \\
		$v \in U \cap W, v = v + 0 = 0 + v$, где сначала $v \in U$, затем $v \in W$, тогда по единственности разложения $v = 0$
	\end{itemize}
	\section{Теорема о размернорсти ядра и образа.}
	\textbf{Теорема:} $\mathcal{L}: V \to W$ - линейное отображение, тогда $\dim V = \dim \ker \mathcal{L} + \dim Im \ \mathcal{L}$ \\
	\textbf{Доказательство:} пусть $e_1, \dots, e_k$ - базис $\ker \mathcal{L}$ - линейно независимое в $V \implies$ мы можем расширить до $e_1, \dots e_k, e_{k + 1}, \dots, e_n$ - базиса пространства $V$. \\ Покажем, что $\mathcal{L}\left(e_{k + 1}\right), \dots, \mathcal{L}\left(e_n\right)$ - базис $Im \ \mathcal{L}$ \begin{itemize}
		\item Порождаемость: \\
		пусть $w \in Im \ \mathcal{L} \implies \exists v \in V: w = \mathcal{L}\left(v\right)$, разложим $v$ по базису $V$: \\ $v = \sum\limits_{i = 1}^k\alpha_i e_i + \sum\limits_{i = k + 1}^n \alpha_ie_i$ \\
		$w = \mathcal{L}\left(v\right) = \mathcal{L}\left(\sum\limits_{i = 1}^k\alpha_i e_i + \sum\limits_{i = k + 1}^n \alpha_ie_i\right) = \sum\limits_{i = 1}^k\alpha_i\mathcal{L}\left(e_i\right) + \sum\limits_{i = k + 1}^n\alpha_i\mathcal{L}\left(e_i\right) = \sum\limits_{i = k + 1}^n\alpha_i\mathcal{L}\left(e_i\right)$, так как $e_1, \dots, e_k \in \ker \mathcal{L} \implies \mathcal{L}\left(e_i\right) = 0 \ \forall i = 1,\dots,k$
		\item Линейная независимость: \\
		Пусть $\sum\limits_{i = k + 1}^n\beta_i\mathcal{L}\left(e_i\right) = 0 = \mathcal{L}\left(\sum\limits_{i = k + 1}^n\beta_ie_i\right)\implies \sum\limits_{i = k + 1}^n\beta_ie_i \in \ker \mathcal{L}$, тогда распишем эту сумму по базису ядра: $\sum\limits_{i = k + 1}^n\beta_ie_i  = \sum\limits_{i = 1}^k\gamma_ie_i \implies \sum\limits_{i = k + 1}^n\beta_ie_i - \sum\limits_{i = 1}^k\gamma_ie_i = 0 \implies \beta_i = 0 \ \forall i = k+1, \dots, n, \gamma_i = 0 \ \forall i = 1, \dots, k$, так как это базис $V$ 
	\end{itemize}
	$$ \dim V = n = k + \left(n - k\right) = \dim \ker \mathcal{L} + \dim Im \ \mathcal{L}$$
	
	\section{Матрица линейного отображения. Изоморфизм \texorpdfstring{$Hom\left(k^n,k^m\right)\overset{\sim}{=}M_{m\times n} \left(k\right)$}{HOM(k^n,k^m) = M_mxn(k)}. Матрица композиции операторов.}
	\subsection{Матрица линейного отображения.}
	\textbf{Определение:} $\mathcal{L}: V \to W$ - линейное отображение, $e = \left(e_1, \dots, e_n\right)$ - базис $V$, $f = \left(f_1, \dots, f_m\right)$ - базис $W$ $$ \left[\mathcal{L}\right]_{e,f} \text{ - матрица линейного отображения $\mathcal{L}$ в базисах $e$ и $f$}$$
	Нам достаточно определить только как отображение действует на базисных векторах: $\mathcal{L}\left(e_i\right)$ - какой то вектор в $W$, значит мы можем выразить его в базисе $W$ \\
	$\mathcal{L}e_1 = a_{11}f_1 + a_{21}f_2 + a_{31}f_3 + \dots + a_{m1}f_m$ \\
	$\mathcal{L}e_2 = a_{12}f_1 + a_{22}f_2 + a_{32}f_3 + \dots + a_{m2}f_m$ \\
	\vdots \\
	$\mathcal{L}e_j = a_{1j}f_1 + a_{2j}f_2 + a_{3j}f_3 + \dots + a_{mj}f_m$ 
	$$\left[\mathcal{L}\right]_{e,f} = \begin{pmatrix}
		a_{11} & a_{12} & \cdots & a_{1n} \\
		a_{21} & a_{22} & \cdots & a_{2n} \\
		\vdots  & \vdots  & \ddots & \vdots  \\
		a_{m1} & a_{m2} & \cdots & a_{mn}
	\end{pmatrix}$$
	\subsection{Изоморфизм \texorpdfstring{$Hom\left(k^n,k^m\right)\overset{\sim}{=}M_{m\times n} \left(k\right)$}{HOM(k^n,k^m) = M_mxn(k)}.} Введем такое отображение $\pi:Hom\left(k^n, k^m\right) \to M_{m \times n}\left(k\right)$. Так как $k^n \overset{\sim}{=} V, \dim V = n$ и $k^m \overset{\sim}{=} W, \dim W = m$, будем использовать векторные пространства $V$ и $W$. Нам нужно доказать, что $\pi$ - биекция и линейное отображение. Пусть $e_1, \dots, e_n$ - базис $V$, $f_1, \dots, f_m$ - базис $W$. Возьмем линейное отображение $\mathcal{L} \in Hom\left(V, W\right)$. Подействуем им на базис $V$ и выразим результат через базис $W$: $\mathcal{L}e_j = \sum\limits_{i = 1}^ma_{ij}f_i$. Наше отображение $\pi\left(\mathcal{L}\right) = \left(a_{ij}\right) \in M_{m \times n}$. Покажем, что $\pi$ - линейное отображение: \begin{enumerate}
		\item $\mathcal{L} \in Hom\left(V, W\right), \mathcal{T} \in Hom\left(V, W\right)$ \\
		Определим $\pi\left(\mathcal{L} + \mathcal{T}\right) = \pi\left(\mathcal{L}\right) + \pi\left(\mathcal{T}\right)$ как просто поэлементное суммирование элементов двух матриц.
		\item $\mathcal{L} \in Hom\left(V, W\right), \gamma \in k$ \\
		Определим $\pi\left(\gamma\mathcal{L}\right) = \gamma\pi\left(\mathcal{L}\right)$ как умножение каждого элемента матрицы на элемент из поля.
	\end{enumerate}
	Теперь нам надо доказать биекцию. \begin{itemize}
		\item Инъекция \\
		пусть $\mathcal{L} \in \ker \pi$, тогда $\pi\left(\mathcal{L}\right) = 0$, тогда $\mathcal{L}e_i = 0 \ \forall i = 1, \dots, n$, тогда $\mathcal{L}v = 0 \ \forall v \in V \implies \ker \pi = 0 \implies \pi$ - инъекция
		\item Сюръекция \\
		Возьмем любую матрицу $A$ из $M_{m\times n}$. Пусть у нас есть $\mathcal{L}\left(e_j\right) = \sum\limits_{i = 1}^ma_{ij}f_i$. По свойству базиса если мы можем отправить базис в область значений, то существует единственное такое линейное отображение. Тогда $\pi\left(\mathcal{L}\right) = A$
	\end{itemize}
	\subsection{Матрица композиции операторов.}
	Мы можем определить эту операцию только когда $V = W$ в $Hom\left(V, W\right)$ Пусть у нас есть две матрицы: $$A_{m\times n} = \begin{pmatrix}
		a_{11} & a_{12} & \cdots & a_{1n} \\
		a_{21} & a_{22} & \cdots & a_{2n} \\
		\vdots  & \vdots  & \ddots & \vdots  \\
		a_{m1} & a_{m2} & \cdots & a_{mn}
	\end{pmatrix}$$
	$$B_{n\times k} = \begin{pmatrix}
		b_{11} & b_{12} & \cdots & b_{1k} \\
		b_{21} & b_{22} & \cdots & b_{2k} \\
		\vdots  & \vdots  & \ddots & \vdots  \\
		b_{n1} & b_{n2} & \cdots & b_{nk}
	\end{pmatrix} $$
	композиция в данном случае аналогична умножению матриц, когда мы берем строку из $A$ и поэлементно умножаем на столбец из $B$, то есть наш элемент в $AB$ задается следующим образом: $$ab_{ik} = \sum\limits_{j = 1}^n a_{ij}b_{jk}$$
	Новая матрица имеет размер $m \times k$
	\section{Замена базиса (в области определения и области значения). Канонический вид матрицы линейного отображения.}
	\subsection{Замена базиса}
	$\mathcal{L}: V \to W$, $e$ - базис $V$ размерности $n$, $f$ - базис $W$. Тогда построим матрицу $$ \left[\mathcal{L}\right]_{e, f} = \begin{pmatrix}
		\alpha_{11} & \alpha_{12} & \cdots & \alpha_{1n} \\
		\alpha_{21} & \alpha_{22} & \cdots & \alpha_{2n} \\
		\vdots & \vdots & \ddots & \vdots \\
		\alpha_{m1} & \alpha_{m2} & \cdots & \alpha_{mn}
	\end{pmatrix}$$
	Теперь мы хотим например заменить базис $e$ на базис $r$. Запишем матрицу перехода $T_1$, в которой $i$ строка - разложение $r_i$ базисного вектора по базисным векторам $e$.
	$$ T_1 = \begin{pmatrix}
		\gamma_{11} & \gamma_{12} & \cdots & \gamma_{1n} \\
		\gamma_{21} & \gamma_{22} & \cdots & \gamma_{2n} \\
		\vdots & \vdots & \ddots & \vdots \\
		\gamma_{n1} & \gamma_{n2} & \cdots & \gamma_{nn}
	\end{pmatrix}
	$$
	$$\left[\mathcal{L}\right]_{e,f} T_1 = \left[L\right]_{r, f}$$
	Аналогично хотим заменить базис $f$ на базис $s$
	$$ T_2 = \begin{pmatrix}
		\beta_{11} & \beta_{12} & \cdots & \beta_{1m} \\
		\beta_{21} & \beta_{22} & \cdots & \beta_{2m} \\
		\vdots & \vdots & \ddots & \vdots \\
		\beta_{m1} & \beta_{m2} & \cdots & \beta_{mm}
	\end{pmatrix}
	$$
	$$T_2\left[\mathcal{L}\right]_{e,f} = \left[\mathcal{L}\right]_{e,s}$$
	\subsection{Канонический вид матрицы линейного отображения.}
	\textbf{Теорема:} $\mathcal{L}: V \to W$ - линейное отображение, Тогда $\exists$ базис $e_1, \dots, e_n$ пространства $V$ и базис $f_1, \dots, f_m$ пространства $W$, такое что: $$\left[\mathcal{L}\right]_{e,f} = \begin{pmatrix}
		1 & 0 & 0 & 0 & 0 &\cdots & 0 \\
		0 & 1 & 0 & 0 & 0 &\cdots & 0 \\
		0 & 0 & \ddots & 0 & 0 &\cdots & 0\\
		0 & 0 & 0 & 1 & 0 &\cdots & 0 \\
		0 & 0 & 0 & 0 & 0 & \cdots & 0 \\
		\vdots & \vdots & \vdots & \vdots & \vdots & \ddots & \vdots \\
		0 & 0 & 0 & 0 & 0 & \cdots & 0
	\end{pmatrix}$$
	\textbf{Доказательство:} пусть $e_{k + 1}, \dots, e_n$ - базис $\ker \mathcal{L}$ дополним этот базис до базиса $V$. $\mathcal{L}\left(e_1\right), \dots, \mathcal{L}\left(e_k\right)$ - базис $Im \ \mathcal{L} = f_1, \dots, f_k$, достроим до базиса $W$. \\ Теперь попытаемся построить матрицу: \\ $\mathcal{L}e_1 = 1f_1 + 0f_2 + \dots \\ \mathcal{L}e_2 = 0f_1 + 1f_2 + 0f_3 + \dots \\ \vdots \\ \mathcal{L}e_k = 0f_1 + 0f_2 + \dots + 1f_k$ \\
	Дальше мы будем обращаться к базисным векторам из ядра, следовательно везде будут нули.
	\section{Универсальное свойство базиса.}
	$V$ - векторное пространство, $e_1, \dots, e_n$ - базис $V$, пусть $\mathcal{L} \in Hom\left(V, W\right)$. $$\mathcal{L}\left(v\right) = \mathcal{L}\left(\sum\limits_{i = 1}^n\alpha_ie_i\right) = \sum\limits_{i = 1}^n\alpha_iL\left(e_i\right)$$
	\textbf{Утверждение:} $\forall f\left(e_1, \dots, e_n\right) \to W \ \exists! \text{ линейное отображение } \mathcal{L}:V\to W: \mathcal{L}\left(e_i\right) = f\left(e_i\right)$ \\
	\textbf{Доказательство:} $\mathcal{L}\left(v\right) = \mathcal{L}\left(\sum\limits_{i = 1}^n\alpha_ie_i\right)$. Определим $\mathcal{L}\left(v\right)$ как $\sum\limits_{i = 1}^n\alpha_if\left(e_i\right)$, теперь покажем, что это линейное отображение. 
	\begin{enumerate}
		\item $\mathcal{L}\left(v + \widetilde{v}\right) = \mathcal{L}\left(\sum\limits_{i = 1}^n\alpha_ie_i + \sum\limits_{i = 1}^n\widetilde{\alpha}_ie_i\right) = \mathcal{L}\left(\sum\limits_{i = 1}^n\left(\alpha_i + \widetilde{\alpha}_i\right)e_i\right) = \sum\limits_{i = 1}^n\left(\alpha_i + \widetilde{\alpha}_i\right)f\left(e_i\right) = \sum\limits_{i = 1}^n \alpha_if\left(e_i\right) + \sum\limits_{i = 1}^n\widetilde{\alpha}_if\left(e_i\right) = \mathcal{L}\left(v\right) + \mathcal{L}\left(\widetilde{v}\right)$
		\item $\mathcal{L}\left(\beta v\right) = \mathcal{L}\left(\beta\sum\limits_{i = 1}^n\alpha_ie_i\right) = \mathcal{L}\left(\sum\limits_{i = 1}^n\beta\alpha_ie_i\right) = \sum\limits_{i = 1}^n \beta\alpha_if\left(e_i\right) = \beta \sum\limits_{i = 1}^n \alpha_if\left(e_i\right) = \beta \mathcal{L}\left(v\right)$
	\end{enumerate} 
	Также покажем единственность такого отображения: пусть $\mathcal{L}$ и $\mathcal{T}$: \\ $\mathcal{L}\left(e_i\right) = f\left(i\right) \forall i = 1\dots n, \mathcal{T}\left(e_i\right) = f\left(i\right) \forall i = 1\dots n $ $$\left(\mathcal{L} - \mathcal{T}\right)\left(v\right) = \left(\mathcal{L} - \mathcal{T}\right)\left(\sum\limits_{i = 1}^n \alpha_ie_i\right) = \sum\limits_{i = 1}^n \alpha_if\left(e_i\right) - \sum\limits_{i = 1}^n \alpha_if\left(e_i\right) = 0 \ \forall v \in V$$
	\section{Двойственный базис. Изоморфизм \texorpdfstring{$V \text{ и } V^*$}{V и V^*} для конечномерного $V$. Канонический изоморфизм $V$ и $V^{**}$ для конечномерного $V$.}
	\subsection{Двойственный базис. Изоморфизм \texorpdfstring{$V \text{ и } V^*$}{V и V^*} для конечномерного $V$.}
	\textbf{Определение:} $V$ - векторное пространство, $V^* = Hom\left(V, k\right)$ - пространство линейных функционалов или двойственное пространство. \\ Пусть $e_1, \dots, e_n$ - базис $V$, тогда определим набор векторов $e_{1}^*, \dots, e_{n}^* \in V^*$ следующим образом: $$e^*_i\left(e_j\right) = \delta_{ij} = \begin{cases}
		1, i=j \\
		0, i\neq j
	\end{cases}$$ \\
	\textbf{Утверждение:} $e_1^*, \dots, e_n^* = e^*$ - базис $V^*$ \\
	\textbf{Доказательство:} \begin{itemize}
		\item порождаемость: \\
		Пусть $\varphi \in V^*$, мы хотим получить равенство $\varphi = \sum\limits_{i = 1}^n\beta_ie_i^*$. Посмотрим как ведет себя выражение с обеих сторон на базисном векторе $e_j$: $\varphi\left(e_j\right) = \beta_je^*_j\left(e_j\right) = \beta_j$, тогда можно утверждать, что $\varphi = \sum\limits_{i = 1}^n \varphi\left(e_i\right)e_i^*$. Это просто два линейных функционала, они равны, если равны на всех базисных векторах. $\varphi\left(e_i\right) = \beta_i = \varphi\left(e_i\right)e_i^*\left(e_i\right) = \varphi\left(e_i\right)$
		\item линейная независимость: \\
		Пусть $\sum\limits_{i = 1}^n \gamma_i e_i^* = 0$, подставим базисный вектор $e_j$: $\gamma_je^*_j\left(e_j\right) = \gamma_j = 0 \ \forall j = 1, \dots, n$
	\end{itemize}
	Из этого следует, что $V \overset{\sim}{=} V^*$
	\subsection{Канонический изоморфизм $V$ и $V^{**}$ для конечномерного $V$.}
	\textbf{Теорема:} пространства $V$ и $\left(V^*\right)^*$ канонически изоморфны. \\
	\textbf{Доказательство:} $V^{**} = Hom\left(Hom\left(V, k\right), k\right) \overset{\overset{!}{\sim}}{=} V$ - мы хотим построить такой изоморфизм. \\
	Построим $\zeta: V \to V^{**}$ следующим образом: $v \mapsto \left(\varphi \in V^* \mapsto \varphi\left(v\right)\right)$. Утверждаем, что $\zeta$ - изоморфизм. Так как у нас $\dim V = \dim V^* = \dim V^{**}$, нам достаточно просто доказать, что $\zeta$ - инъекция. Пусть $v \in \ker \zeta$, тогда отображение $\left(\varphi \mapsto \varphi\left(v\right)\right)$ - нулевое. Покажем, что $v = 0$. В пространстве $V^*$ существует линейный функционал $v^*: v^*\left(v\right) = 1$, если $v \neq 0$. Действительно, если $v \neq 0$, то мы можем построить базис $v,e_2, \dots, e_n$ и двойственный базис $v^*, e_2^*, \dots, e_n^*$. Значит $v^*\left(v\right) \neq 0$, противоречие. Тогда $\ker \zeta = 0 \implies \zeta$ - инъекция, а значит и биекция.\\
	Также нам надо показать, что $\zeta$ - линейное отображение. \\
	$\zeta\left(v_1 + v_2\right) = \varphi\left(v_1 + v_2\right) = \varphi\left(v_1\right) + \varphi\left(v_2\right) = \zeta\left(v_1\right) + \zeta\left(v_2\right), v_1, v2 \in V, \varphi \in V^*$ \\
	$\zeta\left(\gamma v\right) = \varphi\left(\gamma v\right) = \gamma\varphi\left(v\right) = \gamma\zeta\left(v\right), v \in V, \gamma \in k$ \\
	Тогда $\zeta$ - биекция и линейное отображение, а следовательно - изоморфизм, причем канонический, так как мы не определяли базис.
	\section{Лемма о размерности аннулятора.}
	\textbf{Определение:} Пусть $W \subset V$ - подпространство, тогда $Ann W = \left\{\varphi \in V^*: \varphi\left(w\right) = 0 \ \forall w \in W\right\}$
	\textbf{Лемма:} $W \subset V$, тогда $\dim W + \dim \ AnnW = \dim V$ \\
	\textbf{Доказательство:} пусть $e_1, \dots, e_k$ - базис W, достроим его до $e_1, \dots, e_k, e_{k+1}, \dots, e_n$ - базиса $V$. Покажем, что $e^*_{k + 1}, \dots, e^*_n$ - базис $AnnW$. Линейная независимость вытекает из того, что изначально эти векторы были линейно независимыми в базисе $V^*$, а мы просто взяли подмножество. \\
	Порождаемость: пусть $\varphi \in AnnW \implies \varphi = \sum\limits_{i = 1}^k\beta_ie_i^* + \sum\limits_{i = k + 1}^ n \beta_ie_i^*$. Но так как $e_1, \dots, e_k$ - базис $W$, на нем $\varphi$ будет равна нулю, следовательно $\varphi = \sum\limits_{i = k + 1}^n \beta_ie_i^*$, тогда это базис, чего мы и хотели.
	\section{Сопряженное отображение. Матрица сопряженного отображения.}
	\subsection{Сопряженный оператор.} 
	\textbf{Определение:} пусть $\mathcal{L}: V \to U$ - линейное отображение, тогда определим $\mathcal{L}^*: U^* \to V^*$ следующим образом: пусть $\varphi \in U^*, \mathcal{L}^*\left(\varphi\right) =\varphi\circ\mathcal{L} \in V^*$ - сопряженное или двойственноеу к $\mathcal{L}$
	\subsection{Матрица сопряженного оператора.} 
	$\left[\mathcal{L}^*\right]_{f^*, e^*} = \left[\mathcal{L}\right]^T_{e, f}$, если $A = \left(a_{ij}\right)$ - матрица размера $m\times n$, тогда $A^T = \left(a_{ji}\right)$ - матрица размера $n \times m$, то есть мы берем просто матрицу, переворачиваем ее и отзеркаливаем.
	\section{Аннулятор образа равен ядру сопряженного.}
	\textbf{Утверждение:} $Ann\left(Im\mathcal{L}\right) = \ker \mathcal{L}^*, \mathcal{L}: V \to U$ \\
	\textbf{Упражнение:} $W \subset V, AnnW \subset V^*:$ \\
	Пусть $\varphi \in V^*: \varphi\left(v\right) = 0 \ \forall v \in V$, среди этих элементов также есть те, которые лежат в $W$, тогда получается, что $\varphi \in AnnW$. \\ Теперь пусть $\varphi_1, \varphi_2 \in AnnW \\$ 
	$\left(\varphi_1 + \varphi_2\right)\left(w\right) = \varphi_1\left(w\right) + \varphi_2\left(w\right) = 0$ \\
	$\left(\gamma\varphi\right)\left(w\right) = \gamma\varphi\left(w\right) = 0$ \\
	\textbf{Доказательство:} $\varphi \in Ann\left(Im\mathcal{L}\right) \iff \varphi\left(u\right) = 0 \ \forall u \in Im\mathcal{L} \iff \varphi\left(\mathcal{L}\left(v\right)\right) = 0 \ \forall v \in V \iff \mathcal{L}^*\left(\varphi\right)\left(v\right) = 0 \ \forall v \in V \iff \mathcal{L}^*\left(\varphi\right) = 0 \iff \varphi \in \ker\mathcal{L}^*$
	\section{Ранг оператора. Равенство строчного и столбцового рангов оператора. Теорема Кронекера–Капелли.}
	\subsection{Ранг оператора.}
	\textbf{Определение (через матрицы):} пусть $\mathcal{L}: \underset{e}{V}\to \underset{f}{W}$. Тогда строчный ранг $\left[\mathcal{L}\right]_{e, f}$ - максимальное количество линейно независимых строк этой матрицы. Столбцовый ранг $\left[\mathcal{L}\right]_{e, f}$ - максимальное количество линейно независимых столбцов этой матрицы. \\
	\textbf{Определение:} $\mathcal{L}: \underset{e}{V}\to \underset{f}{W}$. Строчный ранг - $\dim Im\mathcal{L}^*$, столбцовый ранг - $\dim Im\mathcal{L}$
	\subsection{Равенство строчного и столбцового рангов оператора.} 
	\textbf{Теорема:} $\dim Im \mathcal{L} = \dim Im \mathcal{L}^*$ - столбцовый ранг равен строчному рангу \\
	\textbf{Доказательство:} Пусть $m = \dim U^*$. $\dim Im\mathcal{L}^* = \dim U^* - \dim \ker \mathcal{L}^* = m - \dim Ann\left(Im\mathcal{L}\right) = m - \left(m - \dim Im\mathcal{L}\right) = \dim Im\mathcal{L}$ \\
	Тогда $rank\mathcal{L}$ - любой из этих рангов
	\subsection{Теорема Кронекера-Капелли.}
	\textbf{Теорема:} решение $\mathcal{L}x = b$ существует $\iff rank\left[\mathcal{L}\right] = rank\left(\left[\mathcal{L}\right]\right |b)$ \\
	\textbf{Доказательство:} пусть $\mathcal{L}: V \to U, b\in U.$ Хотим найти $x \in V:\mathcal{L}x = b$ \\
	Решение существует если $b \in Im \mathcal{L} \iff b \in \left\langle\mathcal{L}e_1, \dots, \mathcal{L}e_n\right\rangle$ где эти вектора базисные в $V$ $\iff$ Максимальное количество линейно независимых в $\left\{\mathcal{L}e_1, \dots, \mathcal{L}e_n\right\}$ равно максимальному количеству линейно независимых элементов в $\left\{\mathcal{L}e_1, \dots, \mathcal{L}e_n, b\right\} \iff rank\left[\mathcal{L}\right] = rank\left(\left[\mathcal{L}\right] | b\right)$
	\section{Тензорное произведение, существование и единственность. Тензорная алгебра.}
	\textbf{Определение:} \\
	отображение $\varphi: V \times W \to U $ называют билинейным
	когда оно линейно по обоим аргументам по отдельности, то есть:
	пусть для каждого $v \in V$ определено $f_v: W \to U$
	такое что $\forall w \in W: f_v(w)=\varphi(v,w)$
	и для каждого $ w\in W$ определено $g_w: V \to U$
	такое что $\forall v\in V:g_w(v)=\varphi(v,w)$ 
	тогда если $ \forall v\in V :f_v\in Hom(W,U)$ и $ \forall w \in W: g_w \in Hom(V,U)$ то
	$ \varphi$ билинейное отображение\\
	\textbf{Определение:} \\
	тензорным произведением пространств $V$ и $W$ называют такую пару $(T,\tau)$
	где $T$ это векторное пространство, $\tau$ это билинейное отображение из $V,W$ в $T$
	что для каждого билинейного отображения $\varphi$ из $V,W$ в любое пространство $U$
	выполняется: \\
	$ \exists! \widetilde{\varphi}\in Hom(T,U):\forall (v,w)\in V \times W:\widetilde{\varphi}(\tau(v,w))=\varphi(v,w)$
	то есть $ \exists!\widetilde{\varphi}:\widetilde{\varphi}\circ\tau=\varphi$
	такое пространство $T$ с естественным $ \tau$ обозначается $ V\otimes W$ \\
	\textbf{Доказательство:} \\
	докажем, что такое пространство единственно с точностью до изоморфизма, 
	пусть таким свойством относительно $V,W$ обладают два пространства $(T,\tau)$ и $(\widetilde{T},\widetilde{\tau})$, докажем что $T\cong\widetilde{T}$.
	По определению, для $ \widetilde{\tau}:V\times W\to \widetilde{T}$
	существует единственное $ \psi: T\to \widetilde{T}$ такое что $ \psi\circ\tau=\widetilde{\tau}$
	и в другую сторону для $ \tau:V\times W\to \widetilde T$
	существует единственное $ \widetilde\psi:\widetilde T \to T$ такое что $ \widetilde\psi\circ\widetilde\tau=\tau$
	то есть $ \psi\circ\tau=\widetilde\tau$
	$ \widetilde\psi\circ\widetilde\tau=\tau$ Значит $\widetilde\psi\circ\psi\circ\tau=\widetilde\psi\circ\widetilde\tau=\tau$,
	заметим, что для $ \widetilde\psi\circ\psi$ подходит $ id_T$,
	но так как мы определяли $ \psi,\widetilde\psi$ как единственные подходящие,
	то $ \widetilde\psi\circ\psi$ и будет только $\mathrm{id}_T$ \\
	аналогично будет $ \psi\circ\widetilde{\psi}=id_{\widetilde{T}}$, а так как $ \psi,\widetilde\psi$ друг другу обратны слева и справа, значит это взаимнообратные биекции, а так как это были линейные отображения между $T$ и $\widetilde T$, то мы и нашли изоморфизм \\
	существование такого $ V\otimes W$ докажем конструктивно \\
	пусть $ e_1,e_2,...,e_n$ это базис $ V$
	пусть $ f_1,f_2,...,f_m$ это базис $ W$
	тогда составим для всех $ i=1,2,...,n$ и $ j=1,2,...,m$
	пары из элементов базисов, и обозначим их новыми буквами
	$ \tau(e_i,f_j)=e_i\otimes f_j$, распишем значения в общем виде \\
	пусть $ v=\sum\limits_{i=1}^n\alpha_i e_i$ и $ w=\sum\limits_{j=1}^m \beta_j f_j$,
	тогда $ \tau(v,w)=\tau\left( \sum\limits_{i=1}^n\alpha_i e_i , \sum\limits_{j=1}^m \beta_j f_j \right)=\sum\limits_{i=1}^n \sum\limits_{j=1}^m \alpha_i \beta_j (e_i\otimes f_j)$ \\
	заметим, что теперь можем по любому билинейному $ \varphi:V\times W\to U$
	построить линейное $ \widetilde\varphi:V\otimes W\to U$, определив его таким образом:
	$ \widetilde\varphi(e_i\otimes f_j)=\varphi(e_i,f_j)$ и, так как имеем совпадение результатов на базисе и однозначное определение $ \widetilde\varphi$ от $ \varphi$, то получаем, что наше $ V\otimes W$ соответствует определению. \\
	заметим, что как векторное пространство,
	$ V\otimes W\cong k^{n\cdot m}$ где $k$ это поле, над которым определены $V,W$, определим тензорное произведение последовательности пространств $ V_1,V_2,...,V_n$ где $ n>2$ по цепному правилу:
	$$ \bigotimes\limits_{i=1}^n V_i=V_1\otimes \left(\bigotimes\limits_{i=1}^{n-1} V_{i+1}\right)$$
	так, определим тензорную степень пространства:
	$ V^{\otimes n}=\bigotimes\limits_{i=1}^n V$ если $ n>0$
	$ V^{\otimes 0}=k$ где $ k$ это поле, над которым задано $ V$, если $ n=0$\\
	определим прямую (внешнюю) сумму последовательности пространств
	пронумерованной целыми неотрицательными числами
	как конечные (финитные) последовательности элементов пространств
	из последовательности, то есть последовательности, где
	первый элемент принадлежит первому пространству,
	второй второму,
	и так далее, но с некоторого момента, элементы становятся
	стабильно нулями
	а именно
	$$
	\bigoplus\limits_{n=0}^\infty V_n
	=
	\left\{
	s\in \times_{n=0}^\infty V_n :
	\exists n\in \mathbb{Z}_{+}\ \forall m\in \mathbb{Z}_+\ (m>n)\Rightarrow (s_m=0)
	\right\}
	$$
	тогда тензорной алгеброй называют такое пространство
	$$ TV=\bigoplus\limits_{n=0}^\infty V^{\otimes n}$$
	в котором определено не коммутативное, дистрибутивное относительно сложения умножение,
	для $v,w\in TV$ обозначается $ v\otimes w$, пусть $ e_1,e_2,...,e_n$ это базис $ V$
	тогда зададим базис в $ TV$ , это будет
	$$ \bigcup\limits_{k=0}^\infty \left\{ \bigoplus\limits_{j=1}^k e_{i_j}:i \in \left(\mathbb{Z}\cap[1;n]\right)^k\right\}$$
	где как и в тензорном произведении двух пространств, такие элементы
	будут считаться отдельными буквами
	то есть элементы базиса это конечные последовательности индексов
	элементов базиса изначального $ V$ , например: \\
	при $ i=()$ получим произведение пустой последовательности векторов \\
	при $ i=(2)$ получим $ e_2$ \\
	при $ i=(1,2,3)$ получим $ e_1\otimes e_2\otimes e_3$ \\
	а произведение двух элементов $ v,w\in TV$ в общем случае будет таким:
	сначала разложим элементы по введённому базису, а после этого переберём все элементы $ \alpha\cdot e_{i_1}\otimes e_{i_2}\otimes \dots \otimes e_{i_k}$в разложении $v$ и элементы $ \beta\cdot e_{j_1}\otimes e_{j_2}\otimes \dots \otimes e_{j_l}$, тогда к произведению прибавится $ \alpha\cdot \beta \cdot e_{i_1}\otimes e_{i_2}\otimes \dots \otimes e_{i_k}\otimes e_{j_1}\otimes e_{j_2}\otimes \dots \otimes e_{j_l}$, то есть $ v,w$ раскладываются по базису,
	перебираются все пары произведений элементов базиса $ V$,
	находится коэффициент при одном в $ v$ и другом в $ w$ соответственно, и результат это сумма по всем парам элементов базиса $ TV$
	произведения коэффициентов при них в $ v$ и $ w$ соответственно, на их конкатенацию (склеивание строк)
	например: \\
	$ (e_1+e_1\otimes e_2)\otimes (3\cdot e_2\otimes e_1+2\cdot e_1\otimes e_1)$ раскладываем
	по дистрибутивности, получаем: $ e_1\otimes (3\cdot e_2\otimes e_1)+e_1\otimes (2\cdot e_1\otimes e_1)+e_1\otimes e_2\otimes(3\cdot e_2\otimes e_1)+e_1\otimes e_2\otimes(2\cdot e_1\otimes e_1)$
	упростим выражение, приведём подобные слагаемые, вынесем константы:
	$$ 3\cdot e_1\otimes e_2\otimes e_1+2\cdot e_1\otimes e_1\otimes e_1+3\cdot e_1\otimes e_2\otimes e_2\otimes e_1+2\cdot e_1\otimes e_2\otimes e_1\otimes e_1$$
	\section{Внешняя степень, построение. Внешняя алгебра.}
	в алгебре $TV$ определим такой двусторонний идеал:
	$$I=\left\langle v\otimes v :~v\in V \right\rangle$$
	то есть идеал, натянутый на произведение
	двух одинаковых векторов подряд.
	если записать как множество, то получится
	$$I=\left\{ w_1\otimes v\otimes v\otimes w_2 : v\in V,\; w_1,w_2\in TV\right\}.$$
	тогда внешней алгеброй над $V$ назовём $TV/I$,
	обозначается $\wedge V$.
	в такой алгебре умножение обозначается $\wedge$,
	и станет антикоммутативным, то есть
	$$\forall a,b\in \wedge V:\; a\wedge b=-(b\wedge a).$$
	так как
	$$(a+b)\wedge(a+b)=a\wedge a+a\wedge b+b\wedge a+b\wedge b
	=0+a\wedge b+b\wedge a+0,$$
	а значит
	$$a\wedge b+b\wedge a=0.$$
	$k$-й внешней степенью $V$ будем называть пространство $V^{\otimes k}$
	спроецированное на $\wedge V$,
	обозначается $\wedge^k V$,
	то есть пространство произведений последовательностей из $k$ векторов из $V$.
	\section{Базис внешней степени. Подсчет размерности внешней степени.}
	мы уже знаем, что
	$$
	V^{\otimes k}=\left\{ \bigoplus\limits_{j=1}^k e_{i_j}:~i\in \left(\mathbb{Z}\cap[1;n]\right)^k \right\},
	$$
	где $e_1,e_2,...,e_n$ это базис $V$.
	
	тогда если факторизовать $V^{\otimes k}$ по $I$
	то уже знаем, что получатся все те же произведения,
	но без порядка множителей, ведь из одного порядка
	можно перейти в другой, если домножить на $-1$
	и переставить два соседних элемента по антикоммутативности
	внешнего произведения.
	
	тогда чтобы построить базис,
	надо выбрать по представителю из всех классов эквивалентности последовательностей
	из $k$ элементов из $n$ элементов базиса $V$, где нет повторяющихся,
	и где две последовательности считаются эквивалентными, если
	они совпадут при перестановке элементов.
	количество таких элементов это по определению $C_n^k$.
	
	тогда чтобы построить пример базиса,
	например упорядочим все последовательности определённым образом:
	например индексы элементов $e_i$ будут неубывающими,
	но так как мы выкинули все последовательности с повторяющимися
	элементами, то $i$ станут строго возрастать.
	тогда базисом можно выбрать такое множество:
	$$
	\left\{ \bigwedge\limits_{j=1}^k e_{i_j} :
	i\in (\mathbb{Z}\cap[1;n])^k,\ 
	\forall j\in \mathbb{Z}\cap[1;k-1]: i_j<i_{j+1}
	\right\}.
	$$
	\section{ Индуцированные гомоморфизмы на тензорной и внешней алгебрах. Определитель. Мультипликативность определителя.}
	пусть дано линейное отображение $\mathcal{L}:~V\mapsto W$,
	тогда определим $k$-ю тензорную степень отображения $\mathcal{L}$ как:
	
	$$
	\mathcal{L}^{\otimes k}:~V^{\otimes k}\mapsto W^{\otimes k}
	$$
	
	такое что
	
	$$
	\forall i \in (\mathbb{Z}\cap[1;n])^k:\ 
	\mathcal{L}^{\otimes k}\left(\bigotimes\limits_{j=1}^k e_{i_j}\right)
	=
	\bigotimes\limits_{j=1}^k \mathcal{L}(e_{i_j}).
	$$
	
	на остальные вектора из $V^{\otimes k}$ значения продолжатся по линейности.
	
	аналогично $k$-я внешняя степень отображения $\mathcal{L}$ это:
	
	$$
	\mathcal{L}^{\wedge k}:~V^{\wedge k}\mapsto W^{\wedge k}
	$$
	
	такое что
	
	$$
	\forall i\in (\mathbb{Z}\cap[1;n])^k:\ 
	\mathcal{L}^{\wedge k}\left( \bigwedge\limits_{j=1}^k e_{i_j} \right)
	=
	\bigwedge\limits_{j=1}^k \mathcal{L}(e_{i_j}).
	$$
	
	на остальные вектора из $V^{\wedge k}$ значения продолжатся по линейности.
	
	пусть дан оператор $\mathcal{L}:~V\mapsto V$, где $V$ это векторное пространство размерности $n$.
	тогда определителем оператора $\mathcal{L}$ называют $\mathcal{L}^{\wedge n}$.
	
	заметим, что это константа, так как $\dim V^{\wedge n}=C_n^n=1$,
	то есть определитель принимает одномерный вектор вида
	$\alpha\bigwedge\limits_{i=1}^n e_i$
	и возвращает
	$\beta\alpha\bigwedge\limits_{i=1}^n e_i$,
	то есть по сути эта константа $\beta$ и определяет определитель,
	обозначается $\det \mathcal{L}$.
	
	докажем мультипликативность определителя, а именно:
	пусть дано пространство $V$, тогда
	$$
	\forall \mathcal{L},\mathcal{T}\in \operatorname{End}(V):\ 
	\det(\mathcal{L}\circ\mathcal{T})=\det(\mathcal{L})\cdot \det(\mathcal{T}).
	$$
	
	для этого просто раскроем внешнюю степень,
	то есть докажем, что
	$$
	(\mathcal{L}\circ\mathcal{T})^{\wedge n}
	=
	\mathcal{L}^{\wedge n}\circ\mathcal{T}^{\wedge n}.
	$$
	
	раскроем внешние произведения из определений,
	и покажем, что это одна и та же запись:
	
	$$
	\mathcal{T}^{\wedge n}\left(\bigwedge\limits_{i=1}^n e_i\right)
	=
	\bigwedge\limits_{i=1}^n \mathcal{T}(e_i),
	$$
	
	$$
	\mathcal{L}^{\wedge n}\left(\bigwedge\limits_{i=1}^n \mathcal{T}(e_i)\right)
	=
	\bigwedge\limits_{i=1}^n \mathcal{L}(\mathcal{T}(e_i))
	=
	(\mathcal{L}\circ \mathcal{T})^{\wedge n}\left( \bigwedge\limits_{i=1}^n e_i \right).
	$$
	\section{Изоморфизм \texorpdfstring{$V \overset{\sim}{=}\left(\wedge^{n-1}V\right)^*$}{V = (wedge^n-1V)}}
	изоморфность доказывается тривиально:
	пусть $\dim V=n$, тогда
	$\dim\wedge^{n-1}V=\dim\left(\wedge^{n-1}V\right)^*=C_n^{n-1}=n$.
	
	но важен определённый изоморфизм:
	
	пусть $V$ определено над полем $k$, тогда
	формой объёма называют $\omega:~V^{\wedge n}\mapsto k$,
	это $\omega\in (V^{\wedge n})^*$ такое что:
	
	$$
	\omega\left( \bigwedge\limits_{i=1}^n e_i\right)=1.
	$$
	
	заметим, что такое $\omega$ просто равно константе при
	произведении последовательности всех векторов из базиса.
	
	определим $F:V\mapsto (V^{\wedge n-1})^*$ такое что
	$$
	\forall v\in V:\ \forall u \in V^{\wedge n-1}:\ 
	F_v(u)=\omega(v\wedge u).
	$$
	
	заметим, что $F$ линейно, так как выполняется
	дистрибутивность при раскрытии $\wedge$.
	
	и $F$ изоморфизм, так как биекция, потому что инъективный
	гомоморфизм в конечномерном пространстве.
	\section{ Критерий обратимости оператора в терминах определителя. Построение обратного оператора с помощью изоморфизма \texorpdfstring{$V \overset{\sim}{=}\left(\wedge^{n-1}V\right)^*$}{V = (wedge^n-1V)}.}
	определим $\mathcal{L}^{\vee}:~V\mapsto V$ единственным образом
	таким способом:
	
	$$
	\forall v \in V:\ \forall u \in V^{\wedge n-1}:\ 
	\omega(v\wedge\mathcal{L}^{\wedge n-1}(u))
	=
	\omega(\mathcal{L}^{\vee}(v)\wedge u).
	$$
	
	докажем, что $(\det \mathcal{L}\ne 0) \Leftrightarrow (\exists \mathcal{L}^{-1})$.
	
	справа налево:
	
	с одной стороны
	$$
	\det (\mathcal{L}\circ\mathcal{L}^{-1})=\det(\operatorname{id}_V)=1,
	$$
	с другой стороны
	$$
	\det (\mathcal{L}\circ\mathcal{L}^{-1})
	=
	\det(\mathcal{L})\cdot\det(\mathcal{L}^{-1}).
	$$
	тогда получается, что
	$$
	1=\det(\mathcal{L})\cdot \det(\mathcal{L}^{-1}),
	$$
	так что $\det \mathcal{L}$ не могло быть нулём.
	
	слева направо:
	
	построим  
	$$
	\mathcal{L}^{-1}=\frac{1}{\det\mathcal{L}}\cdot \mathcal{L}^{\vee}.
	$$
	
	конструктивно показали, что $\exists \mathcal{L}^{-1}$.
	
	\section{ Модуль над кольцом; определение конечнопорожденного и свободного модуля. Построение копредставления для конечно-порожденного модуля.}
	\section{Нормальная форма Смита для матриц над областью главных идеалов.}
	\section{Теорема о структуре конечно-порожденного модуля над ОГИ.}
	\section{Структура конечно-порожденных абелевых групп.}
	\section{Теорема Гамильтона–Кэли.}
	\section{Существование и единственность жордановой нормальной формы	над алгебраически замкнутым полем.}
	\newpage
	\section{Определения}
	\begin{description}
		\item [Соответсвие] Между множетсвами $A$ и $B$, подмножество $R \subset A \times B$. 
		\item [Отображение множеств] Соответвие $f$ из $A$ в $B$, в котором выполняется $\forall a \in A : \exists! b \in B : (a, b) \in f$ (элемент из $A$ соответсвует ровно одному элементу из $B$).
		\item [Инъекция] Отображение $f$ из $A$ в $B$, для которого выполняется $\forall a, a' \in A : f(a) = f(a') \implies a = a'$ (все элементы из $A$ отображаются в различные значения).
		\item [Сюръекция] Отображение $f$ из $A$ в $B$, для которого выполняется $\forall b \in B : \exists a \in A : f(a) = b$ (для каждого значения в $B$ есть соответствующее значение в $A$).
		\item [Обратимость отображения слева и справа] Отношение $g$ является обратным слева к $f$, если $g \circ f = \text{id}_A$. Справа -- если $f \circ g = \text{id}_B$.
		\item [Бинарная операция] На множестве $X$, отображение $X \times X \rightarrow X$.
		\item [Моноид] Множество $M$ с бинарной операцией $\cdot$, для которых выполняется:
		\begin{description}
			\item [Ассоциативность] $\forall a, b, c \in M : a \cdot (b \cdot c) = (a \cdot b) \cdot c$.
			\item [Существование нейтрального элемента] $\exists e \in M : \forall m \in M : e \cdot m = m \cdot e = e$.
		\end{description}
		\item [Группа] Моноид, в котором дополнительно существует обратный элемент: $\forall m \in M \setminus \{e\} : \exists m^{-1} \in M : m \cdot m^{-1} = e$.
		\item [Абелева группа] Группа, в которой операция коммутативна ($a \cdot b = b \cdot a$).
		\item [Кольцо] Множество $R$ с операциями $(\cdot, +)$, такое что:
		\begin{itemize}
			\item $(R, +)$ -- комутативная (абелева) группа
			\item Выполняется дистрибутивность умножения:
			$$\forall a, b, c \in R : \begin{cases}
				a \cdot (b + c) = a \cdot b + a \cdot c \\
				(a + b) \cdot c = a \cdot c + b \cdot c
			\end{cases}$$
		\end{itemize}
		\item [Подкольцо] Подмножество $S$ кольца $R$, в котором результаты операций тоже лежат в подмножестве.
		$$\begin{cases}
			S \subset R \\
			\forall s_1, s_2 \in S : s_1 + s_2 \in S \\
			\forall s_1, s_2 \in S : s_1 \cdot s_2 \in S
		\end{cases}$$
		\item [Идеал] Непустое подмножество кольца, для которого выполняется:
		\begin{itemize}
			\item $\forall a, b \in I : a + b \in I$
			\item $\forall a \in I : \forall r \in R : r \cdot a \in I$
		\end{itemize}
		\item [Делитель нуля] Ненулевой элемент кольца $r \in R \setminus \{0\}$, такой что $\exists s \in R : s \ne 0 \land r \cdot s = 0$.
		\item [Область целостности] Кольцо, в котором нет делителей нуля.
		\item [Ассоциированные элементы] Элементы кольца $a \ne 0$, $b \ne 0$, такие что $a\ \vert\ b$ и $b\ \vert\ a$.
		\item [Евклидова область] Область целостности в котором сущесвует функция нормы $N : R \setminus \{0\} \rightarrow \mathbb{N}_0$, такая что:
		\begin{itemize}
			\item $N(a \cdot b) \ge N(a)$ (и $N(a \cdot b) = N(a)$ если $b \in R^{\times}$)
			\item Можно делить с остатком, уменьшая норму: $\forall a, b \ne 0 \in R : \exists q, r \in R : a = q \cdot b + r \land (N(r) < N(b) \lor r = 0)$.
		\end{itemize}
		\item [Наибольший общий делитель] Для двух элементов кольца $a, b \in R$, максимальный по норме элемент $g \in R$, такой что $g\ \vert\ a$, $g\ \vert\ b$.
		\item [Область главных идеалов] Область целостности, в которой каждый идеал главный (т.е. имеет вид $(i) = \{r \cdot i : r \in R\}$).
		\item [Неприводимый элемент] Элемент $r \in R$, для которого из равенства $r = s \cdot t$ следует, что $s \in R^{\times}$ или $t \in R^{\times}$ (нельзя разбить на необратимые элементы).
		\item [Фактор-кольцо] Кольцо $R/I$, в котором $a = b \iff (a - b) \in I$.
		\item [Гомоморфизм колец] Отображение $\varphi : R \rightarrow S$, которое сохраняет структуру кольца:
		\begin{itemize}
			\item $\varphi(a) + \varphi(b) = \varphi(a + b)$
			\item $\varphi(a) \cdot \varphi(b) = \varphi(a \cdot b)$
			\item $\varphi(1_R) = 1_S$ для колец с единицей
		\end{itemize}
		\item [Ядро гомоморфизма] Все элементы, которые отправляются в ноль: \\
		$\text{Ker}\ \varphi = \{r : r \in R, \varphi(r) = 0 \}$.
		\item [Образ гомоморфизма] Множество значений гомоморфизма: \\
		$\text{Im}\ \varphi = \{\varphi(r) : r \in R \} = \{s \in S : \exists r \in R, \varphi(r) = s \}$.
		\item [Произведение колец] Кольцо пар из элементов других колец с покомпонентными операциями:
		\begin{itemize}
			\item $R \times S = \{(r, s) : r \in R, s \in S \}$
			\item $(r_1, s_1) + (r_2, s_2) = (r_1 + r_2, s_1 + s_2)$
			\item $(r_1, s_1) \cdot (r_2, s_2) = (r_1 \cdot r_2, s_1 \cdot s_2)$
		\end{itemize}
		\item [Функция эйлера] Функция $\varphi : \mathbb{N} -> \mathbb{N}$, и $\varphi(n)$ равно количеству натуральных чисел $\le n$, которые взаимно просты с $n$. \\
		$\varphi(n) = |\{x : 1 \ge x \le n \land \text{gcd}(x, n) = 1\}|$.
		\item [Кольцо многочленов над полем] Кольцо $K[x]$, которое содержит последовательности коэффицентов $(k_0, k_1, \dots, k_n, 0, \dots)$, каждая из которых соответствует выражению $k_0 + k_1 x + k_2 x^2 + \cdots + k_n x^n$.
		\item [Корень многочлена] Элемент поля $x \in K[x]$ когда $P(x) = 0$ (очев?).
		\item [Формальное равенсто многочленов] Многочлен $P(x) = p_0 + p_1 x + p_2 x^2 + \dots$ формально равен $Q(x) = q_0 + q_1 x + q_2 x^2 + \dots$ если $p_0 = q_0, p_1 = q_1, \dots$ ($\forall i : p_i = q_i$).
		\item [Фукнциональное равенство многочленов] Многочлен $P \in K[x]$ функционально равен $Q \in K[x]$ если $\forall x \in K : P(x) = Q(x)$. 
		\item [Сумма, произведение, пересечение идеалов] \
		\begin{description}
			\item [Cумма] $I + J = \{i + j : i \in I, j \in J\}$
			\item [Произведение] $IJ = \{\sum_k i_k j_k : i_k \in I, j_k \in J\}$
			\item [Пересечение] $I \cap J = \{x : x \in I \land x \in J\}$
		\end{description}
		\item [Взаимно простые (комаксимальные) идеалы] Идеалы $I$ и $J$ кольца $R$, такие что $I + J = R \iff 1 \in I + J$.
		\item [Формула Тейлора] \textbf{Тут нужна просто формула $m$-ой производной?}
		\item [Кратность корня многочлена] Корень $\alpha$ многочлена $P$ имеет кратность $m$ если $P = (x - \alpha)^m \cdot G$, где $G \ne 0$.
		\item [Итерполяционный многочлен Лагранжа] Способ построить многочлен, проходящий через множество точек $\{(x_0, y_0), \dots\, (x_n, y_n)\}$.
		\begin{align*}
			f(x)   &= \sum_{i=0}^n y_i L_i(x) \\
			L_i(x) &= \frac{(x - x_0) \cdots (x - x_{i-1})(x - x_{i+1}) \cdots (x - x_n)}{(x_i - x_0) \cdots (x_i - x_{i-1})(x_i - x_{i+1}) \cdots (x_i - x_n)} = \frac{\prod_{j \ne i} x - x_j}{\prod_{j \ne i} x_n - x_j}
		\end{align*}
		\item [Максимальный идеал] Идеал $I \triangleleft R$, $I \ne R$, когда для некоторого $J$: $I \subset J \implies (J = I) \lor (J = R)$.
		\item [Простой идеал] Идеал $I \triangleleft R$, $I \ne R$, в котором $ab \in I \implies (a \in I) \lor (b \in I)$.
		\item [Квадратичный вычет] Число $x \in \mathbb{Z}_p$, такое что $\exists y \in \mathbb{Z}_p : y^2 = x$ ($x$ является квадратом некоторого числа по модулю $p$).
		\item [Символ Лежандра] Фукнция $\left(\frac{a}{p}\right)$, которая показывает, является ли $a$ квадратичным вычетом по модулю $p$. \\
		$$
		\left(\frac{a}{p}\right) = \begin{cases}
			1,  & a\ \text{-- кв. вычет mod $p$} \\
			-1, & \text{иначе} \\
		\end{cases}
		$$
		\item [Поле комплексных чисел] Поле $\mathbb{C} = \mathbb{R}[i] / (i^2 + 1)$ -- дополнение действительных чисел, в котором $i^2 = -1$.
		\item [Комплексное сопряжение] Для числа $x = a + bi \in \mathbb{C}$: $\bar{x} = a - bi$.
		\item [Модуль комплекного числа] Для $x = a + bi$, $|x| = \sqrt{a^2 + b^2} = \sqrt{x \bar{x}}$, расстояние от $0$ на комплекной плоскости.
		\item [Векторное пространство] Множество $V$ поверх $K$ с операциями $+ : V \times V \rightarrow V$ и $\cdot : K \times V \rightarrow V$, где дополнительно:
		\begin{itemize}
			\item $(V, +)$ -- абелева группа
			\item $(\alpha + \beta)v = \alpha v + \beta v$
			\item $\alpha (v_1 + v_2) = \alpha v_1 + \alpha v_2$
			\item $(\alpha \beta) v = \alpha (\beta v)$
			\item $1 \cdot v = v$
		\end{itemize}
		\item [Линейная независимость] Векторы $v_1$ и $v_2$ линейно независимы, если не существует таких $a$ и $b$, что $a v_1 + b v_2 = \vec{0}$.
		\item [Базис векторного пространства] Три эквивалентных определения:
		\begin{itemize}
			\item Линейно независимое порождающее множество
			\item Максимальное по включению множество независимых векторов
			\item Минимальное по включение порождающее множество
		\end{itemize}
		\item [Размерность векторного пространства] $\dim V := |B|$, где $B$ -- базис пространства $V$.
		\item [Линейное отображение] Фукнция (гомоморфизм) $\mathcal{L} : V \rightarrow W$, такая что
		\begin{itemize}
			\item $\mathcal{L}(v_1 + v_2) = \mathcal{L}(v_1) + \mathcal{L}(v_2)$
			\item $\mathcal{L}(\alpha \cdot v) = \alpha \mathcal{L}(v)$
		\end{itemize}
		\item [Ядро линейного отображения] Вектора, которые отправляются в $\vec{0}$: \\
		$\text{Ker}\ \mathcal{L} = \{v : v \in V, \mathcal{L}(v) = \vec{0}\}$
		\item [Образ линейного отображения] Все достижимые вектора после отображения: \\
		$\text{Im}\ \mathcal{L} = \{u : \exists v \in V, \mathcal{L}(v) = u \} = \{\mathcal{L}(v) : v \in V\}$
		\item [Матрица линейного отображения] Если $e = (e_1, e_2, \dots, e_n)$ -- базис $V$, $f = (f_1, \dots, f_m)$ -- базис $W$, то матрица линейного отображения $\mathcal{L}$ -- таблица вида:
		$$
		\begin{pmatrix}
			a_{1,1} & a_{1,2} & \cdots & a_{1,n} \\
			a_{2,1} & a_{2,2} & \cdots & a_{2,n} \\
			\vdots  & \vdots  & \ddots & \vdots  \\
			a_{m,1} & a_{m,2} & \cdots & a_{m,n}
		\end{pmatrix}
		$$
		Преобразование задаётся как:
		\begin{align*}
			\mathcal{L} e_1 &= a_{1,1} f_1 + \dots + a_{m,1} f_m \\
			\mathcal{L} e_2 &= a_{1,2} f_1 + \dots + a_{m,2} f_m \\
			\vdots \\
			\mathcal{L} e_n &= a_{1,n} f_1 + \dots + a_{m,n} f_m
		\end{align*}
		\item [Двойственное пространство] Пространство линейных функционалов -- функций, преобразующих векторы в скаляры при этом сохраняя линейность:
		\begin{itemize}
			\item $\forall \varphi \in V^*, v, w \in V : \varphi(v) + \varphi(w) = \varphi(v + w)$
			\item $\forall \varphi \in V^*, v \in V, k \in K : k \varphi(v) = \varphi(kv)$
		\end{itemize}
		Можно их считать гомоморфизмами из $V$ в $K$.
		\item [Двойственный базис] Координатные функции для базиса $V$ ($e = (e_1, e_2, \dots, e_n)$):
		$$
		e_i^*(e_j) = \delta_i^j =
		\begin{cases}
			1, i = j \\
			0, i \ne j
		\end{cases}
		$$
		\item [Аннулятор множества] Множество линейных функционалов, которые обнуляют все векторы в подпространстве $W \subset V$:
		$\text{Ann}\ W = \{\varphi \in V^* : \forall \omega \in W\ \varphi(\omega) = 0 \}$
		\item [Сопряжённый оператор] Оператор над двойственными векторами, который соответствует применению линейного оператора к пространству перед применением линейного функционала: \\
		$\mathcal{L}^*(\varphi)(v) = \varphi(\mathcal{L}(v)) \implies \mathcal{L}^*(\varphi) = \varphi \circ \mathcal{L}$
		
		Переход: $V \xrightarrow{\mathcal{L}} U \xrightarrow{\varphi} K$.
		\item [Ранг оператора] Равные понятия ранга:
		\begin{description}
			\item [Строчный] $\text{rk}_r(\mathcal{L}) = \dim (\text{Im}\ \mathcal{L}^*)$
			\item [Столбцовый] $\text{rk}_l(\mathcal{L}) = \dim (\text{Im}\ \mathcal{L})$
		\end{description}
		\item [Тензорное произведение]
		\item [Тензорная алгебра]
		\item [Внешнаяя степень]
		\item [Внешнаяя алгебра]
		\item [Определитель]
		\item [Модуль] Над кольцом $R$, абелева группа $M$ с отображением $\cdot : R \times M \rightarrow M$ и
		\begin{itemize}
			\item $\forall r_1, r_2 \in R, m \in M : (r_1 + r_2) \cdot m = r_1 \cdot m + r_2 \cdot m$
			\item $\forall r \in R, m_1, m_2 \in M : r \cdot (m_1 + m_2) = r \cdot m_1 + r \cdot m_2$
			\item $\forall r_1, r_2 \in R, m \in M : r_1 \cdot (r_2 \cdot m) = (r_1 \cdot r_2) \dot m$
			\item $\forall m \in M : 1 \cdot m = m$
		\end{itemize}
		\item [Свободный модуль] Модуль, для которого $\exists n : M \cong R^n$.
		\item [Конечнопорожденный модуль] Все элементы модуля являются линейными комбинациями конечного порождающего множества: \\
		$\exists m_1, \dots, m_n \in M : \forall x \in M : \exists r_1, \dots, r_n \in R : x = r_1 \cdot m_1 + \dots + r_n \cdot m_n$
		\item [Нормальная форма Смита] Если $R$ -- ОГИ и $A$ -- матрица, то существуют обратимые матрицы $U$ и $V$, такие что
		$$
		UAV = \begin{pmatrix}
			\alpha_1 & 0        & 0       & \dots    & 0      & \dots & 0      \\
			0        & \alpha_2 & 0       &          &        &       &        \\
			0        & 0        & \ddots  &          & \vdots &       & \vdots \\
			\vdots   &          &         & \alpha_t &        &       &        \\
			0        &          & \dots   &          & 0      & \dots & 0      \\
			\vdots   &          &         &          & \vdots &       & \vdots \\
			0        &          & \dots   &          & 0      & \dots & 0      \\
		\end{pmatrix}
		$$
		\item [Нильпотентный оператор] Оператор, если применить который $n$ раз подряд, он станет отправлять всё в $\vec{0}$.
		\item [Жорданова клетка] Компонент нормальной формы, матрица вида:
		$$
		\begin{pmatrix}
			\lambda &         &         & \dots   &        & 0       \\
			1       & \lambda &         &         &        &         \\
			0       & 1       & \lambda &         &        & \vdots  \\
			0       & 0       & 1       & \lambda &        &         \\
			\vdots  & \vdots  &         & \ddots  & \ddots & 0       \\
			0       & 0       & \dots   & 0       & 1      & \lambda \\
		\end{pmatrix}
		$$
		\item [Жорданова нормальная форма] Способ записи матрицы оператора в диагональном виде в определенном базисе:
		$$
		\begin{pmatrix}
			J(\lambda_1) &              & \dots        &        & 0            \\
			& J(\lambda_2) &              &        &              \\
			\vdots       &              & J(\lambda_3) &        & \vdots       \\
			&              &              & \ddots &              \\
			0            &              & \dots        &        & J(\lambda_m) \\
		\end{pmatrix}
		$$
		Где $J(\lambda_i)$ -- жорданова клетка.
	\end{description}
\end{document}