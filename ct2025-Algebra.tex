\documentclass[12pt]{article}
\usepackage{graphicx} % Required for inserting images
\usepackage[a4paper, margin=2.5cm]{geometry}
\usepackage[T2A]{fontenc}
\usepackage[utf8]{inputenc}
\usepackage[russian]{babel}
\usepackage{amsfonts}
\usepackage{amsmath}
\usepackage{amssymb}
\usepackage{float}
\usepackage[
colorlinks=true,
linkcolor=black,
urlcolor=black,
citecolor=black
]{hyperref}


\begin{document}
	
	\begin{titlepage}
		
		\centering
		\vspace*{5cm}
		{\bfseries\LARGE Линейная алгебра КТ 2025 M3136 \par}
		\vspace*{3cm}
		\Large created by \href{https://github.com/fyddan}{fyddan}
		\\
		\vspace{3cm}
		\href{https://www.youtube.com/playlist?list=PLd7QXkfmSY7ZbGBitHpwqW7Lv5h6QGDaj}{\includegraphics[width=0.5\textwidth]{sources/label.png}}
		
	\end{titlepage}
	
	\tableofcontents
	\newpage
	
	\section{Соответствия, композиции соответствий, ассоциативность композиции.}
	
	\subsection{Соответствия}
	\textbf{Определение:} соответствием из $A$ в $B$ называют подмножество $R \subset A \times B$, такое что $\forall \ a \in A \ \exists \ b \ \in B : (a, b) \in R$, разница с отношением в том, что тут мы требуем, чтобы каждый элемент из первого множества имел хотя бы один парный элемент из второго множества.
	\begin{figure}[h!]
		\centering
		\includegraphics[width=0.6\linewidth]{sources/relations.png}
	\end{figure}
	
	\subsection{Композиция соответствий}
	\textbf{Определение:} Пусть $R$ - соответствие из $A$ в $B$, $S$ - соответствие из $B$ в $C$, тогда композиция $S \circ R$ это такое соответствие из $A$ в $C$: $$S \circ R = \{(a, c) \in A \times C \ | \ \exists b \in B: (a, b) \in R, \ (b, c) \in S\}$$
	\begin{figure}[h!]
		\centering
		\includegraphics[width=0.6\linewidth]{sources/rel_composition.png}
	\end{figure}
	
	\subsection{Ассоциативность композиции}
	Композиция обладает свойством ассоциативности, докажем это: пусть у нас есть соответствие $R$ из $A$ в $B$, $S$ из $B$ в $C$, $T$ из $C$ в $D$, тогда сделаем предположение что $\left(T \circ S \right) \circ R$ равносильно $T \circ \left( S  \circ R \right)$, покажем что это одно и то же: 
	\\ \\ 
	$\left(T \circ S \right) \circ R = \{(a, d) \in A \times D : \exists b \in B : (a, b) \in R, (b, d) \in T\circ S\}= \{(a,d)\in A \times D:\exists b \in B, \exists c \in C: (a, b) \in R, (b, c) \in S, (c, d) \in T\}$
	\\ \\
	$T \circ \left(S  \circ R \right) = \{(a, d) \in A \times D : \exists c \in C: (a, c) \in S\circ R, (c, d) \in T\} = \{(a, d) \in A \times D : \exists b \in B,\exists c \in C:(a,b) \in R, (b, c) \in S, (c, d) \in T\}$
	\\ \\
	Заметим что правила задания множества полностью совпадают, следовательно это равные множества, что и требовалось доказать.
	
	\begin{figure}[h!]
		\centering
		\includegraphics[width=0.6\linewidth]{sources/comp_associativity.png}
	\end{figure}
	
	\section{Отображение множеств как частный случай соответствия. Инъекция и сюръекция. Обратимость отображения слева и справа. Совпадение левого и правого обратных.}
	\subsection{Отображения}
	\textbf{Определение:} назовем отображением соответствие $f \subset A\times B$ такое что: $\forall a \in A \ \exists ! \ b \in B: (a, b) \in f$
	\begin{figure}[h!]
		\centering
		\includegraphics[width=0.6\linewidth]{sources/mapping.png}
	\end{figure}
	\subsection{Инъекция}
	\textbf{Определение:} инъекция это такое отображение для которого выполняется: $$f(a) = f(b) \iff a = b$$
	\begin{figure}[h!]
		\centering
		\includegraphics[width=0.6\linewidth]{sources/injection.png}
	\end{figure}
	\subsection{Сюръекция}
	\textbf{Определение:} сюръекция это такое отображение для которого выполняется: $$ \forall b \in B \ \exists a \in A: f(a) = b$$
	\begin{figure}[h!]
		\centering
		\includegraphics[width=0.6\linewidth]{sources/surjection.png}
	\end{figure}
	
	\subsection{Биекция}
	\textbf{Определение:} биекция это отображение, которое и инъективно и биективно, то есть можно составить взаимно однозначное соответствие.
	
	\begin{figure}[H]
		\centering
		\includegraphics[width=0.6\linewidth]{sources/bijection.png}
	\end{figure}
	
	\subsection{Обратное отображение слева}
	\textbf{Определение:} Пусть у нас есть отображение $f: A \to B$, тогда отображение $g:B \to A$ называется обратным слева, если $g\circ f = idA$, то есть элемент из $A$ возвращается в себя же. Интересно что $f$ обязано быть инъективным, так как иначе при $a_1 \neq a_2, f(a_1) = f(a_2)$ случится противоречие, так как $g(f(a_1)) = g(f(a_2)) \implies a_1 = a_2$. Обратный слева также называют $idA$, так как он всегда возвращает свое же значение.
	\begin{figure}[h!]
		\centering
		\includegraphics[width=0.6\linewidth]{sources/reverseL.png}
	\end{figure}
	\subsection{Обратное отображение справа}
	\textbf{Определение:} Пусть у нас есть отображение $f: A \to B$, тогда отображение $g:B \to A$ называется обратным справа, если $f\circ g = idB$, то есть элемент из B возвращается в себя же. $f$ обязано быть сюръективным, то есть содержать прообразы для каждого $b \in B$, иначе мы не сможем для любого элемента из $B$ совершить путь через $A$ обратно в себя. Обратный справа также называют $idB$. 
	\begin{figure}[h!]
		\centering
		\includegraphics[width=0.6\linewidth]{sources/reverseR.png}
	\end{figure}
	\subsection{Совпадение обратного правого и обратного левого}
	Из определений обратного правого и левого следует, что $f: A \to B$ одновременно инъективно и сюръективно, следовательно это биекция, тогда $g:B \to A$ это просто обратное отображение, которое можно также обозначить как $f^{-1}$.
	
	\section{Определение бинарной операции, моноида, группы. Единственность нейтрального и обратного элемента. Примеры моноидов и групп.}
	\subsection{Бинарная операция}
	\textbf{Определение:} Бинарной операцией $\cdot$ на множестве $M$ называется отображение из $M \times M \to M$. Например умножение на $\mathbb{N}$ это бинарная операция, а вот на $\mathbb{N} \cup \{-1\}$ уже нет, так как при умножении числа на -1 мы выйдем за пределы исходного множества.
	\subsection{Полугруппа}
	\textbf{Определение:} Полугруппой $(M, \cdot)$ называется такая алгебраическая структура, в которой соблюдается ассоциативность, то есть: $m_1 \cdot (m_2 \cdot m_3) = (m_1 \cdot m_2) \cdot m_3$, пример - $(\mathbb{N}, \cdot)$. У нас соблюдается ассоциативность.
	\subsection{Моноид}
	\textbf{Определение:} моноидом называется полугруппа, в которую добавили нейтральный элемент $e$, для которого выполняется: $e \cdot m = m = m \cdot e$, если также добавляется коммутативность, то есть $m_1 \cdot m_2 = m_2 \cdot m_1$, тогда мы называем это коммутативным (абелевым) моноидом. Пример: $(\mathbb{N} \cup \{0\}, +)$
	
	\subsection{Единственность нейтрального элемента}
	\textbf{Утверждение:} нейтральный элемент единственный, если он существует. \\
	\textbf{Доказательство:} пусть у нас все таки есть $e_1$ и $e_2$, оба нейтральные элементы, тогда $e_1 = e_2$ : $$e_1 = e_1 \cdot e_2= e_2 $$
	что и требовалось доказать.
	
	\subsection{Группа}
	\textbf{Определение:} группой называют моноид, в котором $\forall m \in M \ \exists m^{-1}: m \cdot m^{-1} = e$. То есть для каждого элемента найдется обратный, который при проведении операции с таковым вернет нейтральный элемент. Пример: $(\mathbb{Q} \setminus \{0\}, \cdot)$ или $(\mathbb{Z}, +)$.
	
	\subsection{Единственность обратного элемента}
	\textbf{Утверждение:} обратный элемент единственный, если он существует. \\
	\textbf{Доказательство:} пусть у нас есть обратные к $g$ элементы $h_1$ и $h_2$, тогда $h_1 = h_2:$  $$h_1 = h_1 \cdot (g \cdot h_2) = (h_1 \cdot g) \cdot h_2 = h_2 $$
	что и требовалось доказать.
	
	\section{Определение кольца. Закон нуля. Умножение на -1. Коммутативность и ассоциативность. Примеры колец.}
	
	\subsection{Кольцо}
	\textbf{Определение:} кольцом называется множество $M$ с определенными двумя бинарными операциями, например: $\cdot, +$. Так, что $(M, +)$ - абелева группа, а $(M, \cdot)$ - полугруппа. А также операции связаны законом дистрибутивности: $\forall a, b, c \in M \ a \cdot (b + c) = a \cdot b + a\cdot c, (a + b) \cdot c = a\cdot c + b\cdot c$.
	
	\subsection{Ассоциативное кольцо}
	\textbf{Определение:} ассоциативным кольцом называется то, у которого вторая операция также ассоциативна.
	
	\subsection{Кольцо с единицей} 
	\textbf{Определение:} кольцом с единицей называется то, у которого для второй операции определен нейтральный элемент.
	
	\subsection{Коммутативное кольцо} 
	\textbf{Определение:} коммутативным кольцом называется то, у которого вторая операция коммутативна.
	
	\subsection{Примеры:}
	\begin{itemize}
		\item $(\mathbb{Z}, +, \cdot)$ - коммутативное ассоциативное с единицей
		\item $(3\mathbb{Z}, +, \cdot)$ - коммутативное ассоциативное
	\end{itemize}
	
	\section{Делимость в коммутативном кольце. Делители нуля. Область целостности. Ассоциированные элементы. Критерий ассоциированности в области целостности. Сокращение в области целостности.}
	
	\subsection{Делимость}
	\textbf{Определение:} $a | b, a,b \in R$ означает $\exists c \in R \ b = a \cdot c$. 
	
	\subsection{Делители нуля}
	\textbf{Определение:} $r \neq 0 \in R$ - делитель нуля, если $\exists s \neq 0 \in R: r\cdot s = 0$ 
	
	\subsection{Область целостности}
	\textbf{Определение:} кольцо $R$ называется областью целостности, если в нем нет делителей нуля.
	
	\subsection{Ассоциированные элементы}
	\textbf{Определение:} $a, b \neq 0 \in R$ называют ассоциированными $(a \sim b)$, если $a | b$ и $b | a$.
	\\ 
	\textbf{Утверждение:} Пусть $R$ - область целостности, $a, b \in R$ и $a \sim b$, тогда $\exists u \in R^{\times}: a = u \cdot b$.
	\\
	\textbf{Доказательство:} \begin{align*}
		a \sim b &\implies 
		\begin{cases} 
			a|b \\ b|a  
		\end{cases} 
		\implies 
		\begin{cases}
			\exists c\in R: b = a \cdot c \\ 
			\exists d \in R: a = b \cdot d 
		\end{cases} \implies \\ 
		&\implies b = b \cdot c \cdot d 
		\implies b(1 - c \cdot d) = 0 \implies \\ 
		&\implies c \cdot d = 1 
		\implies c, d \in R^{\times} 
		\implies u = d
	\end{align*}
	
	\subsection{Сокращение в области целостности}
	\textbf{Утверждение:} $R$ - область целостности и $a \neq 0$, тогда $(a\cdot b = a\cdot c) \implies (b = c)$
	\\ \textbf{Доказательство:} $a\cdot(b-c) = 0 \implies b -c = 0 \iff b = c$
	
	\section{Алгоритм Евклида в евклидовой области. Существование НОД в
		евклидовой области. Теорема о линейном представлении НОД в евклидовой области.}
	
	\subsection{Евклидова область}
	\textbf{Определение:} область целостности $R$, в которой существует функция нормы $N:R \setminus \{0\} \to \mathbb{N}$ такая что: $N(a \cdot b) \geq N(a)$ и $\forall a, b \in R \ \exists c, r \in R, b\neq 0: a = b\cdot c + r : N(r) < N(b)$, или $r = 0$.
	
	\subsection{НОД}
	\textbf{Определение:} $R$ - коммутативное кольцо с единицей, $a, b \in R$, $gcd(a, b) = d \in R$: 1) $d|a, d|b$ 2) $\forall c \in R: c | a, c | b  \ c | d$. \\
	\textbf{Свойство:} если $gcd(a, b)$ существует, то единственный с точностью до ассоциированного. Если $gcd(a, b) = d = d'$, то $d | d'$ и $d' | d$, а если $R$ - область целостности, то $d = d' \cdot u, u \in R^{\times}$ \\
	\textbf{Доказательство:} $d | a, d | b, d'|a, d' |b \implies d' | d$, аналогично для $d$
	
	\subsection{Алгоритм Евклида}
	\textbf{Теорема:} $R$ - евклидово кольцо $a, b \in R$ и они одновременно не ноль, тогда: $$1) \exists gcd(a, b) = d \in R \ 2) \exists x, y \in R: d = ax + by $$
	\textbf{Доказательство:} 
	1 случай: $a = 0$ или $b = 0$, тогда $gcd(0, b) = b$ или $gcd(a, 0) = a$, так как $a = gcd(a, 0) = a \cdot 1 + 0 \cdot 1$ и аналогично для $b$ 
	2 случай: $a, b \neq 0$ 
	$a = q_0b + r_0 \land N(r_0) < N(b)$ 
	$b = q_1r_0 + r_1 \land N(r_1) < N(r_0)$ 
	$r_0 = q_2r_1 + r_2 \land N(r_2) < N(r_1)$ 
	$\dots$ 
	$r_{n-1} = q_{n-1}r_n + r_{n+1} \land N(r_{n+1}) < N(r_n)$ 
	на каком то шаге станет:
	$r_n = q_nr_{n + 1} \implies r_{n+2} = 0$ \\
	\textbf{Утверждение:} $r_{n+1} = gcd(a, b)$ \\
	\textbf{Доказательство:}
	\begin{enumerate}
		\item Первое условие НОД \\
		$r_{n+1} | a \land r_{n+1} | b$: 
		$r_{n+1} | r_n \implies r_{n+1} |r_{n-1} \implies \dots \implies r_{n + 1} | b \implies r_{n+1} | a$ 
		\item Существует линейная комбинация с $r_{n+1}$ \\
		$r_0 = a - q_0b$ \\
		$r_1 = b - q_1r_0 = b - q_1(a - q_0 b) = b - q_1a + q_0q_1b = -q_1a + b(1 + q_0q_1b)$, пусть коэффициент $a$ это $S_1$, а коэффициент с $b$ это $T_1$. \\
		$r_2 = S_2a + T_2b$ \\
		$\dots$ \\
		$r_{n + 1} = S_{n + 1} a + T_{n + 1}b$
		\item Также нам нужно доказать второе условие НОД \\
		$\forall c \in R: c | a, c | b \implies c | S_{n + 1}a \land c | T_{n + 1}b \implies c | r_{n + 1} \implies gcd(a, b) = r_{n + 1}$ 
	\end{enumerate}
	
	\section{Евклидова область является областью главных идеалов.}
	\subsection{Область главных идеалов}
	\textbf{Определение:} область целостности $R$ - область главных идеалов, если для любого идеала $I \subseteq R$ существует такой элемент $x \in I: (x)  = I$ 
	\subsection{Идеал}
	\textbf{Определение:} $$I - \text{идеал} \ M \iff \begin{cases}
		I \subseteq M, I\neq \varnothing \\
		\forall a,b \in I \ a + b \in I \\
		\forall m \in M \ m\cdot i \in I
	\end{cases}$$
	\subsection{Главный идеал}
	\textbf{Определение:} $$ (x) - \text{главный идеал} \ M \iff (x) = \{xm : m \in M\}$$ \\
	\textbf{Утверждение:} любое евклидовое кольцо $R$ - область главных идеалов. \\
	\textbf{Доказательство:} Пусть $I$ - идеал в $R$, $I \neq \{0\}$, $S = \{N(x): x \in I\} \implies \exists N(d)$ - наименьшая норма. Можно утверждать, что $(d) = I$:
	\begin{enumerate}
		\item левое включение: $d \in I \implies rd \in I \implies (d) \subset I$ 
		\item правое включение: Пусть $x \in I$, разделим $x$ на $d$ с остатком: $x = dq \cdot r$, нам нужно, чтобы $r = 0$. У нас допустима ситуация, что $N(r) < N(d)$, но такая ситуация на самом деле невозможна, так как $r = x - dq \implies r \in I$, но $N(d)$ - наименьшая норма для элементов из $I$, поэтому $r$ может быть только нулем, тогда $I \subset (d)$ 
	\end{enumerate}
	
	\section{В области главных идеалов каждая возрастающая цепочка идеалов
		стабилизируется.}
	\subsection{Теорема}
	Пусть $R$ - область главных идеалов, также у нас есть цепочка вложенных идеалов: $I_1 \subset I_2 \subset I_3 \dots$, тогда $\exists N \in \mathbb{N}: I_N = I_{N  + 1} = I_{N + 2} \dots$, то есть цепочка стабилизируется. Такое явление еще называют Нетеровым кольцом. \\
	\textbf{Доказательство:} Так как $R$ - ОГИ, каждый $I_k = (i_k)$, то есть каждый идеал главный. Допустим у нас есть $I_{\infty} = \bigcup\limits_{k = 1} I_k$, теперь покажем, что $I_{\infty}$ это тоже идеал:
	\begin{enumerate}
		\item Замкнутость по сложению \\
		Пусть $a, b \in I_{\infty}$, тогда $\exists n_1, n_2 \in \mathbb{N}: a \in I_{n_1}, b \in I_{n_2}$, тогда не умаляя общности: $a, b \in I_{n_2} \implies a + b \in I_{n_2} \in I_{\infty}$ 
		\item Замкнутость по умножению на элемент из кольца \\
		Пусть $a \in I_{\infty}$ и $r \in R$, тогда $\exists n \in \mathbb{N}: a \in I_n$, так как $I_n$ - идеал: $ar \in I_n \in I_{\infty}$
	\end{enumerate}
	Так как $I_{\infty}$ - идеал и он лежит в $R$, $\exists x \in I_{\infty}: (x) = I_{\infty}$, тогда $\exists N: x \in I_N$, но мы можем сказать, что $I_N = I_{\infty}$, тут не очевидно только левое включение: $\forall y \in I_{\infty} \implies y = sx, s \in R$, $x \in I_N \implies sx  = y \in I_N \implies I_N = I_{\infty}$, тогда начиная с $N$: $I_N = I_{N + 1} = I_{N + 2} \dots$ 
	
	\section{В области главных идеалов необратимый элемент раскладывается
		в произведение неприводимых.}
	\subsection{Неприводимый элемент}
	\textbf{Определение:} $r \in R \setminus R^{\times}$ - неприводимый, если из разложения $r = st$ следует, что хотя бы один из элементов обратимый, то есть $s \in R^{\times} \lor t \in R^{\times}$
	\subsection{Теорема}
	\textbf{Утверждение:} $R$ - ОГИ, $r \in R \setminus R^{\times}$, тогда существует разложение $r$ в произведение неприводимых единственное с точностью до перестановки множителей и ассоциированности.
	\\ \textbf{Доказательство:} Пусть у нас есть множество $S = \{(x): x -$ не раскладывается в произведение неприводимых, $ x \neq 0, x \in R\setminus R^{\times}\}$, мы предположим, что оно не пусто, по доказательству конечности цепочки вложенных идеалов мы можем взять максимальный элемент $(z)$, тогда мы можем сказать, что $z$ не может быть неприводимым, так как тогда оно будет иметь разложение на неприводимые $z = z$, что противоречит нашему множеству $S$. Тогда из этого следует, что $z$ - приводимый, тогда он имеет разложение $z = st, s,t \in R \setminus R^{\times}$, но тогда $z$ делится на $s$ или $t$, а это то же самое, что $(z) \subset (s)$ или $(z) \subset (t)$, но $(z)$ - минимальный по включению идеал в $S \implies (s), (t) \notin S$, тогда по определению множества $S$ $s$ и $t$ имеют разложение на неприводимые: $s = q_1 \cdot q_2 \dots q_n$, $t = p_1 \cdot p_2 \dots p_n$, но тогда $z = (q_1 \cdot q_2 \dots q_n) \cdot (p_1 \cdot p_2 \dots p_n)$, следовательно $z$ имеет разложение на неприводимые, получаем противоречие, следовательно $S = \varnothing$, тогда любой элемент в Области главных идеалов имеет разложение на неприводимые.
	
	\section{Основная теорема арифметики в области главных идеалов.}
	\subsection{Простой элемент}
	\textbf{Определение:} элемент $r \in R \setminus{R^{\times}}$ называется простым, если из того, что $r | ab \implies r|a \lor r|b$ 
	\subsection{Лемма}
	\textbf{Утверждение:} $R$ - ОГИ, $r$ - неприводим, тогда $r$ - простой. \\
	\textbf{Доказательство:} Пусть $r | ab$, рассмотрим все возможные линейные комбинации $(r, a)$, тогда $\exists d \in R: (d) = (r, a) \implies d | r, d | a$ , $r \in (r,a) \implies r \in (d) \implies \ \exists s \in R: r = sd \implies$ 1) $d \in R^{\times}$ 2) $r \sim d$.
	\begin{enumerate}
		\item Если $d \in R^{\times} \implies (d) = R \implies d = 1$, получается что наша линейная комбинация это $1 = xr + ya$, мы можем домножить все на $b$, тогда $b = xrb + yab$, тогда $r | xrb, r | ab \implies r | b$
		\item Если $r \sim d$, то по определению ассоциированности $r | d$, а $d | a \implies r | a$
	\end{enumerate}
	То есть для любого случая $r$ - простой
	
	\subsection{Единственность разложения на неприводимые}
	Пусть $r = p_1 \cdot p_2 \dots p_n = q_1 \cdot q_2 \dots q_n$ все элементы неприводимы, а значит простые. Рассмотрим $p_1$: $p_1 | (p_1 \cdot p_2 \dots p_n), p_1 | (q_1 \cdot q_2 \dots q_m)$. $m = n$, иначе бы какие то элементы были единицами, чтобы не нарушить равенство, тогда бы нарушилось условие, что все элементы неприводимы. Затем находим среди элементов $q_i$ такой, что $p_1 \sim q_i$, меняем местами $q_i$ и $q_1$, после всех подобных перестановок получим $p_1 \sim q_1, p_2 \sim q_2, \dots, p_n \sim q_n$.
	
	\section{Фактор-кольцо, корректность операций. Кольцо классов вычетов.
		Обратимые элементы в \texorpdfstring{$\mathbb{Z}/n\mathbb{Z}.$}{Z/nZ}}
	
	\subsection{Отношение эквивалентности}
	\textbf{Определение:} $\sim$ - отношение эквивалентности на множестве $M$, если $\sim \ \subset M \times M$, в котором для пары $(a, b)$ выполняется заданное условие $a \sim b$, а также отношение обладает: 1) рефлексивность, 2) симметричность, 3) транзитивность.
	\subsection{Фактор-кольцо}
	\textbf{Определение:} $R$ - кольцо, $I$ - идеал на этом кольце, тогда $R / I$ это фактор по отношению эквивалентности такому, что $a \sim b \iff a - b \in I$, то есть элементы различаются на элемент из идеала. Тогда $R/I$ это множество из классов эквивалентности $[x] = \{y \in R: x \sim y\}$ 
	\subsection{Корректность операций}
	Докажем, что $[a] + [b] = [a + b] = [a' + b'], a \sim a', b \sim b'$: Нам нужно доказать что $a + b \sim a' + b'$. $(a + b) - (a' + b') = (a - a') + (b - b') \in I$ \\
	Докажем что $[ab] = [a'b']$: $ab - a'b' = ab - ab' + ab' - a'b' = a(b - b') + b'(a - a') \in I$
	\subsection{Кольцо классов вычетов}
	\textbf{Определение:} Кольцо классов вычетов это кольцо $\mathbb{Z}$ факторизованное по главному идеалу $(n)$. Оно содержит классы эквивалентности, в которых $a \sim b \implies a = b \mod  n$. То есть классы эквивалентности содержат элементы, которые дают одинаковый остаток при делении на $n$. 
	\subsection{Обратимые элементы в \texorpdfstring{$\mathbb{Z}/n\mathbb{Z}.$}{Z/nZ}}
	Обратимыми в Z/nZ являются такие $x: gcd(x, n) = 1$, это очень легко доказывается, мы можем представить линейную комбинацию для нод: $xk + yn = 1 \mod n \implies xk = 1 \mod n \implies k$ - обратимый элемент для $x$. То есть в $\mathbb{Z} / p\mathbb{Z}$ все ненулевые элементы обратимы, соответственно это уже не просто кольцо, а поле
	
	
	\section{Гомоморфизм колец. Проекция на фактор - гомоморфизм. Ядро
		гомоморфизма - идеал. Образ гомоморфизма - подкольцо.}
	\subsection{Гомоморфизм колец}
	\textbf{Определение:} отображение $f: R \to S$ если:
	\begin{enumerate}
		\item $f(r_1 + r_2) = f(r_1) + f(r_2)$
		\item $f(r_1 \cdot r_2) = f(r_1) \cdot f(r_2)$
		\item $f(1_R) = 1_S$, требование единицы не является обязательным, если кольца без единицы.
	\end{enumerate}
	
	\subsection{Проекция на фактор кольца}
	\textbf{Утверждение:} Если $I$ - идеал кольца $R$, то $\pi : R \to R/I$ - гомоморфизм. \\
	\textbf{Доказательство:} $\pi(a)$ просто отправляет a в его класс эквивалентности $[a]$.
	\begin{enumerate}
		\item $\pi(a + b) = [a + b] = [a] + [b] = \pi(a) + \pi(b)$
		\item $\pi(ab) = [ab] = [a] \cdot [b] = \pi(a) \cdot \pi(b)$
		\item $\pi(1) = [1] = 1_{R/I}$
	\end{enumerate}
	
	\subsection{Ядро гомоморфизма}
	\textbf{Определение:} для гомоморфизма $f: R \to S$ ядро $\ker f = \{r \in R: f(r) = 0\}$, то есть в ядре лежат все элементы $R$, которые превратятся в ноль. \\
	\textbf{Утверждение:} $\ker f$ - идеал в кольце $R$.
	\\ \textbf{Доказательство:} 
	\begin{enumerate}
		\item $ker f \neq \varnothing$, т.к всегда в ядре находится ноль. 
		\item пусть $a, b \in \ker f$, тогда $f(a) = f(b) = 0 \implies f(a + b) = f(a) + f(b) = 0 + 0 = 0 \in \ker f$
		\item пусть $a \in \ker f$, $b \in R$, тогда $f(ra) = f(r) \cdot f(a) = f(r) \cdot 0 = 0 \in \ker f$
	\end{enumerate}
	Все три условия для того что ядро - идеал доказаны.
	\subsection{Образ гомоморфизма}
	\textbf{Определение:} образ гомоморфизма $f: R \to S$ - $Im \ f \{s \in S: \exists r \in R: f(r) = s\}$
	\\ \textbf{Утверждение:} такой образ - подкольцо $S$.
	\\ \textbf{Доказательство:}
	\begin{enumerate}
		\item $f(0_R) = f(0_R + 0_R) = f(0_R) + f(0_R) \implies f(0_R) = 0_S$
		\item Пусть $x, y \in Im \ f \implies \exists a,b \in R: x = f(a), y = f(b)$, тогда $x - y =f(a) - f(b) = f(a - b) \in Im \ f$ - замкнуто по вычитанию, значит есть для каждого обратный элемент по сложению
		\item $xy = f(a) \cdot f(b) = f(ab) \in Im \ f$ - замкнуто по умножению
	\end{enumerate}
	Следовательно $Im \ f$ - подкольцо $S$.
	
	
	\section{Критерий инъективности гомоморфизма колец через определение
		ядра. Гомоморфизм является изоморфизмом тттк он биекция.}
	\subsection{Критерий инъективности}
	\textbf{Утверждение:} $f : R \to S$ - гомоморфизм, тогда $f$ - инъекция $\iff \ker f = \{0\}$ \\
	\textbf{Доказательство:} 
	\begin{enumerate}
		\item достаточность: \\
		Пусть $\exists a \in \ker f: f(a) = 0$, но $f(0) = 0$, тогда по инъективности $f$ $a = 0$
		\item необходимость: \\
		Пусть $a, b \in R$ и $f(a) = f(b) \implies f(a) - f(b) = 0 \implies f(a - b) = 0 \implies$
		$\implies a - b \in \ker f \implies a - b = 0 \implies a = b \implies f -$ инъекция
	\end{enumerate}
	\subsection{Изоморфизм}
	\textbf{Определение:} Кольца $R$ и $S$ называют изоморфными если для гомоморфизма $f: R \to S$, существует гомоморфизм $g : S \to R$, такой что у $f$ есть обратный слева $g \circ f = id_R$ и обратный справа $f \circ g = id_S$.
	\subsection{Гомоморфизм являтся изоморфизмом}
	\textbf{Утверждение:} если гомоморфизм $f: R \to S$ - изоморфизм $\iff f$ - биекция. \\
	\textbf{Доказательство:} 
	\begin{enumerate}
		\item достаточность: \\
		если $f$ - изоморфизм, то по определению существует $g: S \to R$ такой, что $g \circ f = id_R \implies f$ - инъекция и $f \circ g = id_S \implies f$ - сюръекция, тогда $f$ - биекция.
		\item необходимость: \\
		если $f$ - биекция, то существует обратное отображение $f^{-1}$, нам надо доказать, что это гомоморфизм (наличие обратного левого и правого исходит из биективности $f$). Возьмем любые $y_1, y_2$ такие что $x_1 = f^{-1}(y_1), x_2 = f^{-1}(y_2)$:
		\begin{enumerate}
			\item сложение: \\
			нам нужно доказать, что $f^{-1}(y_1 + y_2) = f^{-1}(y_1) + f^{-1}(y_2)$. 
			$y_1 + y_2 = f(x_1 + x_2)$. $f^{-1}(y_1 + y_2) = f^{-1}(f(x_1 + x_2)) = x_1 + x_2 = f^{-1}(y_1) + f^{-1}(y_2)$
			\item умножение: \\
			нам нужно доказать, что $f^{-1}(y_1 \cdot y_2) = f^{-1}(y_1) \cdot f^{-1}(y_2)$. 
			$y_1 \cdot y_2 = f(x_1 \cdot x_2)$. $f^{-1}(y_1 \cdot y_2) = f^{-1}(f(x_1 \cdot x_2)) = x_1 \cdot x_2 = f^{-1}(y_1) \cdot f^{-1}(y_2)$
		\end{enumerate}
	\end{enumerate}
	Выходит, что $f^{-1}$ - гомоморфизм, тогда $f$ - изоморфизм.
	
	\section{Лемма о пропуске гомоморфизма через фактор-кольцо. Теорема об
		изоморфизме.}
	\subsection{Лемма}
	\textbf{Утверждение:} $f: R \to S$ - гомоморфизм, $I$ - идеал $R$, тогда $\exists \overline{f}: R/I \to S$ - гомоморфизм, такой, что $\overline{f} \circ \pi = f \iff I \subset \ker f$. \\
	\textbf{Доказательство:}
	\begin{enumerate}
		\item достаточность: \\
		$\forall a \in I \ f(a) = \overline{f}(\pi(a)) = \overline{f}(0) = 0 \implies a \in \ker f$
		\item необходимость: \\
		Пусть $[r] \in R/I$, тогда $\overline{f}([r]) = f(r), r \in [r]$. 
		Но нам необходимо доказать корректность:
		Пусть $r, r' \in [r]$, $f(r') = \overline{f}([r]) = f(r)$? Это действительно так: $f(r) - f(r') = f(r - r') \in I$ и равно нулю, по предположению, что $I \in \ker f$.
		Почему $\overline{f}$ - гомоморфизм:
		\begin{itemize}
			\item $\overline{f}([1]) = f(1) = 1$
			\item $\overline{f}([a] + [b]) = \overline{f}([a + b]) = f(a + b) = f(a) + f(b) = \overline{f}([a]) + \overline{f}([b])$
			\item $\overline{f}([a] \cdot [b]) = \overline{f}([a \cdot b]) = f(a \cdot b) = f(a) \cdot f(b) = \overline{f}([a]) \cdot \overline{f}([b])$
		\end{itemize}
	\end{enumerate}
	Также покажем коммутативность: Пусть $r \in R$, $\overline{f}(\pi(r)) = \overline{f}([r]) = f(r)$
	\subsection{Первая теорема об изоморфизме}
	\textbf{Утверждение:} $f: R \to S$ - гомоморфизм, тогда $R/\ker f \xrightarrow{\cong} Im \ f$ \\
	\textbf{Доказательство:} существует гомоморфизм $R \to Im \ f$, тогда, чтобы построить $R/\ker f \to Im \ f$ необходимо и достаточно по лемме, чтобы $\ker f \subset \ker f$, что всегда правда. Обозначим этот гомоморфизм $\overline{f}$, теперь нам надо доказать, что он биективен:
	\begin{enumerate}
		\item инъекция: \\
		$\ker f = \{[r] \in R/\ker f: \overline{f}([r]) = 0\} = \{[r] : f(r) = 0\} = \{0\}$, тогда это инъекция
		\item Сюръекция: \\
		Пусть $s \in Im \ f$, тогда существует $r \in R: f(r) = s$, тогда $\overline{f}([r]) = f(r) = s$, получается сюръекция.
	\end{enumerate}
	$\overline{f}$ - биекция, а значит и изоморфизм. 
	
	\section{Китайская теорема об остатках в \texorpdfstring{$\mathbb{Z}$}{Z}.}
	\textbf{Утверждение:} $gcd(m, n) = 1$, тогда $\sigma: \mathbb{Z}/mn\mathbb{Z} \xrightarrow{\cong} (\mathbb{Z}/n\mathbb{Z}) \times (\mathbb{Z}/m\mathbb{Z})$ \\
	\textbf{Доказательство:} $\sigma(k) = (k \mod n, k \mod m)$ 
	Покажем, что $\sigma$ - гомоморфизм:
	\begin{enumerate}
		\item Сложение: $\sigma(k + k') = ([k + k']_n, [k + k']_m) = ([k]_n + [k']_n, [k]_m + [k']_m) = ([k]_n, [k]_m) + ([k']_n, [k']_m) = \sigma(k) + \sigma(k')$
		\item Умножение: Умножение: $\sigma(k \cdot k') = ([k \ cdot k']_n, [k \cdot k']_m) = ([k]_n \cdot [k']_n, [k]_m \cdot [k']_m) = ([k]_n, [k]_m) \cdot ([k']_n, [k']_m) = \sigma(k) \cdot \sigma(k')$
		\item Сохраняет единицу: $\sigma([1]_{mn}) = ([1]_{mn} \mod n, [1]_{mn} \mod m) = ([1]_n, [1]_m)$
	\end{enumerate}
	Определим $\ker \sigma = \{k \in \mathbb{Z}/nm\mathbb{Z}: k = 0 \mod n \land k = 0 \mod m\}$, пусть у нас $k \in \ker \sigma$, тогда $k$ делится на $m$ и $n$, тогда $k = 0 \mod nm \implies \sigma$ - инъекция. Так как у нас одинаковое количество элементов в $\mathbb{Z}/nm\mathbb{Z}$ и $\mathbb{Z}/n\mathbb{Z} \times \mathbb{Z}/m\mathbb{Z}$ и $\sigma$ - инъекция, каждый элемент слева получает уникальный справа, при этом каждый слева имеет связь, получается левое полностью покрывает правое, тогда это сюръекция и биекция соответственно. Если $\sigma$ - биекция, то мы уже ранее доказывали, что $\sigma$ - изоморфизм.
	
	\section{Количество элементов в произведении колец. Обратимые элементы
		в произведении колец. Мультипликативность функции Эйлера.}
	\subsection{Произведение колец}
	\textbf{Определение:} $R, S$ - кольца, тогда $R \times S = \{(r, s): r\in R, s \in S\}$ с операциями:
	\begin{enumerate}
		\item сложение: $(r_1, s_1) + (r_2, s_2) = (r_1 + r_2, s_1 + s_2)$
		\item умножение: $(r_1, s_1) \cdot (r_2, s_2) = (r_1r_2, s_1s_2)$
	\end{enumerate}
	
	Докажем, что это кольцо:
	\begin{enumerate}
		\item Ассоциативность сложения: \\
		$(r_1, s_1) + ((r_2, s_2) + (r_3, s_3)) = (r_1, s_1) + (r_2 + r_3, s_2 + s_3) = (r_1 + r_2 + r_3, s_1 + s_2 + s_3) = (r_1 + r_2, s_1 + s_2) + (r_3 + s_3) = ((r_1, s_1) + (r_2, s_2)) + (r_3, s_3)$
		\item Существование нейтрального элемента: \\
		$(r_1, s_1) + (0_R, 0_S) = (r_1, s_1) \implies (0_R, 0_S)$ - нейтральный элемент по сложению. 
		\item Существование обратных по сложению: \\
		$\forall r \in R, s \in S, -r, -s$ - обратные по сложению, тогда $(r, s) + (-r, -s) = (0_R, 0_S)$.
		\item Ассоциативность умножения: \\
		$((r_1, s_1) \cdot (r_2, s_2)) \cdot (r_3, s_3) = (r_1 \cdot r_2 \cdot r_3, s_1 \cdot s_2 \cdot s_3) = (r_1, s_1) \cdot ((r_2, s_2) \cdot (r_3, s_3))$
		\item Дистрибутивность: \\
		$(r_1, s_1) \cdot ((r_2, s_2) + (r_3, s_3)) = (r_1, s_1) \cdot (r_2 + r_3, s_2 + s_3) = (r_1r_2 + r_1r_3, s_1s_2 + s_1s_3) = (r_1r_2, s_1s_2) + (r_1r_3, s_1s_3) = (r_1, s_1) \cdot (r_2, s_2) + (r_1, s_1) \cdot (r_3, s_3)$
	\end{enumerate}
	Доказали все аксиомы кольца, значит это кольцо.
	\subsection{Количество элементов в произведении колец}
	\textbf{Утверждение:} $R, S$ - конечные кольца, тогда $|R \times S| = |R| \cdot |S|$ \\
	\textbf{Доказательство:} допустим зафиксируем $r \in R$, тогда пар $(r, s_i)$ будет $|S|$, и всего таких возможных пар с разными $r$ будет $|R| \cdot |S|$.
	\subsection{Обратимые элементы в произведении колец}
	\textbf{Утверждение:} $R, S$ - кольца, тогда $(R \times S)^{\times} = R^{\times} \times S^{\times}$ \\
	\textbf{Доказательство:} 
	\begin{enumerate}
		\item правое включение: \\
		$(r, s) \in R^{\times} \times S^{\times} \implies (r, s)^{-1} = (r^{-1}, s^{-1})$
		\item левое включение: \\
		$(r, s) \in (R \times S)^{\times} \implies (t, u) \in R \times S: (r, s) \cdot (t, u) = (1, 1) \implies (rt, su) = (1, 1) \implies r, s$  - обратимы.
	\end{enumerate}
	\subsection{Функция эйлера}
	\textbf{Определение:} $\varphi: \mathbb{N} \to \mathbb{N}$, $\varphi(n) = |\{k : 1\leq k \leq n, gcd(k, n) = 1\}$, такая функция считает сколько взаимно простых элементов с $n$, которые меньше $n$.
	\subsection{Мультипликативность функции Эйлера}
	\textbf{Утверждение:} Пусть $gcd(m, n) = 1$, тогда $\varphi(mn) = \varphi(m) \cdot \varphi(n)$ \\
	\textbf{Доказательство:} так как в кольце вычетов $\mathbb{Z}/n\mathbb{Z}$ элемент $k$ обратим только когда $gcd(k, n) = 1$, мы можем сказать, что $|(\mathbb{Z}/n\mathbb{Z})^{\times}| \cong \varphi(n)$. 
	Теперь мы можем сказать, что $\varphi(mn) = |(\mathbb{Z}/mn\mathbb{Z})^{\times}| \overset{\text{КТО}}{=} |(\mathbb{Z}/m\mathbb{Z} \times \mathbb{Z}/n\mathbb{Z})^{\times}| = |(\mathbb{Z}/m\mathbb{Z})^{\times} \times (\mathbb{Z}/n\mathbb{Z})^{\times}| = |(\mathbb{Z}/m\mathbb{Z})^{\times}| \cdot |(\mathbb{Z}/n\mathbb{Z})^{\times}| = \varphi(m) \cdot \varphi(n)$   
	
	\section{Вычисление функции Эйлера. Теорема Эйлера.}
	\subsection{Вычисление функции Эйлера} $\varphi(n) = \varphi(\prod p_i^k) = \prod \varphi(p_i^k)$, теперь нам надо понять как быстро считать $\varphi(p^k)$:
	$\varphi(p) = p -1$ 
	$\varphi(p^2)=p^2 - \frac{p^2}{p}$, здесь такая логика, всего у нас элементов $p^2$, мы можем мысленно разделить их на $p$ блоков по $p$ элементов так, что у нас в каждом блоке только последний будет делиться на $p$, например тут это $p, 2p \dots p^k$, их будет столько же, сколько у нас блоков, а их $\frac{p^2}{p}$. 
	В общем случае это $\varphi(p^k) = p^k - \frac{p^k}{p} = p^k - p^{k - 1} = p^{k - 1} \cdot (p - 1)$. 
	\subsection{Теорема Эйлера}
	\textbf{Утверждение:} если $gcd(a, n) = 1$, то $a^{\varphi(n)} = 1 \mod n$ \\
	\textbf{Доказательство:} пусть у нас есть множество взаимно простых с n $\{r_1, r_2, \dots r_{\varphi(n)}\}$, все эти числа обратимы в $\mathbb{Z}/n\mathbb{Z}$, домножим каждый такой элемент на $a$, скажем, что у нас количество элементов осталось прежним, так как если $ar_i = ar_j$, то мы можем сократить на $a$, потому что оно обратимо по модулю $n$, тогда $r_i = r_j \implies i = j$.
	$r_1 \cdot r_2 \dots \cdot r_{\varphi(n)} = ar_1 \cdot ar_2 \dots \cdot ar_{\varphi(n)} \mod n \implies r_1 \cdot r_2 \dots \cdot r_{\varphi(n)} =$
	$= a^{\varphi(n)} \cdot (r_1 \cdot r_2 \dots \cdot r_{\varphi(n)}) \mod n$ , Так как у нас произведение элементов $r$ состоит из обратимых элементов мы можем сократить на него, тогда $a^{\varphi(n)} = 1 \mod n$.
	
	\section{Построение кольца многочленов над полем, деление с остатком. Деление на двучлен. Теорема Безу. Следствие о количестве различных
		корней многочлена.}
	\subsection{Кольцо многочленов}
	\textbf{Определение:} $R[x] = \{(r_1, r_2, \dots, r_n, 0, 0 \dots): r_i \in R \land \exists n\}$ - кольцо многочленов. $(r_1, r_2, \dots, r_n, 0, 0, \dots)$ - финитная последовательность, пусть $e_i \in R[x] = (0, 0, 0, \dots, 1, 0, \dots)$, где единица стоит на $i$-том месте. Тогда $r \cdot e_i = (0, 0, 0, \dots, r, 0, \dots)$, $e_i \cdot e_j = e_{i+j}$. Также мы можем сложить две такие последовательности: $a, b \in R[x]$, $m > n$ , $(a_0, a_1, \dots, a_n, 0, \dots) + (b_0, b_1, \dots, b_m, \dots) = (a_0 + b_0, a_1 + b_1, \dots, a_n + b_n, \dots, b_m, 0, \dots)$. Тогда мы также можем разложить нашу последовательность: $(r_0, r_1, \dots, r_n, 0, \dots) = (r_0, 0, \dots) + (0, r_1, 0, \dots) + (0, 0, \dots, r_n, 0, \dots)$, что то же самое, что и $r_0 \cdot e_0 + r_1 \cdot e_1 + \dots + \dots + r_n \cdot e_n$. Обозначим $e_0$ как $1$, $e_1$ как $x$, $e_k = e_1 \cdot e_1 \cdot e_1 \cdot \dots \cdot e_1 = x^k$.
	Тогда мы можем сказать, что $R[x] = \{r_0 + r_1x + \dots + r_nx^n: r_i \in R \land \exists n\}$.
	\subsection{Поле}
	\textbf{Определение:} кольцо $R$ называется полем, если $\forall r \neq 0 \in R$ существует $r^{-1}: r \cdot r^{-1} = 1$
	\subsection{Кольцо многочленов над полем}
	\textbf{Утверждение:} $(k[x])^{\times} = k^{\times} = k \setminus \{0\}$ \\
	\textbf{Доказательство:} $\deg p(x)$ - номер последнего ненулевого элемента последовательности
	$p(x) \in k[x]$ пусть также $\exists q(x) \in k[x]: p(x) \cdot q(x) = 1 = 1 + 0x + 0x^2 \dots$  
	$(p_0 + p_1x + \dots + p_nx^n) \cdot (q_0 + q_1x + \dots + q_mx^m) = \dots + p_nq_mx^{n+m} = 1$ 
	У нас два варианта: либо $p_nq_m = 0$, но они оба не ноль по условию, а делителей нуля не существует в поле $k$. Тогда $n + m = 0$, в таком случае $n = m = 0$, тогда $p(x) = p_0 \in k$ и $q(x) = q_0 \in k$.
	
	\textbf{Утверждение:} $k[x]$ - область целостности \\
	\textbf{Доказательство:} $p(x), q(x) \in k[x], p(x) \neq 0, q(x) \neq 0$, $\deg(pq) = deg(p) + deg(q)$.
	$\deg(p(x)q(x)) = \deg(p(x)) + \deg(q(x)) = 0 \implies \deg(p(x)) = \deg(q(x)) = 0$
	$p_0 \cdot q_0 = 0$, но в $k$ нет делителей нуля - противоречие.
	
	\textbf{Утверждение:} $k[x]$ - евклидова область. \\
	\textbf{Доказательство:} $k[x]$ - область целостности из прошлого доказательства. Теперь нам надо доказать существование функции $N$. 
	$N(p) = \deg p \ \forall p \in k[x], p \neq 0$, проверим свойства нормы:
	\begin{enumerate}
		\item $N(p(x)q(x)) \geq N(p(x))$, равенство только когда $q(x)$ - обратим
		\item Покажем, что $\forall f(x), g(x) \in k[x], g(x) \neq 0 \ \exists ! q(x), r(x) \in k[x]$, так что: 
		$f(x) = g(x) \cdot q(x) + r(x) :$ либо $r(x) = 0$, либо $N(r(x)) < N(g(x))$ или $\deg r < \deg g$
	\end{enumerate}
	Рассмотрим два случая деления многочленов с остатком: \\
	1 случай: \\
	$\deg f < \deg g$, то $q(x) = 0, r(x) = f(x)$, тогда $f(x) = 0\cdot g(x) + f(x)$, пример: $x = 0\cdot x^2 + x$ \\
	2 случай: \\
	Пусть $\deg f(x) \geq g(x)$, $f(x) = a_nx^n + \dots + a_0, g(x) = b_mx^m + \dots + b_0, n \geq m$. \\
	тогда: 
	$f_1(x) = f(x) - \frac{a_n}{b_m}x^{n - m}g(x), \deg f_1(x) < n$ \\
	$f_2(x) = f_1(x) - ?g(x),$ если подставить $f_1(x)$, то $f_2(x) = f(x) - \frac{a_n}{b_m}x^{n - m}g(x) - ?x^{?}g(x)$, можно выносить $g(x)$ за скобки \\
	$\dots$ \\ 
	$f_k(x) = f(x) - (?)g(x)$ до тех пор, пока $\deg f_k < \deg g$, тогда $f_k(x) = 0\cdot g(x) + f_k(x)$ \\
	Получается: $f(x) = (?)g(x) + f_k(x)$, тогда у нас два варианта: $f_k = 0$ либо $\deg f_k < \deg g$ \\
	Также докажем единственность такого деления: \\
	Пусть существуют $q_1, q_2, r_1, r_2 \in k[x]: f(x) = q_1(x)g(x) + r_1(x) = q_2(x)g(x) + r_2(x)$ \\
	$g(q_1 - q_2) = r_2 - r_1$, Пусть $r_2 - r_1 \neq 0, r_1 \neq 0, \deg r_1 > \deg r_2$, тогда $\deg ((q_1 - q_2)g) = \deg(q_1 - q_2) + \deg g > \deg (r_1) \geq (r_2 - r_1)$, но у нас $\deg((q_1 - q_2)g) = \deg(r_1 - r_2)$, получаем противоречие. Тогда $r_1 - r_2 = 0 \implies r_1 = r_2$. Получается, что $(q_1 - q_2) = 0 \implies q_1 = q_2$. 
	\subsection{Деление на двучлен}
	Пусть мы хотим разделить $f(x)$ на $g(x) = (x - a)$ 
	$f(x) = q(x)(x-a) + r(x)$, стоит заметить, что в данном случае $r(x)$ это просто $r \in k$, так как либо он ноль, либо его степень меньше 1, а значит это тоже константа.
	\subsection{Теорема Безу}
	\textbf{Теорема:} при делении $f(x)$ на $g(x) = x - a$ $r(x) = f(a)$ \\
	\textbf{Доказательство: } $f(x) = q(x)(x-a) + r(x)$, $f(a) = q(x)\cdot 0 + r(x) \implies f(a) = r(x)$
	\subsection{Первое следствие:} $\exists \alpha: f(\alpha) = 0$, тогда $f(x) = q(x)(x - \alpha)$, то есть $f(x)$ делится на $(x - \alpha)$
	\subsection{Второе следствие:} 
	\textbf{Утверждение:} Пусть $\deg f = n$, тогда количество различных корней $f(x) \leq n$. \\
	\textbf{Доказательство:} Пусть $a_1, a_2, \dots a_n$ - корни $f(x)$, тогда $f(x) = g(x)(x - a_1)$, подставим $a_2$: $f(a_2) = g(a_2)(a_2 - a_1) = 0 \implies g(a_2) = 0$, тогда по первому следствию $g(x)$ можно дальше разложить так, чтобы $f(x) = g(x)(x - a_1)(x - a_2)$. таким образом мы можем разложить все корни так, что $\deg$ такого разложения будет равен количеству корней + степени оставшегося $g(x)$, что меньше чем исходный $n$, так как у нас при разложении всегда уменьшается степень $g(x)$.
	
	\section{Теорема о формальном и функциональном равенстве многочленов.
		Антипример для конечного поля.}
	\subsection{Характеристика}
	\textbf{Определение:} Пусть $R$ - кольцо, Характеристика $R$ ($char \ R$) - это наименьшее натуральное $n$, такое что сумма $n$ единиц равняется нулю. Если такого $n$ не существует, то $char \ R = 0$. Из этого следует, что $R$ - бесконечно.
	\subsection{Формальное и функциональное равенство многочленов}
	\textbf{Определение:} $f(x), g(x) \in k[x]$. Они равны формально, если просто равны в кольце многочленов. И равны функционально, если $\forall c \in k$  $f(c) = g(c)$. \\
	\textbf{Теорема:} из функционального равенства следует формальное, если $char \ k = 0$ \\
	\textbf{Доказательство:} Возьмем $h(x) = f(x) - f(g) \in k[x]$, если у нас не формально равны, то $h \neq 0$, тогда допустим $\deg h = n$. Но тогда $h$ имеет бесконечное количество различных корней, хотя по следствию из теоремы безу он имеет только не больше $n$ различных корней - противоречие, значит $f(x) = g(x)$ \\
	\textbf{Антипример:} если мы берем конечное поле, например $\mathbb{Z}/p\mathbb{Z}$, то в случае например многочленов $f(x) = x^p$ и $g(x) = x$ у нас они совпадают функционально, так как по малой теореме ферма $a^{p-1} = 1 \mod p \implies a^p = a \mod p$, но при этом формально они отличаются.
	
	\section{Сумма, произведение и пересечение идеалов. Проверка. Связь произведения и пересечения в общем случае и в случае взаимно простых (комаксимальных) идеалов.}
	\subsection{Сумма идеалов} 
	\textbf{Определение:} $R$ - кольцо, $I, J$ - идеалы в $R$, тогда $I + J = \{i + j: i \in I, j \in J\}$ \\
	\textbf{Утверждение:} $I + J$ - идеал. \\
	\textbf{Доказательство:} 
	\begin{enumerate}
		\item $(i_1 + j_1) + (i_2 + j_2) = (i_1 + i_2) + (j_1 + j_2)  \in I + J$
		\item $r \in R$, $r(i_1 + j_1) = ri_1 + rj_1 \in I + J$
	\end{enumerate}
	Пример: в $\mathbb{Z}$ сложим $12\mathbb{Z} + 18\mathbb{Z}$ = $\{18m + 12n\}_{n,m \in \mathbb{Z}}$ = $6\mathbb{Z}$ ($gcd(12, 18)$) так работает для всех целых чисел
	\subsection{Пересечение Идеалов} 
	\textbf{Определение:} $I \cap J$ - пересечение идеалов \\
	\textbf{Утверждение:} $I \cap J$ - идеал \\
	\textbf{Доказательство:} пусть $a, b \in I \cap J$
	\begin{enumerate}
		\item $a + b \in I, a + b \in J \implies a + b \in I \cap J$
		\item $r \in R,$ $ra \in I, ra \in J \implies ra \in I \cap J$
	\end{enumerate}
	Пример: $R$ = $\mathbb{Z}$ , $I  = 12\mathbb{Z}, J = 18\mathbb{Z}$, тогда $I \cap J = 36 \mathbb{Z}$ или же Нок(12, 18)
	\subsection{Умножение идеалов}
	\textbf{Определение:} $I \cdot J = \{\sum\limits_{k = 1} i_kj_k: i_k \in I, j_k \in J\}$ \\
	\textbf{Утверждение:} $I \cdot J$ - идеал \\
	\textbf{Доказательство:} 
	\begin{enumerate}
		\item $\sum\limits_k i_k j_k  + \sum\limits_s i'_s j'_s \in I \cdot J$
		\item $r \in R$ $r \sum\limits_k i_kj_k = \sum\limits_k (ri_k) \cdot j_k \in I\cdot J$
	\end{enumerate}
	Пример: $I \cdot J = \{\sum\limits_k 12a_k \cdot 18b_k\} = \{216 \sum\limits_k a_k b_k\} = 216 \mathbb{Z}$
	\subsection{Взаимная простота идеалов}
	\textbf{Определение:} $I, J$ - взаимнопросты, если $I + J = R \iff 1 \in I + J$
	\subsection{Связь умножения и пересечения}
	\textbf{Лемма:} 1)$I\cdot J \subset I \cap J$ и если $I, J$ - взаимно просты, то 2)$I\cap J = I \cdot J$ \\
	\textbf{Доказательство:} 
	\begin{enumerate}
		\item Пусть $\sum\limits_k i_kj_k \in I\cdot J$, тогда $i_k j_k \in I \implies \sum\limits_k i_k j_k \in I$, аналогично для $J$, тогда сумма лежит в пересечении $I\cap J$
		\item Пусть $a \in I\cap J$, так как $I, J$ - взаимно просты, $\exists i \in I, j \in J: 1 = j + i \implies a = ai + aj$, тогда пусть $a \in J$ для $i$ и $a \in I$ для $j$, тогда a - сумма элементов $i_k j_k$, а значит она лежит в $I\cdot J$
	\end{enumerate}
	
	\section{Китайская теорема об остатках для колец.}
	\subsection{Теорема:} $I, J$ - взаимно простые идеалы в кольце $R$, тогда $R/I\cdot J \cong R/I \times R/J$ \\
	\textbf{Доказательство:} 
	$\varphi: R \to R/I \times R/J$ - гомоморфизм \\
	$r \mapsto ([r]_I, [r]_J)$ \\
	$Im \ \varphi$: Докажем, что $\varphi$ - сюръекция.\\
	т.к $I, J$ - взаимно простые, $1 \in I + J$ или $\exists i \in I, j \in J: 1 = i + j$ \\
	$\varphi(i) = ([i]_I, [i]_J) = (0, [1 - j]_J) = (0, 1)$ \\
	$\varphi(j) = ([1 - i]_I, [j]_J) = (1, 0)$ \\
	Пусть $([r]_I, [r]_J)$ = $\varphi(rj + si) \implies Im \ \varphi = R/I \times R/J$ \\
	$\ker \varphi$: $r \in \ker \varphi$, $\varphi(r) = ([r]_I, [r]_j) = (0, 0)$ $\iff r \in I \land r \in J \iff r \in I\cap J$, но так как $I, J$ - взаимно просты $I\cap J = I \cdot J$ \\
	Получается $\ker \varphi = I \cdot J$. \\
	Тогда по теореме об изоморфизме: \\
	$\varphi: R/\ker \varphi \xrightarrow{\cong} Im \ \varphi$, что то же самое, что $\varphi: R/I\cdot J \xrightarrow{\cong} R/I \times R/J$, что и требовалось доказать.
	\subsection{Более общая формулировка}
	\textbf{Теорема:} Пусть $R$ - кольцо с идеалами $I_1, I_2, \dots, I_n$, все попарно взаимно просты, тогда $R/I_1 \cdot \dots \cdot I_n \cong R/I_1 \times \dots \times R/I_n$ \\
	\textbf{Доказательство:} Берем просто пару идеалов $I = I_1$ и $J = I_2 \cdot \dots \cdot I_n$, доказываем как для двух идеалов. Но нужно показать, что $I, J$ - взаимно простые. Сделаем линейные комбинации для $I_1$ и каждого другого идеала, затем перемножим их, тогда получится $1 = (i_1 + y_2) (i_2 + y_3)\dots (i_{n-1} + y_n) = y_2y_3 \dots y_n +$ какой то элемент из идеала $I$, тут $y_i$ - элемент $I_i$ идеала. Тогда мы составили линейную комбинацию для $I$ и $J$.
	
	\section{Формула Тейлора для многочлена. Критерий кратности корня.}
	\subsection{Формула Тейлора для многочлена}
	$f(x) \in k[x], f(x) = a_0 + a_1x + \dots + a_nx^n$, 
	$x = y+c, y = x - c$, 
	$f(x)=a_0 + a_1(y+c) + a_2(y+c)^2 + \dots + a_n(y+c)^n$.
	Пусть такой $f(x) = b_0 + b_1y + \dots + b_ny^n$, тогда $f(x) = b_0 + b_1(x - c) + b_2(x - c)^2 + \dots + b_n(x-c)^n$ \\
	\textbf{Определение:} $f'(x)$ - производная многочлена $f(x) \in k[x]$, если для $f(x) = a_0 + a_1x + \dots a_nx^n$, $f'(x) = a_1 + 2a_2x + 3a_3x^2 + \dots + na_nx^{n-1}$ \\
	\textbf{Утверждение:} $b_m = \frac{f^{(m)}(c)}{m!}$, если $char \ k = 0$ \\
	\textbf{Доказательство:} $f(x) = b_0 + b_1(x-c) + \dots + b_m(x-c)^m + \dots + b_n(x - c)^n$ , если мы возьмем производную $m$ раз, то все члены до $(x - c)^m$ занулятся, а $f^{(m)}(x) = m! \cdot b_m + ?(x - c)$, тогда при подстановке $f^{(m)}(c) = m! \cdot b_m \implies b_m = \frac{f^{(m)}(c)}{m!}$
	\subsection{Критерий кратности корня}
	\textbf{Определение:} $p(x) \neq 0 \in k[z]$, элемент $\alpha \in k$ называется корнем кратности $m \neq 1$, если $p(x) = (x - \alpha)^m \cdot g(x)$, где $g(\alpha) \neq 0$ \\
	\textbf{Определение:} корень $\alpha$ называют кратным, если $\alpha$ кратности $\geq 2$ \\
	\textbf{Утверждение:} $\alpha$ - кратный корень $p(x) \iff p(\alpha) = 0$ и $p'(\alpha) = 0$ \\
	\textbf{Доказательство:} 
	\begin{enumerate}
		\item Достаточность: $\alpha$ - кратный корень $p(x) \implies p(x) = (x - \alpha)^2\cdot h(x)$ \\
		$p(\alpha) = 0$ \\
		$p'(x) = 2(x-\alpha) \cdot h(x) + (x - \alpha)^2 \cdot h'(x) \implies p'(\alpha) = 0$
		\item Необходимость: $p(\alpha) = 0, p'(\alpha) = 0$ \\
		По теореме Безу: $p(x) = (x - \alpha) s(x)$, $p'(x) = s(x) + (x - \alpha)\cdot s'(x)$, но так как $p'(\alpha) = 0 \implies s(\alpha) = 0$, тогда снова по теореме безу: $s(x) = (x - \alpha) \cdot h(x)$, тогда $p(x) = (x - \alpha)^2 \cdot h(x) \implies \alpha$ - корень кратности $\geq 2$.
	\end{enumerate}
	\textbf{Утверждение:} $p(x) \neq 0 \in k[x]$, $\alpha$ - корень кратности $m \iff f(\alpha) = 0, f'(\alpha) = 0, \dots f^{(m-1)}(\alpha) = 0$ и $f^{(m)}(\alpha) \neq 0$
	
	\section{Количество корней с кратностями не больше степени многочлена.}
	\textbf{Теорема:} $p(x) \neq 0 \in k[x], \alpha_1,\dots, \alpha_n$ - различные корни кратности $l_1, \dots, l_n$ соответственно.
	Тогда $p(x) = (x - \alpha_1)^{l_1} \cdot (x - \alpha_2)^{l_2} \cdot \dots \cdot (x - \alpha_n)^{l_n} \cdot g(x), g(\alpha) \neq 0$ \\
	\textbf{Доказательство:} Так как $k[x]$ - евклидова область, это ОГИ, значит работает разложение на неприводимые, пусть $p(x) = u_1(x) \cdot \dots \cdot u_n(x)$ - разложение на неприводимые. Пусть $\alpha_1, \dots, \alpha_n$ - наши корни кратности $l_1, \dots, l_n$ соответственно, тогда каждый двучлен вида $(x - \alpha_i)$ забирает ровно $l_i$ элементов $u_i$. Тогда наш многочлен выглядит так: $p(x) = (x - \alpha_1)^{l_1} \cdot \dots (x - \alpha_n)^{l_n} \cdot h(x)$, где $h(\alpha_i) \neq 0 \forall i$. \\
	\textbf{Следствие:} $l_1 + l_2 + \dots + l_n \leq \deg p(x)$
	
	\section{Интерполяционная формула Лагранжа, единственность.}
	\subsection{Интерполяционная формула Лагранжа}
	Если мы хотим, чтобы наша кривая проходила через определенные точки $y_0, y_1, \dots, y_n \in k$, то мы можем построить многочлен $f(x) \in k[x]: \deg f \leq n, f(x_i) = y_i$. 
	$f(x) = \sum\limits_{i = 0}^n y_i l_i(x)$, где $l_i(x_j) \in k[x]$ - такой многочлен, который равен одному если $i = j$, то есть в этой точке мы должны подняться на $y_i$ вверх, иначе этот многочлен дает 0. $l_i(x) = \frac{(x - x_0)(x - x_1) \dots (x - x_{i-1})(x - x_{i+1}) \dots (x - x_n)}{(x_i - x_0) (x_i - x_1) \dots (x_i - x_{i - 1})(x_i - x_{i + 1}) \dots (x_i - x_n)}$, тогда $f(x_j) = \sum\limits_{i = 0}^n y_i \cdot l_i(x_j) = y_j$
	\subsection{Единственность}
	Пусть $h(x) = f(x) - g(x)$, где $f(x), g(x)$  - решают задачу интерполяции, тогда $h(x)$ будет иметь $n + 1$ корень при условии $\deg h \leq n$, что противоречит теореме безу, значит $h(x) = 0 \implies f(x) = g(x)$.
	
	\section{Критерий максимальности идеала. Критерий простоты идеала.}
	\subsection{Критерий простоты идеала}
	\textbf{Теорема:} $I \lhd R$ - простой $\iff R / I$ - область целостности \\
	\textbf{Доказательство:} 
	\begin{enumerate}
		\item Достаточность: \\
		$\left[a\right] \cdot \left[b\right] = 0$ в $R/I \implies \left[a \cdot b\right] = 0 \implies ab \in I \implies \left[ \begin{array}{l}
			a \in I \\
			b \in I
		\end{array}\right.$ \\ Получается, если $ab = 0$, то либо $a = 0$, либо $b = 0 \implies R/I$ - область целостности 
		\item Необходимость: \\
		$ab \in I \implies \left[ab\right] = \left[a\right] \left[b\right] = 0 \implies \left[\begin{array}{l}
			\left[a\right] = 0 \\
			\left[b\right] = 0
		\end{array}\right. \implies \left[\begin{array}{l}
		a \in I \\
		b \in I \\
		\end{array}\right.$
	\end{enumerate}
	\subsection{Критерий максимальности идеала}
	\textbf{Теорема:} $I \lhd R$ - максимален $\iff R / I$ - поле \\
	\textbf{Доказательство:} 
	\begin{enumerate}
		\item Достаточность: \\
		Пусть $\left[a\right] \in R/I$ и $\left[a\right] \neq 0$, мы хотим найти $\left[a\right]^{-1}$. $\left[a\right] \neq 0 \implies a \notin I$, введем идеал $J = I + \left(a\right), I \nsubseteqq J \implies J = R \implies 1 \in J \implies 1 \in I + \left(a\right) \implies 1 = i + ra \implies \left[1\right] = \left[ra\right] = \left[r\right] \left[a\right] \implies \left[r\right] = \left[a\right]^{-1} \implies R/I$ - поле
		\item Необходимость: \\
		$I \nsubseteqq J$, мы хотим показать, что $R = J$. Пусть $x \in J \setminus I \implies \left[x\right] \neq 0$. Введем $y: \left[x\right]\left[y\right] = 1 \implies \left[xy\right] = 1 \implies \left[xy - 1\right] = 0 \implies xy - 1 = i \in I \implies 1 = xy - i$, где $xy \in J, i \in J \implies 1 \in J \implies J = R$
	\end{enumerate}
	
	\section{ Квадратичные вычеты и невычеты. Символ Лежандра. Формула
		для символа Лежандра. Мультипликативность. Подсчет \texorpdfstring{$\left(\frac{-1}{p}\right)$}{(-1/p)}.}
	\subsection{Квадратичные вычеты и невычеты}
	Пусть мы хотим определить имеет ли уравнение $x^2 = a$ в поле $\mathbb{Z}/p\mathbb{Z}, a \neq 0, p \notin 2\mathbb{Z}$. Если такой $x$ существует, то $x^{p-1} = 1$ по малой теореме Ферма. \\ $1 = x^{p-1} = \left(x^2\right)^{\frac{p-1}{2}}=a^{\frac{p-1}{2}}$ \\
	$\left(\mathbb{Z}/p\mathbb{Z}\right)^{\times} = \left(\mathbb{Z}/p\mathbb{Z}\right)\setminus{\left\{0\right\}}$ является циклической группой по умножению: \\
	$\exists \gamma \in \left(\mathbb{Z}/p\mathbb{Z}\right):\forall b \in \left(\mathbb{Z}/p\mathbb{Z}\right)^{\times} \exists k \in \mathbb{N} \cup \{0\}: b = \gamma^k$ \\
	$\gamma^{p-1} = 1$ и $\gamma^m \neq 1 \forall m \in \left[1, p-2\right]$ \\
	Пусть $a^{\frac{p-1}{2}} = 1$ и $\exists k: a=\gamma^k$, $a^{\frac{p-1}{2}} = 1 = \gamma^{k\frac{p-1}{2}} \implies \frac{p-1}{2}k \ \vdots \ p-1 \iff \frac{k}{2} \in \mathbb{Z} \iff k \in 2\mathbb{Z}$. Пусть $k = 2n$, тогда если $x = \gamma^n$, то $x^2 = \gamma^{2n} = \gamma^k = a$ \\
	$a^{\frac{p-1}{2}}$ - решение $x^2 = 1$, т.к мы находимся в поле, то у этого уравнения не больше двух решений по теореме безу, оба решения мы знаем: $x=1, x=-1$ \\
	\textbf{Утверждение:} уравнение $x^2 = a$, где $a \neq 0$ имеет решения в $\mathbb{Z}/p\mathbb{Z} \iff a^{\frac{p-1}{2}} = 1$, иначе $a^{\frac{p-1}{2}} = -1$ \\
	\subsection{Символ Лежандра. Формула для символа Лежандра. Мультипликативность}
	\textbf{Определение:} символом Лежандра $\left(\frac{a}{p}\right)$ называется функция $\left(\mathbb{Z}/p\mathbb{Z}\right)^{\times} \to \{1, -1\}:$ 
	$$\left(\frac{a}{p}\right) = \begin{cases}
		1, x^2 = a \ \text{имеет решения} \\
		-1, x^2 = a \ \text{не имеет решений}
	\end{cases}$$
	$$\left(\frac{a}{p}\right) = a^{\frac{p-1}{2}}$$ 
	\textbf{Свойства:} 
	\begin{enumerate}
		\item $\left(\frac{a^2}{p}\right) = 1$
		\item $\left(\frac{ab}{p}\right) = \left(\frac{a}{p}\right) \left(\frac{b}{p}\right)$ - мультипликативность
	\end{enumerate}
	\subsection{Подсчет \texorpdfstring{$\left(\frac{-1}{p}\right)$}{(-1/p)}}
	$x^2 = -1 \mod p$,
	\begin{itemize}
		\item если имеет решение, то $\left(-1\right)^{\frac{p-1}{2}} = 1 \iff \frac{p-1}{2} \in 2\mathbb{Z} \iff p-1 \in 4\mathbb{Z} \iff p = 1 \mod 4$
		\item если не имеет решения, то $\left(-1\right)^{\frac{p-1}{2}} = -1 \iff \frac{p-1}{2} \not in 2\mathbb{Z} \iff \frac{p-1}{2} = 2m + 1, m \in Z \iff p-1 = 4m + 2 \iff p = 4m + 3 \mod p \iff p = 3 \mod p$
	\end{itemize} 
	$$\left(\frac{-1}{p}\right) - \begin{cases}
		1, p = 1 \mod 4 \\
		-1, p = 3 \mod 4
	\end{cases}$$
	\section{Формула Гаусса для символа Лежандра (с доказательством). Вычисление \texorpdfstring{$\left(\frac{2}{p}\right)$}{2/p}
	}
	\subsection{Лемма:} $a \in \left(\mathbb{Z}/p\mathbb{Z}\right), p \notin 2\mathbb{Z}$.
	Рассмотрим $\{a, 2a, 3a, \dots, \frac{p-1}{2} a\}$ по модулю $p = \{r_1, r_2,\dots, r_{p-1}\}, 1 < r_i < p-1$. Назовем такое множество $S$, пусть $L = |\{x\in S: x > \frac{p}{2}\}|$, тогда $\left(\frac{a}{p}\right) = \left(-1\right)^{L}$ \\
	\textbf{Доказательство:} 
	$S = \{u_1, \dots, u_L\} \cup \{v_{L+1}, \dots, v_{\frac{p-1}{2}}\}$, где $\frac{p}{2} < u_i < p - 1$ и $v_j < \frac{p}{2}$ \\
	Перемножим все элементы в $S$: \\
	$\prod\limits_{s\in S} s \underset{p}{\equiv} a \cdot 2a \cdot 3a \cdot \dots \cdot \frac{p-1}{2}a \underset{p}{\equiv} \left(\frac{p-1}{2}\right)!a^{\frac{p-1}{2}}$ \\
	Рассмотрим $\{p-u_1, p-u_2, \dots, p-u_L\}, 1 \leq p - u_i < \frac{p}{2}$ \\
	\textit{Лемма:} все элементы $\{p-u, \dots, p-u_L\} \cup \{v_{L+1}, \dots, v_{\frac{p-1}{2}}\}$ попарно различны \\
	\textit{Доказательство:} \begin{itemize} 
		\item 1 случай: $p - u_i = p - u_j \implies u_i = u_j \implies k_ia = k_ja \implies k_i = k_j \implies$ противоречие 
		\item 2 случай: $v_i  = v_j$ аналогично первому случаю
		\item 3 случай: $p - u_i = v_j \implies u_i + v_j \underset{p}{\equiv} 0 \implies k_ia + k_ja \underset{p}{\equiv} 0 \implies k_i + k_j \underset{p}{\equiv} 0$ но $k_i \in\left[1;\frac{p-1}{2}\right] \implies k_i + k_j \in \left[2, p-1\right]$
	\end{itemize}
	В $\{p - u_i, \dots, p-u_L\} \cup \{v_{L+1}, \dots, v_{\frac{p-1}{2}}\}$ все элементы $\in \left[1;\frac{p-1}{2}\right]$ \\
	Их $\frac{p-1}{2}$ штук и они все различны по нашей лемме, тогда они просто равны $\{1,2,\dots, \frac{p-1}{2}\}$ \\
	$\prod\limits_{r \in \{1, \dots, \frac{p-1}{2}\}}r = \left(\frac{p-1}{2}\right)! = \prod\limits_{i = 1}^{L} \left(p - u_i\right)\prod\limits_{j=L+1}^{\frac{p-1}{2}}v_j = \left(-1\right)^L \prod\limits_{i = 1}^{L}u_i \prod\limits_{j=L+1}^{\frac{p-1}{2}}v_j = \left(-1\right) ^ {L} \prod\limits_{s\in S} s = \left(\frac{p-1}{2}\right)a^{\frac{p-1}{2}}\left(-1\right)^L$ \\
	$\left(\frac{p-1}{2}\right)! \left(-1\right)^L = \left(\frac{p-1}{2}\right)! a^{\frac{p-1}{2}}$, так как $\left(\frac{p-1}{2}\right)! \neq 0 \mod p$ на них можно сократить: $$a^{\frac{p}{2}} = \left(-1\right)^L \implies \left(\frac{a}{p}\right) = \left(-1\right)^L$$
	\subsection{Вычисление \texorpdfstring{$\left(\frac{2}{p}\right)$}{(2/p)}}
	$\left\{2, 4, 6, \dots, p-1\right\} = S$ \\
	$L = \frac{p-1}{2} - \lfloor\frac{p}{4}\rfloor$ \\
	$\left(\frac{2}{p}\right) = \left(-1\right)^L$ \\
	Пусть $p = 1 \mod 8 \iff p = 1 + 8t$ \\
	$L = 4t - 2t \in 2\mathbb{Z}$ \\
	Пусть $p = 3 \mod 8 \iff p = 3 + 8t$ \\
	$L = 1 + 4t - 2t \notin 2\mathbb{Z}$ \\
	Пусть $p = 5 \mod 8 \iff p = 5 + 8t$ \\
	$L = 2 + 4t - 2t - 1 \notin 2\mathbb{Z}$ \\
	Пусть $p = 7 \mod 8 \iff p = 7 + 8t$ \\
	$L = 3 + 4t - 1 - 2t \in 2\mathbb{Z}$ \\
	Можно подобрать функцию, которая ведет себя таким же образом: $\frac{p^2 - 1}{8}$
	$$\left(\frac{2}{p}\right) = \left(-1\right)^{\frac{p^2-1}{8}}$$
	\section{Квадратичный закон взаимности*.}
	\textbf{Формулировка:} $$\left(\frac{q}{p}\right) = \left(-1\right)^{\frac{p-1}{2} \cdot \frac{q-1}{2}} \cdot \left(\frac{p}{q}\right), p \neq q$$
	\section{Теорема Гаусса. Теорема о мультипликативной группе поля.}
	\subsection{Теорема Гаусса}
	\textbf{Теорема:} $$ n = \sum\limits_{d | n, d \geq 1}\varphi\left(d\right), \forall n \geq 1$$
	\textbf{Доказательство:} \\ $S = \left\{1, \dots, n\right\} = \bigcup\limits_{d | n, d \geq 1}S_d$ \\ $S_d = \left\{k \in S: \gcd\left(k, n\right) = d\right\}$, $S_d \cap S_d' \neq \varnothing \implies d = d'$ \\
	$|S_d| = |\left\{k \in S : \gcd \left(\frac{k}{d}, \frac{n}{d}\right) = 1\right\} = \varphi\left(\frac{n}{d}\right)$ \\
	$n = |S| = \sum\limits_{d | n, d \geq 1}|S_d| = \sum\limits_{d | n, d \geq 1}\varphi\left(\frac{n}{d}\right) = \sum\limits_{d | n, d \geq 1}\varphi\left(d\right)$
	\subsection{Теорема о мультипликативной группе поля}
	\textbf{Теорема:} пусть $F$ - конечное поле, тогда $F^\times = F \setminus \{0\}$ - циклическая группа, то есть $\exists a \in F^\times : \forall g \in F^\times \exists k \in \mathbb{N}: g = a^k$ \\
	\textbf{Доказательство:} $G = F^\times, \psi\left(d\right) =$ количество элементов $G$ порядка $d$ $$g \in G \text{ имеет порядок } d, \text{ если } g^d = 1, \text{ но } g^k \neq 1 \forall k \in \left[1;d-1\right]$$ Пусть мы нашли $h \in G$ порядка $d$, рассмотрим $<h> = \{1, h, h^2, h^3, \dots, h^{d-1}\}$ - сколько тут элементов порядка $d$? $$h^k \in <h> \text{ имеет порядок } d \iff \left(h^k\right)^m \text{ закрасит всю группу}$$
	Когда $\left(h^k\right)^m = h$? $h^{km}=h^1 \iff km = 1 \mod d \iff \gcd \left(k, d\right) = 1$ Получается в $<h>$ ровно $\varphi\left(d\right)$ элементов, которые имеют порядок $d$. Если мы находим хотя бы один элемент порядка $d$, то автоматически находим таких $\varphi\left(d\right)$ штук. \\
	Любой элемент $<h>$ удовлетворяет уравнению $x^d=1$. Так как мы существуем в поле, у этого уравнения $\leq d$ различных решений, значит $<h>$ - все решения $x^d = 1$. \\
	Пусть $t$ - элемент порядка $d$, тогда $t$ решает $x^d = 1 \implies t \in <h>$ $$\psi\left(d\right) = \left[\begin{array}{l}
		0, \text{ если не нашли} \\
		\varphi\left(d\right), \text{ иначе}
	\end{array}\right.\implies \psi\left(d\right) \leq \varphi\left(d\right)$$ $n = \sum\limits_{d | n} \psi\left(d\right) \leq \sum\limits_{d | n}\varphi\left(d\right) = n \implies \psi\left(d\right) = \varphi\left(d\right) \forall d | n$ \\
	Для $d = n \psi\left(n\right) = \varphi(n) \geq 1 \implies$ нашли хотя бы 1 элемент порядка $n$ \\
	\textbf{Лемма:} $G$ - конечная группа, $|G| = n$, $\alpha \in G$, тогда $ord \ \alpha | n$ \\
	\textbf{Доказательство:} $C = <\alpha>$, $|C| = ord \ \alpha$, $G = \bigcup\limits_{g \in G} gC$, это действительно равенство, т.к $\forall g \in G g \in gC$. $$gC \cap g'C = \left[\begin{array}{l}
		\varnothing \\
		gC = g'C
	\end{array}\right.$$ Пусть $z \in gC \cap g'C$ \\ $z = gc_1 = g'c_2 \implies g = g'c_2c^{-1}_1$ \\
	Пусть $h \in gC \implies h = gc = g'c_2c^{-1}_1c \in g'C \implies gC \subset g'C$, аналогично можно доказать в обратную сторону, тогда $gC = g'C$ \\
	$n = |G| = |C \sqcup g_1C \sqcup g_2C \sqcup \dots \sqcup g_kC| = |C| + |g_1C| + \dots + |g_kC| = ord \ \alpha + ord \ \alpha + \dots + ord \ \alpha = \left(k + 1\right) ord \ \alpha \implies ord \ \alpha | n$
	\section{Критерий максимальности для идеалов в k[x]. Построение поля
		комплексных чисел. Алгебраическая запись.}
	\subsection{Критерий максимальности для идеалов в k[x]}
	$k\left[x\right]$ - ОГИ \\
	\textbf{Утверждение: } идеал $\left(f\right) \text{ - максимален } \iff f \text{ - неприводим}$ \\
	\textbf{Доказательство:} \begin{enumerate}
		\item Достаточность: \\
		Пусть $f = g \cdot h$ и $\deg g \geq 1, \deg h \geq 1$, $\left(f\right) \nsubseteqq \left(g\right)$, так как $g \in \left(g\right)$ и $ g \notin \left(f\right)$, но $\left(g\right) \notin R \implies \left(f\right)$ - не максимален, противоречие, следовательно $f$ - неприводим. 
		\item Необходимость: \\
		Пусть $\left(f\right) \nsubseteqq J, J$ - главный идеал $\implies J = \left(g\right)$
		$$\left(f\right) \nsubseteqq \left(g\right) \implies g | f \implies \left[\begin{array}{l}
			g \equiv const \ \neq 0 \implies \left(g\right) k\left[x\right] \\
			g \sim f \implies \left(f\right) = \left(g\right) \text{, но по условию не равны}
		\end{array}\right.$$
	\end{enumerate}
	\subsection{Построение поля комплексных чисел. Алгебраическая запись.}
	Возьмем кольцо многочленов и факторизуем его по неприводимому $x^2 + 1$, таким образом получается новое поле комплексных чисел, это действительно поле, так как идеал максимален.
	$$ R\left[x\right]/\left(x^2+1\right)  = \mathbb{C}$$
	$$ \mathbb{C} = \left\{a + bx: a,b \in \mathbb{R}\right\}$$ 
	И если заменить $x$ на $i$ то получим алгебраическую запись: $a + bi$
	\section{Классификация автоморфизмов \texorpdfstring{$\mathbb{C}$}{C} над \texorpdfstring{$\mathbb{R}$}{R}, модуль комплексного числа. Его мультипликативность. Оценка суммы.}
	\subsection{Классификация автоморфизмов \texorpdfstring{$\mathbb{C}$}{C} над \texorpdfstring{$\mathbb{R}$}{R}} 
	\textbf{Определение:} автоморфизмом $\mathbb{C}$ над $\mathbb{R}$ называется изоморфизм $$\sigma: \mathbb{C} \to \mathbb{C}: \forall r \in \mathbb{R} \sigma\left(r\right) = r$$
	\textbf{Утверждение:} существует два автоморфизма $\mathbb{C}$ над $\mathbb{R}$: $id, -$, $Gal(\mathbb{C}/\mathbb{R}) \overset{\sim}{=} \mathbb{Z}/2\mathbb{Z}$ \\
	\textbf{Доказательство:} Пусть $\sigma$ - автоморфизм $\mathbb{C}$ над $\mathbb{R}$ \\
	$\sigma\left(a + ib\right) = \sigma\left(a\right) + \sigma\left(i\right)\sigma\left(b\right) = a + \sigma\left(i\right)b$ \\
	$\sigma\left(i\right)^2 = \sigma\left(i^2\right) = \sigma(-1) = -1$
	$$ \text{так как } \sigma\left(i\right) \text{ - решение } x^2 = -1 \sigma\left(i\right) = \left[\begin{array}{l}
		i \implies id \\
		-i
	\end{array}\right.$$
	$$ \text{сопряжение $\overline{a + ib}$ : } \sigma\left(a + ib\right) = a - ib$$
	Свойство: $z \in \mathbb{C}, \overline{z} = z \iff z \in \mathbb{R}$
	\subsection{Модуль комплексного числа. Его мультипликативность. Оценка суммы}
	Хотим построить отображение $\mathbb{C} \to \mathbb{R}$
	$\frac{z + \overline{z}}{2} - $ Re$z$, вещественная часть \\
	$\frac{z - \overline{z}}{2i} - $ Im$z$, мнимая часть \\
	$\overline{z \cdot \overline{z}} = z \cdot \overline{z} \implies z \cdot \overline{z} \in \mathbb{R}$
	$z \cdot \overline{z} = (a +ib)(a - ib) = a^2 + b^2 \geq 0$
	$|z| = \sqrt{z \cdot \overline{z}}$ \\
	\textbf{Свойства модуля:}
	\begin{enumerate}
		\item $|z_1 \cdot z_2| = \sqrt{z_1z_2\overline{z_1}\overline{z_2}} = \sqrt{z_1\overline{z_1}} \cdot \sqrt{z_2\overline{z_2}} = |z_1| \cdot |z_2|$
		\item $|z_1 + z_2| \leq |z_1| + |z_2|$ - неравенство треугольника для векторов $\left(a, b\right)$ и $\left(c, d\right)$, если $z_1 = a + ib, z_2 = c + id$
	\end{enumerate}
	\section{Основная теорема алгебры*.}
	\textbf{Теорема: } $$p\left(z\right) \in \mathbb{C}\left[z\right], \deg p \geq 1 \implies \exists z_0 \in \mathbb{C} : p\left(z_0\right) = 0$$
	\textbf{Следствие: } $$ p \in \mathbb{C}\left[z\right], \deg p = n \implies p \text{ имеет ровно $n$ корней}$$
\end{document}